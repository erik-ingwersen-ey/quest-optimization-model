%% Generated by Sphinx.
\def\sphinxdocclass{report}
\documentclass[letterpaper,10pt,english]{sphinxmanual}
\ifdefined\pdfpxdimen
   \let\sphinxpxdimen\pdfpxdimen\else\newdimen\sphinxpxdimen
\fi \sphinxpxdimen=.75bp\relax

\PassOptionsToPackage{warn}{textcomp}
\usepackage[utf8]{inputenc}
\ifdefined\DeclareUnicodeCharacter
% support both utf8 and utf8x syntaxes
  \ifdefined\DeclareUnicodeCharacterAsOptional
    \def\sphinxDUC#1{\DeclareUnicodeCharacter{"#1}}
  \else
    \let\sphinxDUC\DeclareUnicodeCharacter
  \fi
  \sphinxDUC{00A0}{\nobreakspace}
  \sphinxDUC{2500}{\sphinxunichar{2500}}
  \sphinxDUC{2502}{\sphinxunichar{2502}}
  \sphinxDUC{2514}{\sphinxunichar{2514}}
  \sphinxDUC{251C}{\sphinxunichar{251C}}
  \sphinxDUC{2572}{\textbackslash}
\fi
\usepackage{cmap}
\usepackage[T1]{fontenc}
\usepackage{amsmath,amssymb,amstext}
\usepackage{babel}



\usepackage{times}
\expandafter\ifx\csname T@LGR\endcsname\relax
\else
% LGR was declared as font encoding
  \substitutefont{LGR}{\rmdefault}{cmr}
  \substitutefont{LGR}{\sfdefault}{cmss}
  \substitutefont{LGR}{\ttdefault}{cmtt}
\fi
\expandafter\ifx\csname T@X2\endcsname\relax
  \expandafter\ifx\csname T@T2A\endcsname\relax
  \else
  % T2A was declared as font encoding
    \substitutefont{T2A}{\rmdefault}{cmr}
    \substitutefont{T2A}{\sfdefault}{cmss}
    \substitutefont{T2A}{\ttdefault}{cmtt}
  \fi
\else
% X2 was declared as font encoding
  \substitutefont{X2}{\rmdefault}{cmr}
  \substitutefont{X2}{\sfdefault}{cmss}
  \substitutefont{X2}{\ttdefault}{cmtt}
\fi


\usepackage[Bjarne]{fncychap}
\usepackage{sphinx}

\fvset{fontsize=\small}
\usepackage{geometry}


% Include hyperref last.
\usepackage{hyperref}
% Fix anchor placement for figures with captions.
\usepackage{hypcap}% it must be loaded after hyperref.
% Set up styles of URL: it should be placed after hyperref.
\urlstyle{same}

\addto\captionsenglish{\renewcommand{\contentsname}{Contents:}}

\usepackage{sphinxmessages}
\setcounter{tocdepth}{2}



\title{Quest Optimization Model}
\date{Jan 15, 2021}
\release{1.2}
\author{Ernst \& Young}
\newcommand{\sphinxlogo}{\vbox{}}
\renewcommand{\releasename}{Release}
\makeindex
\begin{document}

\pagestyle{empty}
\sphinxmaketitle
\pagestyle{plain}
\sphinxtableofcontents
\pagestyle{normal}
\phantomsection\label{\detokenize{index::doc}}
\begin{sphinxadmonition}{note}{Note:}
This documentation is still under development. Its final contents, look and feel might change considerably.
\end{sphinxadmonition}

\noindent{\hspace*{\fill}\sphinxincludegraphics[scale=0.25]{{quest_logo}.png}\hspace*{\fill}}



Documentation of the optimization algorithm created to optimize Quest Diagnostics business units inventories.


\chapter{How the Model Works}
\label{\detokenize{index:how-the-model-works}}
The optimization model comprises of multiple scripts that are combined to generate the transfer recommendation between BU’s report.

In summary, the process starts by \sphinxstylestrong{ingesting the necessary input files} this process is done at the {\hyperref[\detokenize{source/optimization.datatools:module-optimization.datatools.pipelines}]{\sphinxcrossref{\sphinxcode{\sphinxupquote{optimization.datatools.pipelines}}}}}.
There, we make sure that all the required fields are in place and also \sphinxstylestrong{validate critical columns}. Afterwards, if no fields present critical errors, the module proceeds to
\sphinxstylestrong{calculate all necessary columns required to formulate the optimization problem}. Then the outout from the data ingestion and manipulation is passed on to
{\hyperref[\detokenize{source/optimization:module-optimization.solspace}]{\sphinxcrossref{\sphinxcode{\sphinxupquote{optimization.solspace}}}}} that generates a \sphinxstylestrong{solution matrix for every unique item ID} in the inventory reports. Finally the output matrix is fed
to the {\hyperref[\detokenize{source/optimization.model:module-optimization.model.optimizer}]{\sphinxcrossref{\sphinxcode{\sphinxupquote{optimization.model.optimizer}}}}}. Finally, the \sphinxstylestrong{problem is formulated}, and based on the \sphinxstylestrong{defined objective function}; the model generates
the \sphinxstylestrong{report with transfer recommendations for every item ID}.


\chapter{What is This Documentation for?}
\label{\detokenize{index:what-is-this-documentation-for}}
This documentation is best used as reference guide. Here we specify how the optimization model \sphinxcode{\sphinxupquote{modules}}, \sphinxcode{\sphinxupquote{classes}}, and \sphinxcode{\sphinxupquote{functions}} work to generate the
transfer recommendations. It should not be used though, as a step\sphinxhyphen{}by\sphinxhyphen{}step guide on how an optimization model works.

Inside this documentation you will also find some configuration options that can be used for changing certain aspects of the model.


\section{Optimization Model}
\label{\detokenize{source/optimization:optimization-model}}\label{\detokenize{source/optimization::doc}}

\subsection{Model}
\label{\detokenize{source/optimization.model:model}}\label{\detokenize{source/optimization.model::doc}}
This module defines the necessary methods that combine all other modules into a single chain of operations.


\subsubsection{Main}
\label{\detokenize{source/optimization.model:module-optimization.model.main}}\label{\detokenize{source/optimization.model:main}}\index{module@\spxentry{module}!optimization.model.main@\spxentry{optimization.model.main}}\index{optimization.model.main@\spxentry{optimization.model.main}!module@\spxentry{module}}
Main  module for running optimization model. This module contains all methods that are necessary to run the optimization model.
\index{ModelOptimization (class in optimization.model.main)@\spxentry{ModelOptimization}\spxextra{class in optimization.model.main}}

\begin{fulllineitems}
\phantomsection\label{\detokenize{source/optimization.model:optimization.model.main.ModelOptimization}}\pysiglinewithargsret{\sphinxbfcode{\sphinxupquote{class }}\sphinxcode{\sphinxupquote{optimization.model.main.}}\sphinxbfcode{\sphinxupquote{ModelOptimization}}}{\emph{\DUrole{n}{df\_inventory}\DUrole{p}{:} \DUrole{n}{pandas.core.frame.DataFrame}}, \emph{\DUrole{n}{optimize\_what}\DUrole{p}{:} \DUrole{n}{str}}, \emph{\DUrole{n}{sku\_qty}\DUrole{p}{:} \DUrole{n}{int} \DUrole{o}{=} \DUrole{default_value}{0}}}{}
Bases: \sphinxcode{\sphinxupquote{object}}

Implements {\hyperref[\detokenize{source/optimization.model:optimization.model.optimizer.OptimizationModel}]{\sphinxcrossref{\sphinxcode{\sphinxupquote{optimization.model.optimizer.OptimizationModel}}}}} \sphinxstylestrong{class} for every \sphinxcode{\sphinxupquote{item ID}}.

This class defines the optimization problem for every item on the inventory report and
returns a combined DataFrame with all transfer recommendations.
\begin{quote}\begin{description}
\item[{Returns}] \leavevmode
Optimization Model transfer recommendations.

\item[{Return type}] \leavevmode
pd.DataFrame

\item[{Raises}] \leavevmode
\sphinxstyleliteralstrong{\sphinxupquote{TypeError}} \textendash{} Raises error if \sphinxcode{\sphinxupquote{df\_inventory}} is not of type \sphinxcode{\sphinxupquote{pd.DataFrame}}.

\end{description}\end{quote}
\index{\_get\_sku() (optimization.model.main.ModelOptimization method)@\spxentry{\_get\_sku()}\spxextra{optimization.model.main.ModelOptimization method}}

\begin{fulllineitems}
\phantomsection\label{\detokenize{source/optimization.model:optimization.model.main.ModelOptimization._get_sku}}\pysiglinewithargsret{\sphinxbfcode{\sphinxupquote{\_get\_sku}}}{}{}
Get SKU List from Inventory.
\begin{quote}\begin{description}
\item[{Parameters}] \leavevmode
\sphinxstyleliteralstrong{\sphinxupquote{df\_inventory}} (\sphinxstyleliteralemphasis{\sphinxupquote{pd.DataFrame}}) \textendash{} Inventory table.

\item[{Returns}] \leavevmode
List with unique SKUs.

\item[{Return type}] \leavevmode
list

\end{description}\end{quote}

\end{fulllineitems}

\index{\_get\_solver\_time() (optimization.model.main.ModelOptimization method)@\spxentry{\_get\_solver\_time()}\spxextra{optimization.model.main.ModelOptimization method}}

\begin{fulllineitems}
\phantomsection\label{\detokenize{source/optimization.model:optimization.model.main.ModelOptimization._get_solver_time}}\pysiglinewithargsret{\sphinxbfcode{\sphinxupquote{\_get\_solver\_time}}}{\emph{\DUrole{n}{sku}}}{}
Get maximum amount of time optimizer can use optimizing a single \sphinxcode{\sphinxupquote{Item ID}}.
\begin{quote}\begin{description}
\item[{Parameters}] \leavevmode
\sphinxstyleliteralstrong{\sphinxupquote{sku}} (\sphinxstyleliteralemphasis{\sphinxupquote{int}}) \textendash{} \sphinxcode{\sphinxupquote{Item ID}} of SKU we want to optimize.

\item[{Returns}] \leavevmode
Total time the solver can spend at a single \sphinxcode{\sphinxupquote{Item ID}}.

\item[{Return type}] \leavevmode
int

\end{description}\end{quote}

\end{fulllineitems}

\index{dynamic\_max\_solver\_time() (optimization.model.main.ModelOptimization method)@\spxentry{dynamic\_max\_solver\_time()}\spxextra{optimization.model.main.ModelOptimization method}}

\begin{fulllineitems}
\phantomsection\label{\detokenize{source/optimization.model:optimization.model.main.ModelOptimization.dynamic_max_solver_time}}\pysiglinewithargsret{\sphinxbfcode{\sphinxupquote{dynamic\_max\_solver\_time}}}{\emph{\DUrole{n}{sku}\DUrole{p}{:} \DUrole{n}{object}}}{}
Determine max time to solve optimization dynamically.

If \sphinxcode{\sphinxupquote{USE\_DYNAMIC\_TIME}} is enabled, this function returns
the time limit to spend at a single item ID.
\begin{quote}\begin{description}
\item[{Parameters}] \leavevmode
\sphinxstyleliteralstrong{\sphinxupquote{sku}} (\sphinxstyleliteralemphasis{\sphinxupquote{object}}) \textendash{} \sphinxcode{\sphinxupquote{Item ID}} that we’re going to optimize.

\item[{Returns}] \leavevmode
Value that ranges from \sphinxcode{\sphinxupquote{MAX\_TIME}} to \sphinxstylestrong{10\%} of \sphinxcode{\sphinxupquote{MAX\_TIME}}

\item[{Return type}] \leavevmode
int

\end{description}\end{quote}

\begin{sphinxadmonition}{tip}{Tip:}
To use this method to determine maximum amount of time the optimization model can send at each BU, you neet to set         the attribute \sphinxcode{\sphinxupquote{USE\_DYNAMIC\_TIME}} inside of {\hyperref[\detokenize{source/optimization:module-optimization.constants}]{\sphinxcrossref{\sphinxcode{\sphinxupquote{optimization.constants}}}}} to \sphinxcode{\sphinxupquote{True}}.
\end{sphinxadmonition}

\end{fulllineitems}

\index{optimizable\_inventory (optimization.model.main.ModelOptimization attribute)@\spxentry{optimizable\_inventory}\spxextra{optimization.model.main.ModelOptimization attribute}}

\begin{fulllineitems}
\phantomsection\label{\detokenize{source/optimization.model:optimization.model.main.ModelOptimization.optimizable_inventory}}\pysigline{\sphinxbfcode{\sphinxupquote{optimizable\_inventory}}\sphinxbfcode{\sphinxupquote{ = \{\textquotesingle{}\$ Value of Transfer\textquotesingle{}: {[}{]}, \textquotesingle{}Inventory Balance\textquotesingle{}: {[}{]}, \textquotesingle{}Item ID\textquotesingle{}: {[}{]}, \textquotesingle{}Items to Expire\textquotesingle{}: {[}{]}\}}}}
\end{fulllineitems}

\index{optimize() (optimization.model.main.ModelOptimization method)@\spxentry{optimize()}\spxextra{optimization.model.main.ModelOptimization method}}

\begin{fulllineitems}
\phantomsection\label{\detokenize{source/optimization.model:optimization.model.main.ModelOptimization.optimize}}\pysiglinewithargsret{\sphinxbfcode{\sphinxupquote{optimize}}}{\emph{\DUrole{n}{sku}\DUrole{p}{:} \DUrole{n}{int}}}{}
Run optimization for specific BU.

When running for entire inventory, this method is called
in a loop for every unique Item ID.

This method determines maximum time the optimization model has
to solve the problem, creates the solution space matrix and calls the
the optimization model solver that returns as output table with the
transfer recommendations.
\begin{quote}\begin{description}
\item[{Parameters}] \leavevmode
\sphinxstyleliteralstrong{\sphinxupquote{sku}} (\sphinxstyleliteralemphasis{\sphinxupquote{object}}) \textendash{} \sphinxcode{\sphinxupquote{Item ID}} that we’re going to optimize.

\item[{Returns}] \leavevmode
\sphinxstylestrong{opt\_df} \textendash{} Pandas dataframe with optimization results. If none could be             obtained, then function returns empty.

\item[{Return type}] \leavevmode
pd.DataFrame

\end{description}\end{quote}

\end{fulllineitems}

\index{run\_all() (optimization.model.main.ModelOptimization method)@\spxentry{run\_all()}\spxextra{optimization.model.main.ModelOptimization method}}

\begin{fulllineitems}
\phantomsection\label{\detokenize{source/optimization.model:optimization.model.main.ModelOptimization.run_all}}\pysiglinewithargsret{\sphinxbfcode{\sphinxupquote{run\_all}}}{}{}
Run optimization model for entire inventory.
\index{USE\_TQDM (optimization.model.main.ModelOptimization attribute)@\spxentry{USE\_TQDM}\spxextra{optimization.model.main.ModelOptimization attribute}}

\begin{fulllineitems}
\phantomsection\label{\detokenize{source/optimization.model:optimization.model.main.ModelOptimization.USE_TQDM}}\pysigline{\sphinxbfcode{\sphinxupquote{USE\_TQDM}}}
When set to True, adds progression bar at console running python.
\begin{quote}\begin{description}
\item[{Type}] \leavevmode
bool

\end{description}\end{quote}

\end{fulllineitems}

\begin{quote}\begin{description}
\item[{Returns}] \leavevmode
\sphinxstylestrong{optimization\_list} \textendash{} List of all optimization results.

\item[{Return type}] \leavevmode
list

\end{description}\end{quote}

\begin{sphinxadmonition}{tip}{Tip:}
\sphinxcode{\sphinxupquote{TQDM}} will only be enabled if found in the Python environment. Else this attribute is set to \sphinxcode{\sphinxupquote{False}} automatically.
\end{sphinxadmonition}

\end{fulllineitems}

\index{transfer\_value (optimization.model.main.ModelOptimization attribute)@\spxentry{transfer\_value}\spxextra{optimization.model.main.ModelOptimization attribute}}

\begin{fulllineitems}
\phantomsection\label{\detokenize{source/optimization.model:optimization.model.main.ModelOptimization.transfer_value}}\pysigline{\sphinxbfcode{\sphinxupquote{transfer\_value}}\sphinxbfcode{\sphinxupquote{ = \textquotesingle{}\$ Value of Transfer\textquotesingle{}}}}
\end{fulllineitems}


\end{fulllineitems}

\index{concat() (in module optimization.model.main)@\spxentry{concat()}\spxextra{in module optimization.model.main}}

\begin{fulllineitems}
\phantomsection\label{\detokenize{source/optimization.model:optimization.model.main.concat}}\pysiglinewithargsret{\sphinxcode{\sphinxupquote{optimization.model.main.}}\sphinxbfcode{\sphinxupquote{concat}}}{\emph{\DUrole{n}{results\_list}\DUrole{p}{:} \DUrole{n}{list}}}{}
Concatenates list of optimization results.

Since the optimization model runs separately for every unique Item ID
and outputs separate recommendation reports for every SKU, this function is used inside
\sphinxcode{\sphinxupquote{main}} to join all recommendations together into the final results.
\begin{quote}\begin{description}
\item[{Parameters}] \leavevmode
\sphinxstyleliteralstrong{\sphinxupquote{results\_list}} (\sphinxstyleliteralemphasis{\sphinxupquote{(}}\sphinxstyleliteralemphasis{\sphinxupquote{list}}\sphinxstyleliteralemphasis{\sphinxupquote{)}}) \textendash{} List with optimization results obtained.

\item[{Returns}] \leavevmode
Combined optimization model results.

\item[{Return type}] \leavevmode
pd.DataFrame

\item[{Raises}] \leavevmode
\sphinxstyleliteralstrong{\sphinxupquote{TypeError}} \textendash{} Gives error if the function input \sphinxcode{\sphinxupquote{results\_list}} is not of type \sphinxcode{\sphinxupquote{list}}

\end{description}\end{quote}

\end{fulllineitems}

\index{main() (in module optimization.model.main)@\spxentry{main()}\spxextra{in module optimization.model.main}}

\begin{fulllineitems}
\phantomsection\label{\detokenize{source/optimization.model:optimization.model.main.main}}\pysiglinewithargsret{\sphinxcode{\sphinxupquote{optimization.model.main.}}\sphinxbfcode{\sphinxupquote{main}}}{\emph{\DUrole{n}{lot\_df}\DUrole{p}{:} \DUrole{n}{pandas.core.frame.DataFrame}}, \emph{\DUrole{n}{nonlot\_df}\DUrole{p}{:} \DUrole{n}{pandas.core.frame.DataFrame}}, \emph{\DUrole{n}{step}\DUrole{p}{:} \DUrole{n}{int}}, \emph{\DUrole{n}{optimize\_what}\DUrole{p}{:} \DUrole{n}{str}}, \emph{\DUrole{n}{save\_res}\DUrole{p}{:} \DUrole{n}{bool} \DUrole{o}{=} \DUrole{default_value}{True}}, \emph{\DUrole{n}{sku\_qty}\DUrole{p}{:} \DUrole{n}{int} \DUrole{o}{=} \DUrole{default_value}{0}}}{}
Combines all necessary steps for running the optimization model.

This function calls all neessary procedures that are needed to perform the otimization
This is the function that \sphinxstylestrong{Alteryx Server} uses to generate the optimization results.
\begin{quote}\begin{description}
\item[{Parameters}] \leavevmode\begin{itemize}
\item {} 
\sphinxstyleliteralstrong{\sphinxupquote{lot\_df}} (\sphinxstyleliteralemphasis{\sphinxupquote{pd.DataFrame}}) \textendash{} Input Inventory report for \sphinxstylestrong{Lot} Items.

\item {} 
\sphinxstyleliteralstrong{\sphinxupquote{nonlot\_df}} (\sphinxstyleliteralemphasis{\sphinxupquote{pd.DataFrame}}) \textendash{} Input Inventory report for \sphinxstylestrong{Non\sphinxhyphen{}Lot} Items.

\item {} 
\sphinxstyleliteralstrong{\sphinxupquote{step}} (\sphinxstyleliteralemphasis{\sphinxupquote{int}}) \textendash{} Was used for performing simulations. \sphinxstylestrong{Not Used Anymore}.

\item {} 
\sphinxstyleliteralstrong{\sphinxupquote{optimize\_what}} (\sphinxstyleliteralemphasis{\sphinxupquote{str}}) \textendash{} 
Type of problem we want to optimize. Can be one of:
\begin{itemize}
\item {} 
\sphinxcode{\sphinxupquote{expire}}: Optimize inventory for reducing \sphinxstylestrong{only} quantity of items to expire.

\item {} 
\sphinxcode{\sphinxupquote{surplus}}: Optimize inventory for reducing \sphinxstylestrong{only} quantity of surplus items.

\item {} 
\sphinxcode{\sphinxupquote{both}}: Optimize inventory for reducing \sphinxstylestrong{both} quantity of surplus items and items to expire at the same time.

\item {} \begin{description}
\item[{\sphinxcode{\sphinxupquote{experimental}}: Experimental objective function that tries to formulate the bi\sphinxhyphen{}objective function in a way that is}] \leavevmode
more complex but has potential to yield better results.

\end{description}

\end{itemize}


\item {} 
\sphinxstyleliteralstrong{\sphinxupquote{save\_res}} (\sphinxstyleliteralemphasis{\sphinxupquote{bool}}\sphinxstyleliteralemphasis{\sphinxupquote{, }}\sphinxstyleliteralemphasis{\sphinxupquote{optional}}) \textendash{} Used if user wants to save optimization results at the folder \sphinxcode{\sphinxupquote{optimization/results}}, by default True

\item {} 
\sphinxstyleliteralstrong{\sphinxupquote{sku\_qty}} (\sphinxstyleliteralemphasis{\sphinxupquote{int}}\sphinxstyleliteralemphasis{\sphinxupquote{, }}\sphinxstyleliteralemphasis{\sphinxupquote{optional}}) \textendash{} \begin{description}
\item[{\sphinxstylestrong{Mainly used for debugging purposes}. Inform the quantity of unique item IDs the model should perform the optimization.}] \leavevmode
If set to 0 runs the optimization for all Item IDs, by default 0

\end{description}


\end{itemize}

\item[{Returns}] \leavevmode
List with all \sphinxcode{\sphinxupquote{transfer recommendations}}. In production environment this Table is then passed on to Alteryx and saved as \sphinxcode{\sphinxupquote{.xslx}} file.

\item[{Return type}] \leavevmode
pd.DataFrame

\item[{Warns}] \leavevmode\begin{itemize}
\item {} 
\sphinxstylestrong{Do not use the experimental option on production environment. Experimental objective function is not 100\% validated and might}

\item {} 
\sphinxstylestrong{give incorrect recommendation transfers in some specific cases.}

\end{itemize}

\end{description}\end{quote}

\end{fulllineitems}



\subsubsection{Optimizer}
\label{\detokenize{source/optimization.model:module-optimization.model.optimizer}}\label{\detokenize{source/optimization.model:optimizer}}\index{module@\spxentry{module}!optimization.model.optimizer@\spxentry{optimization.model.optimizer}}\index{optimization.model.optimizer@\spxentry{optimization.model.optimizer}!module@\spxentry{module}}
Main script for generating transfer recommendations.

This module is used to define the optimization problem that in general is composed by a combination of constraints and objective function.
\index{ObjectiveFunction (class in optimization.model.optimizer)@\spxentry{ObjectiveFunction}\spxextra{class in optimization.model.optimizer}}

\begin{fulllineitems}
\phantomsection\label{\detokenize{source/optimization.model:optimization.model.optimizer.ObjectiveFunction}}\pysigline{\sphinxbfcode{\sphinxupquote{class }}\sphinxcode{\sphinxupquote{optimization.model.optimizer.}}\sphinxbfcode{\sphinxupquote{ObjectiveFunction}}}
Bases: \sphinxcode{\sphinxupquote{object}}

Define objective function.
\index{\_\_init\_\_() (optimization.model.optimizer.ObjectiveFunction method)@\spxentry{\_\_init\_\_()}\spxextra{optimization.model.optimizer.ObjectiveFunction method}}

\begin{fulllineitems}
\phantomsection\label{\detokenize{source/optimization.model:optimization.model.optimizer.ObjectiveFunction.__init__}}\pysiglinewithargsret{\sphinxbfcode{\sphinxupquote{\_\_init\_\_}}}{}{}
\sphinxcode{\sphinxupquote{ObjectiveFunction}} \sphinxstylestrong{class} initialization.

Add any required initialization steps here.

\end{fulllineitems}

\index{\_combined\_objective() (optimization.model.optimizer.ObjectiveFunction method)@\spxentry{\_combined\_objective()}\spxextra{optimization.model.optimizer.ObjectiveFunction method}}

\begin{fulllineitems}
\phantomsection\label{\detokenize{source/optimization.model:optimization.model.optimizer.ObjectiveFunction._combined_objective}}\pysiglinewithargsret{\sphinxbfcode{\sphinxupquote{\_combined\_objective}}}{}{}
Combine both objectives (minimize items to expire and surplus).

If Provider Lot has items to expire and, its BU surplus, we determine what is more
representative: surplus or items to expire and try to minimize that. The other possibility
would be that the BU has shortage but items to expire and so we consider minimizing items
to expire in our objective function.
\begin{align*}\!\begin{aligned}
min \bigg [ \, \text{expire_weight} \, \cdot \, \bigg (\, \sum_{j=1}^{row}\,ITE(\,j\,)\,-\,(\,\sum_{i=1}^{col}\,\, (\,x_i\,) \bigg )\\
+ \, \text{surplus_weight} \cdot \, \bigg ( \, \sum_{j=1}^{col}\,Inv. Balance(\,j\,)\,+ \,(\,\sum_{i=1}^{row}\,\,  (\,x_i\,) \bigg ) \bigg ]\\
\end{aligned}\end{align*}
\end{fulllineitems}

\index{\_experimental\_objective() (optimization.model.optimizer.ObjectiveFunction method)@\spxentry{\_experimental\_objective()}\spxextra{optimization.model.optimizer.ObjectiveFunction method}}

\begin{fulllineitems}
\phantomsection\label{\detokenize{source/optimization.model:optimization.model.optimizer.ObjectiveFunction._experimental_objective}}\pysiglinewithargsret{\sphinxbfcode{\sphinxupquote{\_experimental\_objective}}}{}{}
Experimental objetive function.

This function is used to test new objective functions during development phase.

\end{fulllineitems}

\index{\_expire\_objective() (optimization.model.optimizer.ObjectiveFunction method)@\spxentry{\_expire\_objective()}\spxextra{optimization.model.optimizer.ObjectiveFunction method}}

\begin{fulllineitems}
\phantomsection\label{\detokenize{source/optimization.model:optimization.model.optimizer.ObjectiveFunction._expire_objective}}\pysiglinewithargsret{\sphinxbfcode{\sphinxupquote{\_expire\_objective}}}{}{}
Define items to expire minimization objective.
\begin{equation*}
\begin{split}min\, \bigg ( \,\sum_{j=1}^{n2}\,ITE(\,j\,)\,-\,(\,\sum_{i=1}^{n1}\,\,  (\,x_i\,) \bigg  )\end{split}
\end{equation*}
\end{fulllineitems}

\index{\_get\_total\_transfers() (optimization.model.optimizer.ObjectiveFunction method)@\spxentry{\_get\_total\_transfers()}\spxextra{optimization.model.optimizer.ObjectiveFunction method}}

\begin{fulllineitems}
\phantomsection\label{\detokenize{source/optimization.model:optimization.model.optimizer.ObjectiveFunction._get_total_transfers}}\pysiglinewithargsret{\sphinxbfcode{\sphinxupquote{\_get\_total\_transfers}}}{}{}
Capture quantity of different transfers between BU’s were made.
\begin{quote}\begin{description}
\item[{Returns}] \leavevmode
Total quantity of distinct transfers between BU’s. Can range from 0
to quantity of different combinations of BU’s possible.

\item[{Return type}] \leavevmode
int

\end{description}\end{quote}

\end{fulllineitems}

\index{\_surplus\_objective() (optimization.model.optimizer.ObjectiveFunction method)@\spxentry{\_surplus\_objective()}\spxextra{optimization.model.optimizer.ObjectiveFunction method}}

\begin{fulllineitems}
\phantomsection\label{\detokenize{source/optimization.model:optimization.model.optimizer.ObjectiveFunction._surplus_objective}}\pysiglinewithargsret{\sphinxbfcode{\sphinxupquote{\_surplus\_objective}}}{}{}
Define surplus minimization objective.

We want it to transfer the maximum number of items as possible, given the applied constraints.
\begin{equation*}
\begin{split}min\,\bigg ( \,\sum_{j=1}^{n2}\,Inv. Balance(\,j\,)\,+\,(\,\sum_{i=1}^{n1}\,\,  (\,x_i\,) \bigg )\end{split}
\end{equation*}
\end{fulllineitems}

\index{\_validate() (optimization.model.optimizer.ObjectiveFunction static method)@\spxentry{\_validate()}\spxextra{optimization.model.optimizer.ObjectiveFunction static method}}

\begin{fulllineitems}
\phantomsection\label{\detokenize{source/optimization.model:optimization.model.optimizer.ObjectiveFunction._validate}}\pysiglinewithargsret{\sphinxbfcode{\sphinxupquote{static }}\sphinxbfcode{\sphinxupquote{\_validate}}}{\emph{\DUrole{n}{opt\_problem}\DUrole{p}{:} \DUrole{n}{str}}}{}
Assert opt\_problem is one of possible choices.
\begin{quote}\begin{description}
\item[{Parameters}] \leavevmode
\sphinxstyleliteralstrong{\sphinxupquote{opt\_problem}} (\sphinxstyleliteralemphasis{\sphinxupquote{str}}) \textendash{} 
Type of objective we want to optimize. Can be:
\begin{itemize}
\item {} 
\sphinxstylestrong{expire:} minimize items to expire

\item {} 
\sphinxstylestrong{surplus:} minimize surplus

\item {} 
\sphinxstylestrong{both:} minimize both surplus and items to expire

\item {} 
\sphinxstylestrong{experimental:} bi\sphinxhyphen{}objective function that uses different configurable                 weights for inventory balance and items to expire.

\end{itemize}


\item[{Raises}] \leavevmode
\sphinxstyleliteralstrong{\sphinxupquote{AttributeError}} \textendash{} We don’t have an objective function for the type of
    problem that was specified or someone mistyped its name.

\end{description}\end{quote}

\end{fulllineitems}

\index{hasattr() (optimization.model.optimizer.ObjectiveFunction method)@\spxentry{hasattr()}\spxextra{optimization.model.optimizer.ObjectiveFunction method}}

\begin{fulllineitems}
\phantomsection\label{\detokenize{source/optimization.model:optimization.model.optimizer.ObjectiveFunction.hasattr}}\pysiglinewithargsret{\sphinxbfcode{\sphinxupquote{hasattr}}}{\emph{\DUrole{n}{attr}}}{}
Looks for a given attribute \sphinxcode{\sphinxupquote{attr}} inside \sphinxcode{\sphinxupquote{\_\_dict\_\_}}
\begin{quote}\begin{description}
\item[{Parameters}] \leavevmode
\sphinxstyleliteralstrong{\sphinxupquote{attr}} (\sphinxstyleliteralemphasis{\sphinxupquote{object}}) \textendash{} Attribute we’re trying to find.

\item[{Returns}] \leavevmode
\sphinxstylestrong{attr} \textendash{} Method returns the Attribute itself if found.

\item[{Return type}] \leavevmode
object

\end{description}\end{quote}

\begin{sphinxadmonition}{note}{Note:}
{\hyperref[\detokenize{source/optimization.model:optimization.model.optimizer.ObjectiveFunction}]{\sphinxcrossref{\sphinxcode{\sphinxupquote{ObjectiveFunction}}}}} is the \sphinxcode{\sphinxupquote{base class}} of         {\hyperref[\detokenize{source/optimization.model:optimization.model.optimizer.OptimizationModel}]{\sphinxcrossref{\sphinxcode{\sphinxupquote{OptimizationModel}}}}}. We created this method because         the \sphinxcode{\sphinxupquote{base class}} inherits attributes and methods from its child class and we want to know, speciffically         if inside the base class, there is an attribute named \sphinxcode{\sphinxupquote{opt\_problem}}. If there is, we automatically set         the \sphinxstylestrong{objective function} to our optimization problem without having to call any method         (right at the instantiation of this \sphinxcode{\sphinxupquote{base class}})
\end{sphinxadmonition}

\end{fulllineitems}

\index{set\_objective() (optimization.model.optimizer.ObjectiveFunction method)@\spxentry{set\_objective()}\spxextra{optimization.model.optimizer.ObjectiveFunction method}}

\begin{fulllineitems}
\phantomsection\label{\detokenize{source/optimization.model:optimization.model.optimizer.ObjectiveFunction.set_objective}}\pysiglinewithargsret{\sphinxbfcode{\sphinxupquote{set\_objective}}}{}{}
Used to define which objective function to call.

The argument used by this method is passed to the class \sphinxcode{\sphinxupquote{OptimizationModel}} as parameter.

\sphinxstylestrong{Can be either:}
\begin{itemize}
\item {} 
\sphinxcode{\sphinxupquote{expire}}: Optimize inventory for reducing \sphinxstylestrong{only} quantity of items to expire.

\item {} 
\sphinxcode{\sphinxupquote{surplus}}: Optimize inventory for reducing \sphinxstylestrong{only} quantity of surplus items.

\item {} 
\sphinxcode{\sphinxupquote{both}}: Optimize inventory for reducing \sphinxstylestrong{both} quantity of surplus items and items to expire at the same time.

\item {} 
\sphinxcode{\sphinxupquote{experimental}}: Experimental objective function that tries to formulate the bi\sphinxhyphen{}objective function in a way that is         more complex but has potential to yield better results.

\end{itemize}

\begin{sphinxadmonition}{warning}{Warning:}
Do not use the experimental option on production environment. Experimental objective function is not 100\% validated and might
give incorrect recommendation transfers in some specific cases.
\end{sphinxadmonition}

\end{fulllineitems}


\end{fulllineitems}

\index{OptimizationModel (class in optimization.model.optimizer)@\spxentry{OptimizationModel}\spxextra{class in optimization.model.optimizer}}

\begin{fulllineitems}
\phantomsection\label{\detokenize{source/optimization.model:optimization.model.optimizer.OptimizationModel}}\pysiglinewithargsret{\sphinxbfcode{\sphinxupquote{class }}\sphinxcode{\sphinxupquote{optimization.model.optimizer.}}\sphinxbfcode{\sphinxupquote{OptimizationModel}}}{\emph{\DUrole{n}{smatrix}\DUrole{p}{:} \DUrole{n}{numpy.ndarray}}, \emph{\DUrole{n}{item\_id}\DUrole{p}{:} \DUrole{n}{object}}, \emph{\DUrole{n}{opt\_problem}\DUrole{p}{:} \DUrole{n}{str}}, \emph{\DUrole{n}{solver\_time}\DUrole{p}{:} \DUrole{n}{int}}}{}
Bases: {\hyperref[\detokenize{source/optimization.model:optimization.model.optimizer.ObjectiveFunction}]{\sphinxcrossref{\sphinxcode{\sphinxupquote{optimization.model.optimizer.ObjectiveFunction}}}}}

Create transfer recommendations.

Class used to define optimization problem. The class has two main steps: define the \sphinxstylestrong{model constraints} and \sphinxstylestrong{objective function}.
\begin{quote}\begin{description}
\item[{Returns}] \leavevmode
Pandas dataframe with transfer recommendations between BU’s.

\item[{Return type}] \leavevmode
pd.DataFrame

\end{description}\end{quote}
\index{EXTRA\_COLUMNS (optimization.model.optimizer.OptimizationModel attribute)@\spxentry{EXTRA\_COLUMNS}\spxextra{optimization.model.optimizer.OptimizationModel attribute}}

\begin{fulllineitems}
\phantomsection\label{\detokenize{source/optimization.model:optimization.model.optimizer.OptimizationModel.EXTRA_COLUMNS}}\pysigline{\sphinxbfcode{\sphinxupquote{EXTRA\_COLUMNS}}\sphinxbfcode{\sphinxupquote{ = 30}}}
\end{fulllineitems}

\index{EXTRA\_ROWS (optimization.model.optimizer.OptimizationModel attribute)@\spxentry{EXTRA\_ROWS}\spxextra{optimization.model.optimizer.OptimizationModel attribute}}

\begin{fulllineitems}
\phantomsection\label{\detokenize{source/optimization.model:optimization.model.optimizer.OptimizationModel.EXTRA_ROWS}}\pysigline{\sphinxbfcode{\sphinxupquote{EXTRA\_ROWS}}\sphinxbfcode{\sphinxupquote{ = 25}}}
\end{fulllineitems}

\index{PROVIDING\_COLUMNS (optimization.model.optimizer.OptimizationModel attribute)@\spxentry{PROVIDING\_COLUMNS}\spxextra{optimization.model.optimizer.OptimizationModel attribute}}

\begin{fulllineitems}
\phantomsection\label{\detokenize{source/optimization.model:optimization.model.optimizer.OptimizationModel.PROVIDING_COLUMNS}}\pysigline{\sphinxbfcode{\sphinxupquote{PROVIDING\_COLUMNS}}\sphinxbfcode{\sphinxupquote{ = \{\textquotesingle{}avg\_cons\_prov\textquotesingle{}: 4, \textquotesingle{}bu\_doi\_balance\textquotesingle{}: 10, \textquotesingle{}bu\_provide\textquotesingle{}: 0, \textquotesingle{}days\_expire\textquotesingle{}: 6, \textquotesingle{}inv\_provide\textquotesingle{}: 8, \textquotesingle{}lot\_id\textquotesingle{}: 9, \textquotesingle{}max\_expire\textquotesingle{}: 5, \textquotesingle{}max\_provide\textquotesingle{}: 1, \textquotesingle{}min\_ship\_prov\textquotesingle{}: 2, \textquotesingle{}price\_prov\textquotesingle{}: 3, \textquotesingle{}provide\_bu\_address\textquotesingle{}: 20, \textquotesingle{}provide\_bu\_descrip\textquotesingle{}: 21, \textquotesingle{}provide\_bu\_item\_qty\_in\_transf\textquotesingle{}: 25, \textquotesingle{}provide\_bu\_oh\_plus\_transit\textquotesingle{}: 12, \textquotesingle{}provide\_bu\_qty\_on\_hand\textquotesingle{}: 24, \textquotesingle{}provide\_bu\_region\textquotesingle{}: 15, \textquotesingle{}provide\_can\_receive\_inventory\textquotesingle{}: 28, \textquotesingle{}provide\_can\_transfer\_inventory\textquotesingle{}: 27, \textquotesingle{}provide\_chart\_of\_accounts\textquotesingle{}: 19, \textquotesingle{}provide\_contact\_email\textquotesingle{}: 16, \textquotesingle{}provide\_days\_of\_inventory\textquotesingle{}: 13, \textquotesingle{}provide\_default\_shipment\_days\textquotesingle{}: 26, \textquotesingle{}provide\_delta\_doi\textquotesingle{}: 11, \textquotesingle{}provide\_expire\_date\textquotesingle{}: 23, \textquotesingle{}provide\_item\_descrip\textquotesingle{}: 14, \textquotesingle{}provide\_item\_stats\textquotesingle{}: 29, \textquotesingle{}provide\_on\_site\_email\textquotesingle{}: 22, \textquotesingle{}provide\_std\_uom\textquotesingle{}: 18, \textquotesingle{}provide\_supplier\_name\textquotesingle{}: 17, \textquotesingle{}target\_provide\textquotesingle{}: 7\}}}}
\end{fulllineitems}

\index{RECEIVING\_COLUMNS (optimization.model.optimizer.OptimizationModel attribute)@\spxentry{RECEIVING\_COLUMNS}\spxextra{optimization.model.optimizer.OptimizationModel attribute}}

\begin{fulllineitems}
\phantomsection\label{\detokenize{source/optimization.model:optimization.model.optimizer.OptimizationModel.RECEIVING_COLUMNS}}\pysigline{\sphinxbfcode{\sphinxupquote{RECEIVING\_COLUMNS}}\sphinxbfcode{\sphinxupquote{ = \{\textquotesingle{}avg\_cons\textquotesingle{}: 4, \textquotesingle{}bu\_receive\textquotesingle{}: 0, \textquotesingle{}inv\_receive\textquotesingle{}: 6, \textquotesingle{}max\_receive\textquotesingle{}: 1, \textquotesingle{}min\_ship\textquotesingle{}: 2, \textquotesingle{}price\textquotesingle{}: 3, \textquotesingle{}receive\_bu\_address\textquotesingle{}: 16, \textquotesingle{}receive\_bu\_descrip\textquotesingle{}: 17, \textquotesingle{}receive\_bu\_item\_qty\_in\_transf\textquotesingle{}: 20, \textquotesingle{}receive\_bu\_oh\_plus\_transit\textquotesingle{}: 8, \textquotesingle{}receive\_bu\_region\textquotesingle{}: 11, \textquotesingle{}receive\_can\_receive\_inventory\textquotesingle{}: 23, \textquotesingle{}receive\_can\_transfer\_inventory\textquotesingle{}: 22, \textquotesingle{}receive\_contact\_email\textquotesingle{}: 12, \textquotesingle{}receive\_days\_of\_inventory\textquotesingle{}: 9, \textquotesingle{}receive\_delta\_doi\textquotesingle{}: 7, \textquotesingle{}receive\_item\_stats\textquotesingle{}: 24, \textquotesingle{}receive\_items\_to\_expire\textquotesingle{}: 19, \textquotesingle{}receive\_on\_site\_email\textquotesingle{}: 18, \textquotesingle{}target\_receive\textquotesingle{}: 5\}}}}
\end{fulllineitems}

\index{\_\_init\_\_() (optimization.model.optimizer.OptimizationModel method)@\spxentry{\_\_init\_\_()}\spxextra{optimization.model.optimizer.OptimizationModel method}}

\begin{fulllineitems}
\phantomsection\label{\detokenize{source/optimization.model:optimization.model.optimizer.OptimizationModel.__init__}}\pysiglinewithargsret{\sphinxbfcode{\sphinxupquote{\_\_init\_\_}}}{\emph{\DUrole{n}{smatrix}\DUrole{p}{:} \DUrole{n}{numpy.ndarray}}, \emph{\DUrole{n}{item\_id}\DUrole{p}{:} \DUrole{n}{object}}, \emph{\DUrole{n}{opt\_problem}\DUrole{p}{:} \DUrole{n}{str}}, \emph{\DUrole{n}{solver\_time}\DUrole{p}{:} \DUrole{n}{int}}}{}
Arguments used at the optimization model.
\begin{quote}\begin{description}
\item[{Parameters}] \leavevmode\begin{itemize}
\item {} 
\sphinxstyleliteralstrong{\sphinxupquote{smatrix}} (\sphinxstyleliteralemphasis{\sphinxupquote{numpy.ndarray}}) \textendash{} Matrix of size (n x m) with data containing the amounts
of items from a given SKU that can be sent and received from one BU to
another. Size n represents all lots from a given item, available on all BUs and m the number o receiving BUs.

\item {} 
\sphinxstyleliteralstrong{\sphinxupquote{item\_id}} (\sphinxstyleliteralemphasis{\sphinxupquote{int}}) \textendash{} ID of the item that we’re trying to optimize.

\item {} 
\sphinxstyleliteralstrong{\sphinxupquote{opt\_problem}} (\sphinxstyleliteralemphasis{\sphinxupquote{str}}) \textendash{} Type of objective to be added to the model. \sphinxcode{\sphinxupquote{expire}}, \sphinxcode{\sphinxupquote{surplus}}, \sphinxcode{\sphinxupquote{both}}.

\item {} 
\sphinxstyleliteralstrong{\sphinxupquote{solver\_time}} (\sphinxstyleliteralemphasis{\sphinxupquote{int}}) \textendash{} Maximum time the solver has to find the optimizal solution to a single item ID.

\end{itemize}

\item[{Returns}] \leavevmode


\item[{Return type}] \leavevmode
None.

\end{description}\end{quote}

\begin{sphinxadmonition}{note}{Note:}
The method {\hyperref[\detokenize{source/optimization.model:optimization.model.optimizer.OptimizationModel._check_optimization_objective}]{\sphinxcrossref{\sphinxcode{\sphinxupquote{\_check\_optimization\_objective()}}}}}             will override the \sphinxcode{\sphinxupquote{opt\_problem}} parameter that was passed if it identifies that there is no need to optimize both \sphinxcode{\sphinxupquote{surplus}} and \sphinxcode{\sphinxupquote{items to expire}}.
\end{sphinxadmonition}

\end{fulllineitems}

\index{\_add\_providing\_columns() (optimization.model.optimizer.OptimizationModel method)@\spxentry{\_add\_providing\_columns()}\spxextra{optimization.model.optimizer.OptimizationModel method}}

\begin{fulllineitems}
\phantomsection\label{\detokenize{source/optimization.model:optimization.model.optimizer.OptimizationModel._add_providing_columns}}\pysiglinewithargsret{\sphinxbfcode{\sphinxupquote{\_add\_providing\_columns}}}{\emph{\DUrole{n}{opt\_df}}}{}
Adds new columns necessary to the output report.

This method is called after the model finishes the optimization process and
adds extra columns required for the output report.
\begin{quote}\begin{description}
\item[{Parameters}] \leavevmode
\sphinxstyleliteralstrong{\sphinxupquote{opt\_df}} (\sphinxstyleliteralemphasis{\sphinxupquote{{[}}}\sphinxstyleliteralemphasis{\sphinxupquote{type}}\sphinxstyleliteralemphasis{\sphinxupquote{{]}}}) \textendash{} Model output right after the optimization process.

\item[{Returns}] \leavevmode
Transfer recommendation with the required additional columns.

\item[{Return type}] \leavevmode
pd.DataFrame

\end{description}\end{quote}

\end{fulllineitems}

\index{\_check\_optimization\_objective() (optimization.model.optimizer.OptimizationModel method)@\spxentry{\_check\_optimization\_objective()}\spxextra{optimization.model.optimizer.OptimizationModel method}}

\begin{fulllineitems}
\phantomsection\label{\detokenize{source/optimization.model:optimization.model.optimizer.OptimizationModel._check_optimization_objective}}\pysiglinewithargsret{\sphinxbfcode{\sphinxupquote{\_check\_optimization\_objective}}}{}{}
Used to verify if problem needs to be bi\sphinxhyphen{}objective or not.

If we have NONLOT items or if the Lot items have no quantity to expire,
we don’t need to use the multi\sphinxhyphen{}objective function. We just need to
minimize surplus and shortage.

On the other hand, if we items to expire and in more quantity than there is
BUs with shortage, then we don’t need to optimize surplus and shortage. All
shortage will be already most likely consumed by the expiring items and some BUs
will end up with a suplus in order to minimize those expiring items.

Finally, if we have items to expire but in less quantity than shortage, then
we need the bi\sphinxhyphen{}objective function to try to arrange the transfers the best way possible.

\sphinxstylestrong{In other words this function will:}
\begin{enumerate}
\sphinxsetlistlabels{\arabic}{enumi}{enumii}{}{.}%
\item {} 
Analyze input data and determine if new objective needs to be set.

\item {} 
Set new objective function and override the one that was passed if it identifies that the objective needs to be changed.

\end{enumerate}

\end{fulllineitems}

\index{\_column\_constraints() (optimization.model.optimizer.OptimizationModel method)@\spxentry{\_column\_constraints()}\spxextra{optimization.model.optimizer.OptimizationModel method}}

\begin{fulllineitems}
\phantomsection\label{\detokenize{source/optimization.model:optimization.model.optimizer.OptimizationModel._column_constraints}}\pysiglinewithargsret{\sphinxbfcode{\sphinxupquote{\_column\_constraints}}}{}{}
Define constraints that are passed to every column of our solution space matrix.

In general column\sphinxhyphen{}wise constraints are related to receiving BU’s.

\end{fulllineitems}

\index{\_consume\_before\_expire\_constraint() (optimization.model.optimizer.OptimizationModel method)@\spxentry{\_consume\_before\_expire\_constraint()}\spxextra{optimization.model.optimizer.OptimizationModel method}}

\begin{fulllineitems}
\phantomsection\label{\detokenize{source/optimization.model:optimization.model.optimizer.OptimizationModel._consume_before_expire_constraint}}\pysiglinewithargsret{\sphinxbfcode{\sphinxupquote{\_consume\_before\_expire\_constraint}}}{\emph{\DUrole{n}{col\_idx}\DUrole{p}{:} \DUrole{n}{int}}}{}
Constraint to limit item transfer to BUs that can consume them before they expire.

For transfers to other BU’s, we check subtract from column \sphinxcode{\sphinxupquote{days\_to\_expire}} the total amount of
days needed to make that transfer (column \sphinxcode{\sphinxupquote{default\_shipment\_days}}).

This function also subtracts items that are being transfered from the receiving BU to ohter Business units.
These items represent a reduction on the amount of days needed to consume that given Lot.
For items being \sphinxstylestrong{transfered out of the BU} we don’t need to take those \sphinxcode{\sphinxupquote{default\_shipment\_days}}.
\begin{quote}\begin{description}
\item[{Parameters}] \leavevmode\begin{itemize}
\item {} 
\sphinxstyleliteralstrong{\sphinxupquote{col\_idx}} (\sphinxstyleliteralemphasis{\sphinxupquote{int}}) \textendash{} Column of our \sphinxcode{\sphinxupquote{solution\_matrix}}

\item {} 
\sphinxstyleliteralstrong{\sphinxupquote{tot}} (\sphinxstyleliteralemphasis{\sphinxupquote{int}}) \textendash{} Total sum of days to expire.

\end{itemize}

\end{description}\end{quote}

\end{fulllineitems}

\index{\_create\_constants() (optimization.model.optimizer.OptimizationModel method)@\spxentry{\_create\_constants()}\spxextra{optimization.model.optimizer.OptimizationModel method}}

\begin{fulllineitems}
\phantomsection\label{\detokenize{source/optimization.model:optimization.model.optimizer.OptimizationModel._create_constants}}\pysiglinewithargsret{\sphinxbfcode{\sphinxupquote{\_create\_constants}}}{}{}
Generate model constants.

These constants are used by the model output or to create the boundaries for the model constraints.

\end{fulllineitems}

\index{\_create\_decision\_variables() (optimization.model.optimizer.OptimizationModel method)@\spxentry{\_create\_decision\_variables()}\spxextra{optimization.model.optimizer.OptimizationModel method}}

\begin{fulllineitems}
\phantomsection\label{\detokenize{source/optimization.model:optimization.model.optimizer.OptimizationModel._create_decision_variables}}\pysiglinewithargsret{\sphinxbfcode{\sphinxupquote{\_create\_decision\_variables}}}{}{}
Define optimizable variables (it is a matrix of size n x n).

\end{fulllineitems}

\index{\_create\_elastic\_constraints() (optimization.model.optimizer.OptimizationModel method)@\spxentry{\_create\_elastic\_constraints()}\spxextra{optimization.model.optimizer.OptimizationModel method}}

\begin{fulllineitems}
\phantomsection\label{\detokenize{source/optimization.model:optimization.model.optimizer.OptimizationModel._create_elastic_constraints}}\pysiglinewithargsret{\sphinxbfcode{\sphinxupquote{\_create\_elastic\_constraints}}}{\emph{\DUrole{n}{col\_idx}\DUrole{p}{:} \DUrole{n}{int}}}{}
Generates model elastic constraints.

These constraints, are soft constraints and when the model reqches their boundary
the objective function gets penalized. Right now, we apply these constraint to penalize
when the model transfers more items than the receivers maximum shortage.
\begin{quote}\begin{description}
\item[{Parameters}] \leavevmode
\sphinxstyleliteralstrong{\sphinxupquote{col\_idx}} (\sphinxstyleliteralemphasis{\sphinxupquote{int}}) \textendash{} Column we’re applying the elastic constraint.

\end{description}\end{quote}

\end{fulllineitems}

\index{\_create\_main\_constraints() (optimization.model.optimizer.OptimizationModel method)@\spxentry{\_create\_main\_constraints()}\spxextra{optimization.model.optimizer.OptimizationModel method}}

\begin{fulllineitems}
\phantomsection\label{\detokenize{source/optimization.model:optimization.model.optimizer.OptimizationModel._create_main_constraints}}\pysiglinewithargsret{\sphinxbfcode{\sphinxupquote{\_create\_main\_constraints}}}{}{}
Add the constraints to our model.

Function calls methods \sphinxcode{\sphinxupquote{\_row\_constraints(self)}} and \sphinxcode{\sphinxupquote{\_column\_constraints(self)}}
that define constraints row and column\sphinxhyphen{}wise. Row\sphinxhyphen{}wise constraints usually
limits transfer recommendations based on Receiving BU limitations and column\sphinxhyphen{}wise
constraints do the same for receiving BU’s/Lot.

\begin{sphinxadmonition}{note}{Note:}
If you want to add new constraints to the model, add them here.
\end{sphinxadmonition}
\begin{quote}\begin{description}
\item[{Returns}] \leavevmode


\item[{Return type}] \leavevmode
None.

\end{description}\end{quote}

\end{fulllineitems}

\index{\_create\_opt\_df() (optimization.model.optimizer.OptimizationModel method)@\spxentry{\_create\_opt\_df()}\spxextra{optimization.model.optimizer.OptimizationModel method}}

\begin{fulllineitems}
\phantomsection\label{\detokenize{source/optimization.model:optimization.model.optimizer.OptimizationModel._create_opt_df}}\pysiglinewithargsret{\sphinxbfcode{\sphinxupquote{\_create\_opt\_df}}}{}{}
Creates optimization transfers table.

The function used to transform optimization problem results in table format.         Furthermore, add additional columns are added and         renamed accordingly to their final name.
:returns: \sphinxstylestrong{Table with transfer recommendations for a given Item ID}
:rtype: pd.DataFrame

\end{fulllineitems}

\index{\_min\_shipment\_value\_constraint() (optimization.model.optimizer.OptimizationModel method)@\spxentry{\_min\_shipment\_value\_constraint()}\spxextra{optimization.model.optimizer.OptimizationModel method}}

\begin{fulllineitems}
\phantomsection\label{\detokenize{source/optimization.model:optimization.model.optimizer.OptimizationModel._min_shipment_value_constraint}}\pysiglinewithargsret{\sphinxbfcode{\sphinxupquote{\_min\_shipment\_value\_constraint}}}{\emph{\DUrole{n}{row\_idx}}, \emph{\DUrole{n}{col\_idx}}}{}
Constraint transfers that are smaller than the \sphinxstylestrong{minimum shipment \$ value} specified at the \sphinxcode{\sphinxupquote{inventory report}}.

This constraint works by using an \sphinxcode{\sphinxupquote{auxiliary variable}} that needs to be zero when \sphinxcode{\sphinxupquote{x\_vars(row\_idx, col\_idx)}}
is smaller than \sphinxcode{\sphinxupquote{min\_sip{[}col\_idx{]}}}.
\begin{quote}\begin{description}
\item[{Parameters}] \leavevmode\begin{itemize}
\item {} 
\sphinxstyleliteralstrong{\sphinxupquote{row\_idx}} (\sphinxstyleliteralemphasis{\sphinxupquote{int}}) \textendash{} Index of the row (\sphinxcode{\sphinxupquote{sender}}) from where we’re transfering items from.

\item {} 
\sphinxstyleliteralstrong{\sphinxupquote{col\_idx}} (\sphinxstyleliteralemphasis{\sphinxupquote{int}}) \textendash{} Index of the column (\sphinxcode{\sphinxupquote{receiver}}) from where we’re transfering items to.

\end{itemize}

\end{description}\end{quote}

\end{fulllineitems}

\index{\_nonlot\_constraints() (optimization.model.optimizer.OptimizationModel method)@\spxentry{\_nonlot\_constraints()}\spxextra{optimization.model.optimizer.OptimizationModel method}}

\begin{fulllineitems}
\phantomsection\label{\detokenize{source/optimization.model:optimization.model.optimizer.OptimizationModel._nonlot_constraints}}\pysiglinewithargsret{\sphinxbfcode{\sphinxupquote{\_nonlot\_constraints}}}{}{}
Constraints used to define the optimization problem for \sphinxcode{\sphinxupquote{NONLOT}} items.

Since \sphinxcode{\sphinxupquote{NONLOT}} items do not have expiration dates, there is no need to define
constraints that restrict the amount of items that can be transfered based on the
expiration date.

This method adds the following constraints to the optimization model

\end{fulllineitems}

\index{\_restrict\_receiver\_surplus() (optimization.model.optimizer.OptimizationModel method)@\spxentry{\_restrict\_receiver\_surplus()}\spxextra{optimization.model.optimizer.OptimizationModel method}}

\begin{fulllineitems}
\phantomsection\label{\detokenize{source/optimization.model:optimization.model.optimizer.OptimizationModel._restrict_receiver_surplus}}\pysiglinewithargsret{\sphinxbfcode{\sphinxupquote{\_restrict\_receiver\_surplus}}}{\emph{\DUrole{n}{col\_idx}\DUrole{p}{:} \DUrole{n}{int}}}{}
Constraint to restrict receiver surplus.

In case we don’t have more \sphinxstylestrong{items to expire} than we have \sphinxstylestrong{shortage},         we shouldn’t transfer more items than what receiver business units with         shortage can acomodate. This method is used in such cases to let the optimization model         know it cannot transfer more items to receiver business units than what they have of shortage.
\begin{quote}
\begin{align*}\!\begin{aligned}
max(\text{ITE}) \, = \,\sum_{i}^{row} \, ITE_{i}\\
max(\text{Shortage}) \, = \,\sum_{i}^{col} \,  Shortage_{i}\\
\text{if} \, max(\text{Shortage}) \, \geq \, max(\text{ITE}^{max}) \, \text{:}\\
\sum_{i}^{col} \, ( Shortage_{i} \, + \, \sum_{i}^{row} \, x(i, j) ) \, \leq \, 0\\
\end{aligned}\end{align*}\end{quote}
\begin{quote}\begin{description}
\item[{Parameters}] \leavevmode
\sphinxstyleliteralstrong{\sphinxupquote{col\_idx}} (\sphinxstyleliteralemphasis{\sphinxupquote{int}}) \textendash{} Column index of receiver business unit we’re applying the constraint to.

\end{description}\end{quote}

\end{fulllineitems}

\index{\_row\_constraints() (optimization.model.optimizer.OptimizationModel method)@\spxentry{\_row\_constraints()}\spxextra{optimization.model.optimizer.OptimizationModel method}}

\begin{fulllineitems}
\phantomsection\label{\detokenize{source/optimization.model:optimization.model.optimizer.OptimizationModel._row_constraints}}\pysiglinewithargsret{\sphinxbfcode{\sphinxupquote{\_row\_constraints}}}{}{}
Define constraints that are passed to every row of our solution space matrix.

In general row\sphinxhyphen{}wise constraints are related to providing BU’s.
This function contains constraints that are applied depending on the objective

\end{fulllineitems}

\index{\_specify\_constants() (optimization.model.optimizer.OptimizationModel method)@\spxentry{\_specify\_constants()}\spxextra{optimization.model.optimizer.OptimizationModel method}}

\begin{fulllineitems}
\phantomsection\label{\detokenize{source/optimization.model:optimization.model.optimizer.OptimizationModel._specify_constants}}\pysiglinewithargsret{\sphinxbfcode{\sphinxupquote{\_specify\_constants}}}{\emph{\DUrole{n}{mapping\_list}\DUrole{p}{:} \DUrole{n}{list}}, \emph{\DUrole{n}{what}\DUrole{p}{:} \DUrole{n}{str} \DUrole{o}{=} \DUrole{default_value}{\textquotesingle{}receiving\textquotesingle{}}}}{}
Specify the mapping of constants with their respective names and indexes.
\begin{quote}\begin{description}
\item[{Parameters}] \leavevmode\begin{itemize}
\item {} 
\sphinxstyleliteralstrong{\sphinxupquote{mapping\_list}} (\sphinxstyleliteralemphasis{\sphinxupquote{list}}) \textendash{} List os constants and their index in the \sphinxcode{\sphinxupquote{solution matrix}}.

\item {} 
\sphinxstyleliteralstrong{\sphinxupquote{what}} (\sphinxstyleliteralemphasis{\sphinxupquote{str}}\sphinxstyleliteralemphasis{\sphinxupquote{, }}\sphinxstyleliteralemphasis{\sphinxupquote{optional}}) \textendash{} If constants being defined are for \sphinxcode{\sphinxupquote{receivers}} or \sphinxcode{\sphinxupquote{providers}}., by default ‘receiving’

\end{itemize}

\end{description}\end{quote}

\end{fulllineitems}

\index{\_specify\_extra\_constants() (optimization.model.optimizer.OptimizationModel method)@\spxentry{\_specify\_extra\_constants()}\spxextra{optimization.model.optimizer.OptimizationModel method}}

\begin{fulllineitems}
\phantomsection\label{\detokenize{source/optimization.model:optimization.model.optimizer.OptimizationModel._specify_extra_constants}}\pysiglinewithargsret{\sphinxbfcode{\sphinxupquote{\_specify\_extra\_constants}}}{}{}
Extra constants used to define model constraints and objective function.
\begin{quote}\begin{description}
\item[{Returns}] \leavevmode


\item[{Return type}] \leavevmode
None.

\end{description}\end{quote}

\end{fulllineitems}

\index{\_specify\_how() (optimization.model.optimizer.OptimizationModel method)@\spxentry{\_specify\_how()}\spxextra{optimization.model.optimizer.OptimizationModel method}}

\begin{fulllineitems}
\phantomsection\label{\detokenize{source/optimization.model:optimization.model.optimizer.OptimizationModel._specify_how}}\pysiglinewithargsret{\sphinxbfcode{\sphinxupquote{\_specify\_how}}}{\emph{\DUrole{n}{what}\DUrole{p}{:} \DUrole{n}{str}}, \emph{\DUrole{n}{value}\DUrole{p}{:} \DUrole{n}{int}}}{}
Used to get lists with informations from receiving and providing BUs.
\begin{quote}\begin{description}
\item[{Parameters}] \leavevmode\begin{itemize}
\item {} 
\sphinxstyleliteralstrong{\sphinxupquote{what}} (\sphinxstyleliteralemphasis{\sphinxupquote{str}}) \textendash{} Flag used to specify if we’re trying to obtain informations from \sphinxcode{\sphinxupquote{Providers}} or \sphinxcode{\sphinxupquote{Receivers}} business units.

\item {} 
\sphinxstyleliteralstrong{\sphinxupquote{value}} (\sphinxstyleliteralemphasis{\sphinxupquote{int}}) \textendash{} Index inside of the solution matrix where that given information is located.

\end{itemize}

\item[{Returns}] \leavevmode
Attribute of receiving or providing BU.

\item[{Return type}] \leavevmode
attr

\end{description}\end{quote}

\begin{sphinxadmonition}{note}{Note:}
The indexes for \sphinxcode{\sphinxupquote{Receivers}} and \sphinxcode{\sphinxupquote{Providers}} is located inside this module and are named \sphinxcode{\sphinxupquote{RECEIVING\_COLUMNS}} and \sphinxcode{\sphinxupquote{PROVIDING\_COLUMNS}}.         If you want to add aditional columns to be passed on to the model or to the output table, you first need to add them to         {\hyperref[\detokenize{source/optimization.datatools:optimization.datatools.pipelines.data_pipeline}]{\sphinxcrossref{\sphinxcode{\sphinxupquote{optimization.datatools.pipelines.data\_pipeline()}}}}} method, then at         {\hyperref[\detokenize{source/optimization:optimization.solspace.SolutionSpace}]{\sphinxcrossref{\sphinxcode{\sphinxupquote{optimization.solspace.SolutionSpace}}}}} \sphinxcode{\sphinxupquote{Class}} you need to add the new column to \sphinxcode{\sphinxupquote{PROVIDING\_AGG}}         and (or) to \sphinxcode{\sphinxupquote{RECEIVING\_AGG}}, specifying an aggregation type (\sphinxcode{\sphinxupquote{max}}, \sphinxcode{\sphinxupquote{first}}, \sphinxcode{\sphinxupquote{sum}}…). After adding the new column there, add the new column         to the end of \sphinxcode{\sphinxupquote{PROVIDING\_COLUMNS}} and (or) \sphinxcode{\sphinxupquote{RECEIVING\_COLUMNS}}. Finally, if you added the new column to both \sphinxcode{\sphinxupquote{PROVIDING\_COLUMNS}} and \sphinxcode{\sphinxupquote{RECEIVING\_COLUMNS}},         you need to add new names to differentiate the column at receiving part and providing part. You can perform this last step specifying the new names at         \sphinxcode{\sphinxupquote{COLUMNS\_RENAME}} inside this module.
\end{sphinxadmonition}

\end{fulllineitems}

\index{save\_model() (optimization.model.optimizer.OptimizationModel method)@\spxentry{save\_model()}\spxextra{optimization.model.optimizer.OptimizationModel method}}

\begin{fulllineitems}
\phantomsection\label{\detokenize{source/optimization.model:optimization.model.optimizer.OptimizationModel.save_model}}\pysiglinewithargsret{\sphinxbfcode{\sphinxupquote{save\_model}}}{\emph{\DUrole{n}{savedir}\DUrole{o}{=}\DUrole{default_value}{None}}, \emph{\DUrole{n}{dtype}\DUrole{p}{:} \DUrole{n}{str} \DUrole{o}{=} \DUrole{default_value}{\textquotesingle{}mps\textquotesingle{}}}}{}
save\_model results.

Function used to save model results. To activate it, please change
option \sphinxcode{\sphinxupquote{save\_model=False}} to \sphinxcode{\sphinxupquote{save\_model=True}} at \sphinxcode{\sphinxupquote{constants.py}}
or \sphinxstylestrong{write on alteryx}:

\begin{sphinxVerbatim}[commandchars=\\\{\}]
\PYG{k+kn}{from} \PYG{n+nn}{optimization} \PYG{k+kn}{import} \PYG{n}{constants}

\PYG{n}{constants}\PYG{o}{.}\PYG{n}{save\PYGZus{}model} \PYG{o}{=} \PYG{k+kc}{True}

\PYG{c+c1}{\PYGZsh{} Continue normal code procedures}
\PYG{o}{.}\PYG{o}{.}\PYG{o}{.}
\end{sphinxVerbatim}
\begin{quote}\begin{description}
\item[{Parameters}] \leavevmode\begin{itemize}
\item {} 
\sphinxstyleliteralstrong{\sphinxupquote{savedir}} (\sphinxstyleliteralemphasis{\sphinxupquote{str}}\sphinxstyleliteralemphasis{\sphinxupquote{, }}\sphinxstyleliteralemphasis{\sphinxupquote{optional}}) \textendash{} File directory to save model. If None is passed, results will be saved automatically on the same
folder as log file, by default None

\item {} 
\sphinxstyleliteralstrong{\sphinxupquote{dtype}} (\sphinxstyleliteralemphasis{\sphinxupquote{str}}\sphinxstyleliteralemphasis{\sphinxupquote{, }}\sphinxstyleliteralemphasis{\sphinxupquote{optional}}) \textendash{} Format user wants to save model, by default “mps”.

\end{itemize}

\item[{Raises}] \leavevmode\begin{itemize}
\item {} 
\sphinxstyleliteralstrong{\sphinxupquote{AssertionError}} \textendash{} Format that was passed to function is not supported.

\item {} 
\sphinxstyleliteralstrong{\sphinxupquote{FileNotFoundError}} \textendash{} Directory user wants to save model not found.

\end{itemize}

\item[{Warns}] \leavevmode
\sphinxstylestrong{**If you decide to save model at normal runtime, one individual file will be generated to every item id**}

\end{description}\end{quote}

\end{fulllineitems}

\index{solve() (optimization.model.optimizer.OptimizationModel method)@\spxentry{solve()}\spxextra{optimization.model.optimizer.OptimizationModel method}}

\begin{fulllineitems}
\phantomsection\label{\detokenize{source/optimization.model:optimization.model.optimizer.OptimizationModel.solve}}\pysiglinewithargsret{\sphinxbfcode{\sphinxupquote{solve}}}{}{{ $\rightarrow$ pandas.core.frame.DataFrame}}
Solve optimization problem.

Main method of our Model class. It calls all the other methods
of the class, creates the model, defines its contraints, adds the
objective function and optimizes it.
\begin{quote}\begin{description}
\item[{Returns}] \leavevmode
\sphinxstylestrong{opt\_df} \textendash{} Dataframe with the results of the optimization model.

\item[{Return type}] \leavevmode
pandas.core.frame.DataFrame

\end{description}\end{quote}

\begin{sphinxadmonition}{note}{Note:}
The model only returns results if they have an \sphinxstylestrong{optimal} solution status.
\end{sphinxadmonition}

\end{fulllineitems}


\end{fulllineitems}



\subsubsection{Module contents}
\label{\detokenize{source/optimization.model:module-optimization.model}}\label{\detokenize{source/optimization.model:module-contents}}\index{module@\spxentry{module}!optimization.model@\spxentry{optimization.model}}\index{optimization.model@\spxentry{optimization.model}!module@\spxentry{module}}
Optimizer Objective Function and main scripts.


\paragraph{Submodules}
\label{\detokenize{source/optimization.model:submodules}}

\begin{savenotes}\sphinxatlongtablestart\begin{longtable}[c]{\X{1}{2}\X{1}{2}}
\hline

\endfirsthead

\multicolumn{2}{c}%
{\makebox[0pt]{\sphinxtablecontinued{\tablename\ \thetable{} \textendash{} continued from previous page}}}\\
\hline

\endhead

\hline
\multicolumn{2}{r}{\makebox[0pt][r]{\sphinxtablecontinued{continues on next page}}}\\
\endfoot

\endlastfoot

{\hyperref[\detokenize{source/optimization.model:module-optimization.model.main}]{\sphinxcrossref{\sphinxcode{\sphinxupquote{main}}}}}
&
Main  module for running optimization model.
\\
\hline
{\hyperref[\detokenize{source/optimization.model:module-optimization.model.optimizer}]{\sphinxcrossref{\sphinxcode{\sphinxupquote{optimizer}}}}}
&
Main script for generating transfer recommendations.
\\
\hline
\end{longtable}\sphinxatlongtableend\end{savenotes}


\subsection{Datatools}
\label{\detokenize{source/optimization.datatools:datatools}}\label{\detokenize{source/optimization.datatools::doc}}
Modules used for \sphinxstylestrong{input data ingestion}, \sphinxstylestrong{input data validation}, \sphinxstylestrong{data cleanning} and \sphinxstylestrong{creating all necessary columns}.
Methods created for data manipulation are stored at {\hyperref[\detokenize{source/optimization.datatools:module-optimization.datatools.dataprep}]{\sphinxcrossref{\sphinxcode{\sphinxupquote{optimization.datatools.dataprep}}}}}.
Using the methods from {\hyperref[\detokenize{source/optimization.datatools:module-optimization.datatools.dataprep}]{\sphinxcrossref{\sphinxcode{\sphinxupquote{optimization.datatools.dataprep}}}}}, {\hyperref[\detokenize{source/optimization.datatools:module-optimization.datatools.pipelines}]{\sphinxcrossref{\sphinxcode{\sphinxupquote{optimization.datatools.pipelines}}}}} stores
all pipelines used to load and manipulate input inventory reports that are then passed to the optimization model.


\begin{savenotes}\sphinxatlongtablestart\begin{longtable}[c]{\X{1}{2}\X{1}{2}}
\hline

\endfirsthead

\multicolumn{2}{c}%
{\makebox[0pt]{\sphinxtablecontinued{\tablename\ \thetable{} \textendash{} continued from previous page}}}\\
\hline

\endhead

\hline
\multicolumn{2}{r}{\makebox[0pt][r]{\sphinxtablecontinued{continues on next page}}}\\
\endfoot

\endlastfoot

{\hyperref[\detokenize{source/optimization.datatools:module-optimization.datatools}]{\sphinxcrossref{\sphinxcode{\sphinxupquote{optimization.datatools}}}}}
&
Scripts used to generate necessary columns and perform data cleanning steps.
\\
\hline
\end{longtable}\sphinxatlongtableend\end{savenotes}


\begin{savenotes}\sphinxatlongtablestart\begin{longtable}[c]{\X{1}{2}\X{1}{2}}
\hline

\endfirsthead

\multicolumn{2}{c}%
{\makebox[0pt]{\sphinxtablecontinued{\tablename\ \thetable{} \textendash{} continued from previous page}}}\\
\hline

\endhead

\hline
\multicolumn{2}{r}{\makebox[0pt][r]{\sphinxtablecontinued{continues on next page}}}\\
\endfoot

\endlastfoot

{\hyperref[\detokenize{source/optimization.datatools:module-optimization.datatools.pipelines}]{\sphinxcrossref{\sphinxcode{\sphinxupquote{optimization.datatools.pipelines}}}}}
&
Pipelines used for adjusting input inventory reports to be passed to optimization model.
\\
\hline
\end{longtable}\sphinxatlongtableend\end{savenotes}


\begin{savenotes}\sphinxatlongtablestart\begin{longtable}[c]{\X{1}{2}\X{1}{2}}
\hline

\endfirsthead

\multicolumn{2}{c}%
{\makebox[0pt]{\sphinxtablecontinued{\tablename\ \thetable{} \textendash{} continued from previous page}}}\\
\hline

\endhead

\hline
\multicolumn{2}{r}{\makebox[0pt][r]{\sphinxtablecontinued{continues on next page}}}\\
\endfoot

\endlastfoot

{\hyperref[\detokenize{source/optimization.datatools:module-optimization.datatools.extra_output}]{\sphinxcrossref{\sphinxcode{\sphinxupquote{optimization.datatools.extra\_output}}}}}
&
Extra output scripts for creating output columns to optimization model.
\\
\hline
\end{longtable}\sphinxatlongtableend\end{savenotes}


\subsubsection{Dataprep}
\label{\detokenize{source/optimization.datatools:module-optimization.datatools.dataprep}}\label{\detokenize{source/optimization.datatools:dataprep}}\index{module@\spxentry{module}!optimization.datatools.dataprep@\spxentry{optimization.datatools.dataprep}}\index{optimization.datatools.dataprep@\spxentry{optimization.datatools.dataprep}!module@\spxentry{module}}
Functions created for data transformation or validation processes

Before optimizing inventory, we use the functions defined at this module for validating and manipulating
input data. Here we defined all necessary building blocks for calculating our necessary columns and also to
make sure that there are no potential inconsistencies in the input data that could impact model recommendaticosnatons.
These methods are used in the form of pipelines.
\index{adjust\_target\_simulation() (in module optimization.datatools.dataprep)@\spxentry{adjust\_target\_simulation()}\spxextra{in module optimization.datatools.dataprep}}

\begin{fulllineitems}
\phantomsection\label{\detokenize{source/optimization.datatools:optimization.datatools.dataprep.adjust_target_simulation}}\pysiglinewithargsret{\sphinxcode{\sphinxupquote{optimization.datatools.dataprep.}}\sphinxbfcode{\sphinxupquote{adjust\_target\_simulation}}}{\emph{\DUrole{n}{x\_df}\DUrole{p}{:} \DUrole{n}{pandas.core.frame.DataFrame}}}{}
Adjust the simulation time based on the fraction of the simulation day.

\begin{sphinxadmonition}{attention}{Attention:}
This function is not used by the model anymore. It was used during development to simulate
impact of transfer recommendations through time.
\end{sphinxadmonition}
\begin{quote}\begin{description}
\item[{Parameters}] \leavevmode
\sphinxstyleliteralstrong{\sphinxupquote{x\_df}} (\sphinxstyleliteralemphasis{\sphinxupquote{pd.DataFrame}}) \textendash{} Inventory report with \sphinxcode{\sphinxupquote{DOI Target}} to be adjusted

\item[{Returns}] \leavevmode
\sphinxstylestrong{x\_df (Object)}

\item[{Return type}] \leavevmode
Pandas dataframe.

\end{description}\end{quote}

\end{fulllineitems}

\index{bu\_qty\_on\_hand() (in module optimization.datatools.dataprep)@\spxentry{bu\_qty\_on\_hand()}\spxextra{in module optimization.datatools.dataprep}}

\begin{fulllineitems}
\phantomsection\label{\detokenize{source/optimization.datatools:optimization.datatools.dataprep.bu_qty_on_hand}}\pysiglinewithargsret{\sphinxcode{\sphinxupquote{optimization.datatools.dataprep.}}\sphinxbfcode{\sphinxupquote{bu\_qty\_on\_hand}}}{\emph{\DUrole{n}{x\_df}\DUrole{p}{:} \DUrole{n}{pandas.core.frame.DataFrame}}}{}
Recalculates the quantity each BU has on hand.

This calculation is necessary, since many BU qty on hand are actually wrong
and in some cases, considering items that have already expired.
\begin{quote}\begin{description}
\item[{Parameters}] \leavevmode
\sphinxstyleliteralstrong{\sphinxupquote{x\_df}} (\sphinxstyleliteralemphasis{\sphinxupquote{pd.DataFrame}}) \textendash{} 

\item[{Returns}] \leavevmode
\sphinxstylestrong{pd.DataFrame}

\item[{Return type}] \leavevmode
Dataframe with updated value for BU Qty on Hand.

\end{description}\end{quote}

\end{fulllineitems}

\index{change\_dtype() (in module optimization.datatools.dataprep)@\spxentry{change\_dtype()}\spxextra{in module optimization.datatools.dataprep}}

\begin{fulllineitems}
\phantomsection\label{\detokenize{source/optimization.datatools:optimization.datatools.dataprep.change_dtype}}\pysiglinewithargsret{\sphinxcode{\sphinxupquote{optimization.datatools.dataprep.}}\sphinxbfcode{\sphinxupquote{change\_dtype}}}{\emph{\DUrole{n}{x\_df}\DUrole{p}{:} \DUrole{n}{pandas.core.frame.DataFrame}}}{}
\end{fulllineitems}

\index{clean\_names() (in module optimization.datatools.dataprep)@\spxentry{clean\_names()}\spxextra{in module optimization.datatools.dataprep}}

\begin{fulllineitems}
\phantomsection\label{\detokenize{source/optimization.datatools:optimization.datatools.dataprep.clean_names}}\pysiglinewithargsret{\sphinxcode{\sphinxupquote{optimization.datatools.dataprep.}}\sphinxbfcode{\sphinxupquote{clean\_names}}}{\emph{\DUrole{n}{x\_df}\DUrole{p}{:} \DUrole{n}{pandas.core.frame.DataFrame}}}{}
Clean dataframe column names.

The method cleans column names by:
\begin{enumerate}
\sphinxsetlistlabels{\arabic}{enumi}{enumii}{}{.}%
\item {} 
Making everything lower case

\item {} 
Adding ‘\_’ instead of ‘ ‘

\item {} 
Removing special character ‘\$’

\item {} 
removing parenthesis ‘(‘ or ‘)’

\end{enumerate}
\begin{quote}\begin{description}
\item[{Parameters}] \leavevmode
\sphinxstyleliteralstrong{\sphinxupquote{x\_df}} (\sphinxstyleliteralemphasis{\sphinxupquote{pd.DataFrame}}) \textendash{} Dataframe with columns to be cleaned.

\item[{Returns}] \leavevmode
\sphinxstylestrong{x\_df} \textendash{} Dataframe with cleaned columns.

\item[{Return type}] \leavevmode
pd.DataFrame

\end{description}\end{quote}

\end{fulllineitems}

\index{column\_exists() (in module optimization.datatools.dataprep)@\spxentry{column\_exists()}\spxextra{in module optimization.datatools.dataprep}}

\begin{fulllineitems}
\phantomsection\label{\detokenize{source/optimization.datatools:optimization.datatools.dataprep.column_exists}}\pysiglinewithargsret{\sphinxcode{\sphinxupquote{optimization.datatools.dataprep.}}\sphinxbfcode{\sphinxupquote{column\_exists}}}{\emph{\DUrole{n}{x\_df}\DUrole{p}{:} \DUrole{n}{object}}, \emph{\DUrole{n}{column\_name}\DUrole{p}{:} \DUrole{n}{str}}, \emph{\DUrole{n}{condition}\DUrole{p}{:} \DUrole{n}{int} \DUrole{o}{=} \DUrole{default_value}{0}}}{}
Assert column exists in dataframe.
\begin{quote}\begin{description}
\item[{Parameters}] \leavevmode\begin{itemize}
\item {} 
\sphinxstyleliteralstrong{\sphinxupquote{x\_df}} (\sphinxstyleliteralemphasis{\sphinxupquote{pd.DataFrame}}) \textendash{} 

\item {} 
\sphinxstyleliteralstrong{\sphinxupquote{column\_name}} (\sphinxstyleliteralemphasis{\sphinxupquote{str}}) \textendash{} Name of the column we want to know if already exists.

\item {} 
\sphinxstyleliteralstrong{\sphinxupquote{condition}} (\sphinxstyleliteralemphasis{\sphinxupquote{int}}) \textendash{} 
If we want to assert that column exists or not. Defaults to 0.
\begin{itemize}
\item {} 
if condition = 0: assert that column exist.

\item {} 
if condition = 1: assert column doesn’t exist.

\end{itemize}


\end{itemize}

\item[{Raises}] \leavevmode
\sphinxstyleliteralstrong{\sphinxupquote{ValueError}} \textendash{} Error message to be displayed if assertion fails. \sphinxstylestrong{If test fail, this function will stop the model from running.}:

\end{description}\end{quote}

\end{fulllineitems}

\index{columns\_error\_handling() (in module optimization.datatools.dataprep)@\spxentry{columns\_error\_handling()}\spxextra{in module optimization.datatools.dataprep}}

\begin{fulllineitems}
\phantomsection\label{\detokenize{source/optimization.datatools:optimization.datatools.dataprep.columns_error_handling}}\pysiglinewithargsret{\sphinxcode{\sphinxupquote{optimization.datatools.dataprep.}}\sphinxbfcode{\sphinxupquote{columns\_error\_handling}}}{\emph{\DUrole{n}{x\_df}\DUrole{p}{:} \DUrole{n}{pandas.core.frame.DataFrame}}, \emph{\DUrole{n}{column\_list}\DUrole{p}{:} \DUrole{n}{list}}}{}
Handle column name errors.

For column names that were not found     in the original dataframe, we try to find If     the column is present but with a different name.

We also handle errors if columns are not found for specific     column names that usually are not present in the non\sphinxhyphen{}lot dataframe.
\begin{quote}\begin{description}
\item[{Parameters}] \leavevmode\begin{itemize}
\item {} 
\sphinxstyleliteralstrong{\sphinxupquote{x\_df}} (\sphinxstyleliteralemphasis{\sphinxupquote{pd.DataFrame}}) \textendash{} Inventory report

\item {} 
\sphinxstyleliteralstrong{\sphinxupquote{column\_list}} (\sphinxstyleliteralemphasis{\sphinxupquote{list}}) \textendash{} List of columns that the method needs to find.

\end{itemize}

\item[{Returns}] \leavevmode
\sphinxstylestrong{andas dataframe with all error handling transformations}

\item[{Return type}] \leavevmode
pd.DataFrame

\end{description}\end{quote}

\end{fulllineitems}

\index{columns\_needed() (in module optimization.datatools.dataprep)@\spxentry{columns\_needed()}\spxextra{in module optimization.datatools.dataprep}}

\begin{fulllineitems}
\phantomsection\label{\detokenize{source/optimization.datatools:optimization.datatools.dataprep.columns_needed}}\pysiglinewithargsret{\sphinxcode{\sphinxupquote{optimization.datatools.dataprep.}}\sphinxbfcode{\sphinxupquote{columns\_needed}}}{\emph{\DUrole{n}{x\_df}\DUrole{p}{:} \DUrole{n}{pandas.core.frame.DataFrame}}, \emph{\DUrole{n}{column\_list}\DUrole{p}{:} \DUrole{n}{list}}}{}
Select need columns.

The method selects only the columns needed based on list of columns needed
that is passed as an argument.
\begin{quote}\begin{description}
\item[{Parameters}] \leavevmode\begin{itemize}
\item {} 
\sphinxstyleliteralstrong{\sphinxupquote{x\_df}} (\sphinxstyleliteralemphasis{\sphinxupquote{pd.DataFrame}}) \textendash{} Dataframe with many unused columns.

\item {} 
\sphinxstyleliteralstrong{\sphinxupquote{column\_list}} (\sphinxstyleliteralemphasis{\sphinxupquote{list}}) \textendash{} List of columns to be mantained.

\end{itemize}

\item[{Returns}] \leavevmode
\sphinxstylestrong{x\_df} \textendash{} Dataframe with only the columns specified by the column\_list.

\item[{Return type}] \leavevmode
pd.DataFrame

\end{description}\end{quote}

\end{fulllineitems}

\index{combine\_lot\_non\_lot() (in module optimization.datatools.dataprep)@\spxentry{combine\_lot\_non\_lot()}\spxextra{in module optimization.datatools.dataprep}}

\begin{fulllineitems}
\phantomsection\label{\detokenize{source/optimization.datatools:optimization.datatools.dataprep.combine_lot_non_lot}}\pysiglinewithargsret{\sphinxcode{\sphinxupquote{optimization.datatools.dataprep.}}\sphinxbfcode{\sphinxupquote{combine\_lot\_non\_lot}}}{\emph{\DUrole{o}{*}\DUrole{n}{args}}}{}
Combine lot and non\sphinxhyphen{}lot inventories.
\begin{quote}\begin{description}
\item[{Returns}] \leavevmode
\sphinxstylestrong{object}

\item[{Return type}] \leavevmode
Combined inventory.

\end{description}\end{quote}

\end{fulllineitems}

\index{concat() (in module optimization.datatools.dataprep)@\spxentry{concat()}\spxextra{in module optimization.datatools.dataprep}}

\begin{fulllineitems}
\phantomsection\label{\detokenize{source/optimization.datatools:optimization.datatools.dataprep.concat}}\pysiglinewithargsret{\sphinxcode{\sphinxupquote{optimization.datatools.dataprep.}}\sphinxbfcode{\sphinxupquote{concat}}}{\emph{\DUrole{n}{results\_list}}}{}
Concatenates list of results.
\begin{quote}\begin{description}
\item[{Parameters}] \leavevmode
\sphinxstyleliteralstrong{\sphinxupquote{results\_list}} (\sphinxstyleliteralemphasis{\sphinxupquote{list}}) \textendash{} List with optimization results.

\item[{Returns}] \leavevmode
\sphinxstylestrong{Optimization results}

\item[{Return type}] \leavevmode
pd.DataFrame:

\end{description}\end{quote}

\end{fulllineitems}

\index{create\_column() (in module optimization.datatools.dataprep)@\spxentry{create\_column()}\spxextra{in module optimization.datatools.dataprep}}

\begin{fulllineitems}
\phantomsection\label{\detokenize{source/optimization.datatools:optimization.datatools.dataprep.create_column}}\pysiglinewithargsret{\sphinxcode{\sphinxupquote{optimization.datatools.dataprep.}}\sphinxbfcode{\sphinxupquote{create\_column}}}{\emph{\DUrole{n}{x\_df}\DUrole{p}{:} \DUrole{n}{pandas.core.frame.DataFrame}}, \emph{\DUrole{n}{column\_name}\DUrole{p}{:} \DUrole{n}{str}}, \emph{\DUrole{n}{default\_value}\DUrole{p}{:} \DUrole{n}{None}}}{}
Create new column based on passed name and fill with default\_value.

The new column will be filled with values specified by the \sphinxcode{\sphinxupquote{default\_value}}
variable. These values can range from \sphinxstylestrong{numbers} to \sphinxstylestrong{strings} and \sphinxstylestrong{even other column
values} from \sphinxcode{\sphinxupquote{x\_df}}.

If \sphinxcode{\sphinxupquote{default\_value}} is equal to a column name from x\_df, the new column
gets filled with values from that other column.

If \sphinxcode{\sphinxupquote{default\_value}} is not equal to a column name from \sphinxcode{\sphinxupquote{x\_df}}, and is of type \sphinxcode{\sphinxupquote{string}},
\sphinxcode{\sphinxupquote{int}} or \sphinxcode{\sphinxupquote{float}}, the new column gets filled with that value.

\begin{sphinxadmonition}{attention}{Attention:}
If a column with the same name as \sphinxcode{\sphinxupquote{column\_name}} already exists at \sphinxcode{\sphinxupquote{x\_df}} this
function will return an error and inform at \sphinxcode{\sphinxupquote{log file}} about its occurance.
\end{sphinxadmonition}
\begin{quote}\begin{description}
\item[{Parameters}] \leavevmode\begin{itemize}
\item {} 
\sphinxstyleliteralstrong{\sphinxupquote{x\_df}} (\sphinxstyleliteralemphasis{\sphinxupquote{pd.DataFrame}}) \textendash{} Dataframe to place new column

\item {} 
\sphinxstyleliteralstrong{\sphinxupquote{column\_name}} (\sphinxstyleliteralemphasis{\sphinxupquote{str}}) \textendash{} Name of new column to be created.

\item {} 
\sphinxstyleliteralstrong{\sphinxupquote{default\_value}} (\sphinxstyleliteralemphasis{\sphinxupquote{object}}) \textendash{} Value to be used as default value.

\end{itemize}

\item[{Returns}] \leavevmode
\sphinxstylestrong{x\_df} \textendash{} Dataframe with newly created column.

\item[{Return type}] \leavevmode
pd.DataFrame

\end{description}\end{quote}

\end{fulllineitems}

\index{cumulative\_days\_of\_inventory() (in module optimization.datatools.dataprep)@\spxentry{cumulative\_days\_of\_inventory()}\spxextra{in module optimization.datatools.dataprep}}

\begin{fulllineitems}
\phantomsection\label{\detokenize{source/optimization.datatools:optimization.datatools.dataprep.cumulative_days_of_inventory}}\pysiglinewithargsret{\sphinxcode{\sphinxupquote{optimization.datatools.dataprep.}}\sphinxbfcode{\sphinxupquote{cumulative\_days\_of\_inventory}}}{\emph{\DUrole{n}{x\_df}\DUrole{p}{:} \DUrole{n}{pandas.core.frame.DataFrame}}}{}
Calculate the cummulative days to expire for every SKU at every BU.

This method works like a window SQL function. It does some calculation
(in this case, the cummulative sum) based on a subgroup from our dataset.
\begin{quote}\begin{description}
\item[{Parameters}] \leavevmode
\sphinxstyleliteralstrong{\sphinxupquote{x\_df}} (\sphinxstyleliteralemphasis{\sphinxupquote{pd.DataFrame}}) \textendash{} Dataframe to be used.

\item[{Returns}] \leavevmode
\sphinxstylestrong{x\_df} \textendash{} Dataframe with the new column added.

\item[{Return type}] \leavevmode
pd.DataFrame

\end{description}\end{quote}


\sphinxstrong{See also:}


\sphinxhref{https://towardsdatascience.com/sql-window-functions-in-python-pandas-data-science-dc7c7a63cbb4}{towardsdatascience.com/sql\sphinxhyphen{}window\sphinxhyphen{}functions\sphinxhyphen{}in\sphinxhyphen{}python}



\end{fulllineitems}

\index{cumulative\_stock() (in module optimization.datatools.dataprep)@\spxentry{cumulative\_stock()}\spxextra{in module optimization.datatools.dataprep}}

\begin{fulllineitems}
\phantomsection\label{\detokenize{source/optimization.datatools:optimization.datatools.dataprep.cumulative_stock}}\pysiglinewithargsret{\sphinxcode{\sphinxupquote{optimization.datatools.dataprep.}}\sphinxbfcode{\sphinxupquote{cumulative\_stock}}}{\emph{\DUrole{n}{x\_df}\DUrole{p}{:} \DUrole{n}{pandas.core.frame.DataFrame}}}{}
Calculate the cummulative days to expire for every SKU at every BU.

This method works like a window SQL function. It does some calculation
(in this case, the cummulative sum) based on a subgroup from our dataset.
\begin{quote}\begin{description}
\item[{Parameters}] \leavevmode
\sphinxstyleliteralstrong{\sphinxupquote{x\_df}} (\sphinxstyleliteralemphasis{\sphinxupquote{pd.DataFrame}}) \textendash{} Dataframe to be used.

\item[{Returns}] \leavevmode
\sphinxstylestrong{x\_df} \textendash{} Dataframe with the new column added.

\item[{Return type}] \leavevmode
pd.DataFrame

\end{description}\end{quote}

\end{fulllineitems}

\index{date\_column() (in module optimization.datatools.dataprep)@\spxentry{date\_column()}\spxextra{in module optimization.datatools.dataprep}}

\begin{fulllineitems}
\phantomsection\label{\detokenize{source/optimization.datatools:optimization.datatools.dataprep.date_column}}\pysiglinewithargsret{\sphinxcode{\sphinxupquote{optimization.datatools.dataprep.}}\sphinxbfcode{\sphinxupquote{date\_column}}}{\emph{\DUrole{n}{x\_df}\DUrole{p}{:} \DUrole{n}{pandas.core.frame.DataFrame}}, \emph{\DUrole{n}{column\_list}\DUrole{p}{:} \DUrole{n}{list}}}{}
Convert columns from string to datetime.
\begin{quote}\begin{description}
\item[{Parameters}] \leavevmode\begin{itemize}
\item {} 
\sphinxstyleliteralstrong{\sphinxupquote{x\_df}} (\sphinxstyleliteralemphasis{\sphinxupquote{pd.DataFrame}}) \textendash{} Dataframe with column dates to be converted.

\item {} 
\sphinxstyleliteralstrong{\sphinxupquote{column\_list}} (\sphinxstyleliteralemphasis{\sphinxupquote{list}}) \textendash{} List of column names that need to be converted from string to datetime.

\end{itemize}

\item[{Returns}] \leavevmode
\sphinxstylestrong{x\_df} \textendash{} Dataframe converted columns.

\item[{Return type}] \leavevmode
pd.DataFrame

\end{description}\end{quote}

\end{fulllineitems}

\index{days\_of\_inv\_with\_transit() (in module optimization.datatools.dataprep)@\spxentry{days\_of\_inv\_with\_transit()}\spxextra{in module optimization.datatools.dataprep}}

\begin{fulllineitems}
\phantomsection\label{\detokenize{source/optimization.datatools:optimization.datatools.dataprep.days_of_inv_with_transit}}\pysiglinewithargsret{\sphinxcode{\sphinxupquote{optimization.datatools.dataprep.}}\sphinxbfcode{\sphinxupquote{days\_of\_inv\_with\_transit}}}{\emph{\DUrole{n}{row}}}{}
Calculates the days of the given row.
\begin{quote}\begin{description}
\item[{Parameters}] \leavevmode
\sphinxstyleliteralstrong{\sphinxupquote{row}} (\sphinxstyleliteralemphasis{\sphinxupquote{pd.Series}}) \textendash{} 

\item[{Returns}] \leavevmode
\sphinxstylestrong{Row based on conditional statement applied}

\item[{Return type}] \leavevmode
pd.Series

\end{description}\end{quote}

\end{fulllineitems}

\index{days\_of\_inventory() (in module optimization.datatools.dataprep)@\spxentry{days\_of\_inventory()}\spxextra{in module optimization.datatools.dataprep}}

\begin{fulllineitems}
\phantomsection\label{\detokenize{source/optimization.datatools:optimization.datatools.dataprep.days_of_inventory}}\pysiglinewithargsret{\sphinxcode{\sphinxupquote{optimization.datatools.dataprep.}}\sphinxbfcode{\sphinxupquote{days\_of\_inventory}}}{\emph{\DUrole{n}{x\_df}\DUrole{p}{:} \DUrole{n}{pandas.core.frame.DataFrame}}, \emph{\DUrole{n}{consider\_transit}\DUrole{p}{:} \DUrole{n}{bool} \DUrole{o}{=} \DUrole{default_value}{False}}}{}
Create new column with days of inventory.

Days of inventory is a calculation based on the number of items in stock
and the business unit average consumption rate for that item.

\begin{sphinxadmonition}{note}{Note:}
If we have an average consumption rate of zero, we should have an
infinite quantity of days of inventory. In reality, this means that
no item is being consumed at that business unit and so, we should send all
of those items to somewhere else.
\end{sphinxadmonition}
\begin{quote}\begin{description}
\item[{Parameters}] \leavevmode
\sphinxstyleliteralstrong{\sphinxupquote{x\_df}} (\sphinxstyleliteralemphasis{\sphinxupquote{pd.DataFrame}}) \textendash{} Dataframe to be used.

\item[{Returns}] \leavevmode
\sphinxstylestrong{x\_df} \textendash{} Dataframe with the new column added.

\item[{Return type}] \leavevmode
pd.DataFrame

\end{description}\end{quote}

\end{fulllineitems}

\index{days\_to\_expire() (in module optimization.datatools.dataprep)@\spxentry{days\_to\_expire()}\spxextra{in module optimization.datatools.dataprep}}

\begin{fulllineitems}
\phantomsection\label{\detokenize{source/optimization.datatools:optimization.datatools.dataprep.days_to_expire}}\pysiglinewithargsret{\sphinxcode{\sphinxupquote{optimization.datatools.dataprep.}}\sphinxbfcode{\sphinxupquote{days\_to\_expire}}}{\emph{\DUrole{n}{x\_df}\DUrole{p}{:} \DUrole{n}{pandas.core.frame.DataFrame}}}{}
Create new column with days to expire.
\begin{quote}\begin{description}
\item[{Parameters}] \leavevmode
\sphinxstyleliteralstrong{\sphinxupquote{x\_df}} (\sphinxstyleliteralemphasis{\sphinxupquote{pd.DataFrame}}) \textendash{} Dataframe to be used.

\item[{Returns}] \leavevmode
\sphinxstylestrong{x\_df} \textendash{} Dataframe with the new column added.

\item[{Return type}] \leavevmode
pd.DataFrame

\end{description}\end{quote}

\end{fulllineitems}

\index{delete\_column() (in module optimization.datatools.dataprep)@\spxentry{delete\_column()}\spxextra{in module optimization.datatools.dataprep}}

\begin{fulllineitems}
\phantomsection\label{\detokenize{source/optimization.datatools:optimization.datatools.dataprep.delete_column}}\pysiglinewithargsret{\sphinxcode{\sphinxupquote{optimization.datatools.dataprep.}}\sphinxbfcode{\sphinxupquote{delete\_column}}}{\emph{\DUrole{n}{x\_df}\DUrole{p}{:} \DUrole{n}{pandas.core.frame.DataFrame}}, \emph{\DUrole{n}{column\_name}\DUrole{p}{:} \DUrole{n}{str}}}{}
delete a column from x\_df
\begin{quote}\begin{description}
\item[{Parameters}] \leavevmode\begin{itemize}
\item {} 
\sphinxstyleliteralstrong{\sphinxupquote{x\_df}} (\sphinxstyleliteralemphasis{\sphinxupquote{pd.DataFrame}}) \textendash{} 

\item {} 
\sphinxstyleliteralstrong{\sphinxupquote{column\_name}} (\sphinxstyleliteralemphasis{\sphinxupquote{str}}) \textendash{} 

\end{itemize}

\end{description}\end{quote}

\end{fulllineitems}

\index{delta\_days\_of\_inventory() (in module optimization.datatools.dataprep)@\spxentry{delta\_days\_of\_inventory()}\spxextra{in module optimization.datatools.dataprep}}

\begin{fulllineitems}
\phantomsection\label{\detokenize{source/optimization.datatools:optimization.datatools.dataprep.delta_days_of_inventory}}\pysiglinewithargsret{\sphinxcode{\sphinxupquote{optimization.datatools.dataprep.}}\sphinxbfcode{\sphinxupquote{delta\_days\_of\_inventory}}}{\emph{\DUrole{n}{x\_df}\DUrole{p}{:} \DUrole{n}{pandas.core.frame.DataFrame}}}{}
Column with delta days of inventory.

Create new column with the difference in days between
the number of days of inventory and the days of inventory target for every
item/business unit.

Days of inventory is a calculation based on the number of items in stock
and the business unit average consumption rate for that item.
\begin{quote}\begin{description}
\item[{Parameters}] \leavevmode
\sphinxstyleliteralstrong{\sphinxupquote{x\_df}} (\sphinxstyleliteralemphasis{\sphinxupquote{pd.DataFrame}}) \textendash{} Dataframe to be used.

\item[{Returns}] \leavevmode
\sphinxstylestrong{x\_df} \textendash{} Dataframe with the new column added.

\item[{Return type}] \leavevmode
pd.DataFrame

\end{description}\end{quote}

\end{fulllineitems}

\index{fill\_na() (in module optimization.datatools.dataprep)@\spxentry{fill\_na()}\spxextra{in module optimization.datatools.dataprep}}

\begin{fulllineitems}
\phantomsection\label{\detokenize{source/optimization.datatools:optimization.datatools.dataprep.fill_na}}\pysiglinewithargsret{\sphinxcode{\sphinxupquote{optimization.datatools.dataprep.}}\sphinxbfcode{\sphinxupquote{fill\_na}}}{\emph{\DUrole{n}{x\_df}\DUrole{p}{:} \DUrole{n}{pandas.core.frame.DataFrame}}, \emph{\DUrole{n}{column\_name}\DUrole{p}{:} \DUrole{n}{str}}, \emph{\DUrole{n}{value}}}{}
Fill null rows from column based on specified value.
\begin{quote}\begin{description}
\item[{Parameters}] \leavevmode\begin{itemize}
\item {} 
\sphinxstyleliteralstrong{\sphinxupquote{x\_df}} (\sphinxstyleliteralemphasis{\sphinxupquote{pd.DataFrame}}) \textendash{} Dataframe to be used.

\item {} 
\sphinxstyleliteralstrong{\sphinxupquote{column\_name}} (\sphinxstyleliteralemphasis{\sphinxupquote{str}}) \textendash{} Name of the column with null values.

\item {} 
\sphinxstyleliteralstrong{\sphinxupquote{value}} (\sphinxstyleliteralemphasis{\sphinxupquote{object}}) \textendash{} 
Value to be used to fill null values.

Can be:
\begin{itemize}
\item {} 
A float or int.

\item {} 
’mean’: will fill with the mean value of the column.

\item {} 
str: will search for a column with this string name at x\_df columns.

\end{itemize}


\end{itemize}

\item[{Returns}] \leavevmode
\sphinxstylestrong{x\_df} \textendash{} Result dataframe.

\item[{Return type}] \leavevmode
pd.DataFrame

\end{description}\end{quote}

\end{fulllineitems}

\index{fix\_last\_results\_join() (in module optimization.datatools.dataprep)@\spxentry{fix\_last\_results\_join()}\spxextra{in module optimization.datatools.dataprep}}

\begin{fulllineitems}
\phantomsection\label{\detokenize{source/optimization.datatools:optimization.datatools.dataprep.fix_last_results_join}}\pysiglinewithargsret{\sphinxcode{\sphinxupquote{optimization.datatools.dataprep.}}\sphinxbfcode{\sphinxupquote{fix\_last\_results\_join}}}{\emph{\DUrole{n}{x\_df}\DUrole{p}{:} \DUrole{n}{pandas.core.frame.DataFrame}}}{}
Join the last results of the last query.
\begin{quote}\begin{description}
\item[{Parameters}] \leavevmode\begin{itemize}
\item {} 
\sphinxstyleliteralstrong{\sphinxupquote{x\_df}} (\sphinxstyleliteralemphasis{\sphinxupquote{pd.DataFrame}}) \textendash{} Removes \sphinxcode{\sphinxupquote{null}} values that appear after combining transfer recommendations back to the inventory report.

\item {} 
\sphinxstyleliteralstrong{\sphinxupquote{.}} \textendash{} 

\end{itemize}

\end{description}\end{quote}

\end{fulllineitems}

\index{get\_bu\_granularity() (in module optimization.datatools.dataprep)@\spxentry{get\_bu\_granularity()}\spxextra{in module optimization.datatools.dataprep}}

\begin{fulllineitems}
\phantomsection\label{\detokenize{source/optimization.datatools:optimization.datatools.dataprep.get_bu_granularity}}\pysiglinewithargsret{\sphinxcode{\sphinxupquote{optimization.datatools.dataprep.}}\sphinxbfcode{\sphinxupquote{get\_bu\_granularity}}}{\emph{\DUrole{n}{x\_df}\DUrole{p}{:} \DUrole{n}{pandas.core.frame.DataFrame}}}{}
Get aggregated inventory report by business unit and item id granularity.
\begin{quote}\begin{description}
\item[{Parameters}] \leavevmode
\sphinxstyleliteralstrong{\sphinxupquote{x\_df}} (\sphinxstyleliteralemphasis{\sphinxupquote{pd.DataFrame}}) \textendash{} 

\item[{Returns}] \leavevmode
\sphinxstylestrong{Dataframe with business unit/item granularity}

\item[{Return type}] \leavevmode
pd.DataFrame

\end{description}\end{quote}

\end{fulllineitems}

\index{get\_mean() (in module optimization.datatools.dataprep)@\spxentry{get\_mean()}\spxextra{in module optimization.datatools.dataprep}}

\begin{fulllineitems}
\phantomsection\label{\detokenize{source/optimization.datatools:optimization.datatools.dataprep.get_mean}}\pysiglinewithargsret{\sphinxcode{\sphinxupquote{optimization.datatools.dataprep.}}\sphinxbfcode{\sphinxupquote{get\_mean}}}{\emph{\DUrole{n}{x\_df}\DUrole{p}{:} \DUrole{n}{pandas.core.frame.DataFrame}}, \emph{\DUrole{n}{column\_name}\DUrole{p}{:} \DUrole{n}{str}}}{}
Mean value from column.
\begin{quote}\begin{description}
\item[{Parameters}] \leavevmode\begin{itemize}
\item {} 
\sphinxstyleliteralstrong{\sphinxupquote{x\_df}} (\sphinxstyleliteralemphasis{\sphinxupquote{pd.DataFrame}}) \textendash{} Dataframe to be used.

\item {} 
\sphinxstyleliteralstrong{\sphinxupquote{column\_name}} (\sphinxstyleliteralemphasis{\sphinxupquote{str}}) \textendash{} Name of the column to calculate mean value.

\end{itemize}

\item[{Returns}] \leavevmode
\sphinxstylestrong{mean\_value} \textendash{} Mean of column analyzed.

\item[{Return type}] \leavevmode
object

\end{description}\end{quote}

\end{fulllineitems}

\index{group\_by() (in module optimization.datatools.dataprep)@\spxentry{group\_by()}\spxextra{in module optimization.datatools.dataprep}}

\begin{fulllineitems}
\phantomsection\label{\detokenize{source/optimization.datatools:optimization.datatools.dataprep.group_by}}\pysiglinewithargsret{\sphinxcode{\sphinxupquote{optimization.datatools.dataprep.}}\sphinxbfcode{\sphinxupquote{group\_by}}}{\emph{\DUrole{n}{x\_df}\DUrole{p}{:} \DUrole{n}{pandas.core.frame.DataFrame}}, \emph{\DUrole{n}{group\_cols}\DUrole{p}{:} \DUrole{n}{Union\DUrole{p}{{[}}str\DUrole{p}{, }list\DUrole{p}{{]}}}}, \emph{\DUrole{n}{agg\_dict}\DUrole{p}{:} \DUrole{n}{dict}}}{}
Perform group by operation on a given DataFrame.
\begin{quote}\begin{description}
\item[{Parameters}] \leavevmode\begin{itemize}
\item {} 
\sphinxstyleliteralstrong{\sphinxupquote{x\_df}} (\sphinxstyleliteralemphasis{\sphinxupquote{pd.DataFrame}}) \textendash{} Inventory report to be used at the \sphinxcode{\sphinxupquote{groupby}}

\item {} 
\sphinxstyleliteralstrong{\sphinxupquote{group\_cols}} (\sphinxstyleliteralemphasis{\sphinxupquote{Union}}\sphinxstyleliteralemphasis{\sphinxupquote{{[}}}\sphinxstyleliteralemphasis{\sphinxupquote{str}}\sphinxstyleliteralemphasis{\sphinxupquote{, }}\sphinxstyleliteralemphasis{\sphinxupquote{list}}\sphinxstyleliteralemphasis{\sphinxupquote{{]}}}) \textendash{} column or list of columns to be used as \sphinxstylestrong{groupings}

\item {} 
\sphinxstyleliteralstrong{\sphinxupquote{agg\_dict}} (\sphinxstyleliteralemphasis{\sphinxupquote{dict}}) \textendash{} Dictionary with columns to be aggregated and their respective kind of aggregation.

\end{itemize}

\end{description}\end{quote}

\end{fulllineitems}

\index{handle\_name\_matching\_errors() (in module optimization.datatools.dataprep)@\spxentry{handle\_name\_matching\_errors()}\spxextra{in module optimization.datatools.dataprep}}

\begin{fulllineitems}
\phantomsection\label{\detokenize{source/optimization.datatools:optimization.datatools.dataprep.handle_name_matching_errors}}\pysiglinewithargsret{\sphinxcode{\sphinxupquote{optimization.datatools.dataprep.}}\sphinxbfcode{\sphinxupquote{handle\_name\_matching\_errors}}}{\emph{\DUrole{n}{x\_df}\DUrole{p}{:} \DUrole{n}{pandas.core.frame.DataFrame}}, \emph{\DUrole{n}{passed\_vars}\DUrole{p}{:} \DUrole{n}{dict}}}{}
Handles the process of trying to find the column name that corresponds
to the one we’re looking for.
\begin{quote}\begin{description}
\item[{Parameters}] \leavevmode\begin{itemize}
\item {} 
\sphinxstyleliteralstrong{\sphinxupquote{x\_df}} (\sphinxstyleliteralemphasis{\sphinxupquote{pd.DataFrame}}) \textendash{} Table with column names.

\item {} 
\sphinxstyleliteralstrong{\sphinxupquote{bu\_id}} (\sphinxstyleliteralemphasis{\sphinxupquote{bool}}\sphinxstyleliteralemphasis{\sphinxupquote{, }}\sphinxstyleliteralemphasis{\sphinxupquote{optional}}) \textendash{} If we need to find the column that corresponds to the business unit ID, defaults to False.

\item {} 
\sphinxstyleliteralstrong{\sphinxupquote{item\_id}} (\sphinxstyleliteralemphasis{\sphinxupquote{bool}}\sphinxstyleliteralemphasis{\sphinxupquote{, }}\sphinxstyleliteralemphasis{\sphinxupquote{optional}}) \textendash{} If we need to find the column that corresponds to the item ID, defaults to False.

\item {} 
\sphinxstyleliteralstrong{\sphinxupquote{cons}} (\sphinxstyleliteralemphasis{\sphinxupquote{bool}}\sphinxstyleliteralemphasis{\sphinxupquote{, }}\sphinxstyleliteralemphasis{\sphinxupquote{optional}}) \textendash{} If we need to find the column that corresponds to the item consumption, defaults to False.

\item {} 
\sphinxstyleliteralstrong{\sphinxupquote{report\_dt}} (\sphinxstyleliteralemphasis{\sphinxupquote{bool}}\sphinxstyleliteralemphasis{\sphinxupquote{, }}\sphinxstyleliteralemphasis{\sphinxupquote{optional}}) \textendash{} If we need to find the column that corresponds to the report date, defaults to False.

\end{itemize}

\item[{Returns}] \leavevmode
\sphinxstylestrong{Table with renamed columns if match was found}

\item[{Return type}] \leavevmode
pd.DataFrame

\end{description}\end{quote}

\end{fulllineitems}

\index{how\_many\_bus\_have\_item() (in module optimization.datatools.dataprep)@\spxentry{how\_many\_bus\_have\_item()}\spxextra{in module optimization.datatools.dataprep}}

\begin{fulllineitems}
\phantomsection\label{\detokenize{source/optimization.datatools:optimization.datatools.dataprep.how_many_bus_have_item}}\pysiglinewithargsret{\sphinxcode{\sphinxupquote{optimization.datatools.dataprep.}}\sphinxbfcode{\sphinxupquote{how\_many\_bus\_have\_item}}}{\emph{\DUrole{n}{x\_df}\DUrole{p}{:} \DUrole{n}{pandas.core.frame.DataFrame}}}{}
Calculate number of unique BU’s that have a givem item.

This function returns new column that gives us the number of
unique BU’s that have a given item. We use this information to
filter out Items that have only one BU.
\begin{quote}\begin{description}
\item[{Parameters}] \leavevmode
\sphinxstyleliteralstrong{\sphinxupquote{x\_df}} (\sphinxstyleliteralemphasis{\sphinxupquote{pd.DataFrame}}) \textendash{} 

\item[{Returns}] \leavevmode
\sphinxstylestrong{pd.DataFrame}

\item[{Return type}] \leavevmode
Dataframe with new column that has number of BU’s that have that a given Item.

\end{description}\end{quote}

\end{fulllineitems}

\index{item\_value\_importance() (in module optimization.datatools.dataprep)@\spxentry{item\_value\_importance()}\spxextra{in module optimization.datatools.dataprep}}

\begin{fulllineitems}
\phantomsection\label{\detokenize{source/optimization.datatools:optimization.datatools.dataprep.item_value_importance}}\pysiglinewithargsret{\sphinxcode{\sphinxupquote{optimization.datatools.dataprep.}}\sphinxbfcode{\sphinxupquote{item\_value\_importance}}}{\emph{\DUrole{n}{x\_df}\DUrole{p}{:} \DUrole{n}{pandas.core.frame.DataFrame}}}{{ $\rightarrow$ pandas.core.frame.DataFrame}}
Calculate value of item based on all inventory.

Calculate Decile for item total value
(total dollar value of given item if you sum the quantity from every BU).

For doing so, we create \sphinxstylestrong{3 new columns}:
\begin{itemize}
\item {} 
\sphinxcode{\sphinxupquote{item\_value}}: Total dolar value for every lot.

\item {} 
\sphinxcode{\sphinxupquote{total\_item\_value}}: Total dolar value for every item.

\item {} 
\sphinxcode{\sphinxupquote{DecileRank}}: Decile of \sphinxcode{\sphinxupquote{total\_item\_value}}.

\end{itemize}

\begin{sphinxadmonition}{note}{Note:}
If configuration for \sphinxcode{\sphinxupquote{USE\_DYNAMIC\_TIME}} is enabled, this function
allows the optimizer to determine the maximum amount of time that it can spend solving
a single \sphinxstylestrong{item id} based on the monetary importance of that item in relation to other Item IDs.

To enable or disable it, go to \sphinxcode{\sphinxupquote{optimization.constants}}.
\end{sphinxadmonition}

\end{fulllineitems}

\index{items\_to\_expire() (in module optimization.datatools.dataprep)@\spxentry{items\_to\_expire()}\spxextra{in module optimization.datatools.dataprep}}

\begin{fulllineitems}
\phantomsection\label{\detokenize{source/optimization.datatools:optimization.datatools.dataprep.items_to_expire}}\pysiglinewithargsret{\sphinxcode{\sphinxupquote{optimization.datatools.dataprep.}}\sphinxbfcode{\sphinxupquote{items\_to\_expire}}}{\emph{\DUrole{n}{x\_df}\DUrole{p}{:} \DUrole{n}{pandas.core.frame.DataFrame}}}{}
Create new column with items to expire.

Items to expire calculation should be calculated
at SKU/Business Unit granularity. This means that for
Business Units with many different Lots, we need to take
into consideration that those items from those Lots are going
to be consumed in a chainwise manner.
\subsubsection*{Example}

The example bellow demonstrates how we need to consider the relationship between
diffenents Lots at the same BU, with different expiration dates.


\begin{savenotes}\sphinxattablestart
\centering
\begin{tabulary}{\linewidth}[t]{|T|T|T|T|T|}
\hline
\sphinxstyletheadfamily 
Inv BU
&\sphinxstyletheadfamily 
Item ID
&\sphinxstyletheadfamily 
Lot ID
&\sphinxstyletheadfamily 
Lot Days to Expire
&\sphinxstyletheadfamily 
BU Cumulative Days to Expire
\\
\hline
BU A
&
SKU 1000
&
Lot H10
&
Days to Expire (10)
&
Cummulative Days to Expire (10)
\\
\hline
BU A
&
SKU 1000
&
Lot H15
&
Days to Expire (18)
&
Cummulative Days to Expire (28)
\\
\hline
BU A
&
SKU 1000
&
Lot X05
&
Days to Expire (25)
&
Cummulative Days to Expire (53)
\\
\hline
BU A
&
SKU 1000
&
Lot O90
&
Days to Expire (30)
&
Cummulative Days to Expire (83)
\\
\hline
BU A
&
SKU 1000
&
Lot L07
&
Days to Expire (50)
&
Cummulative Days to Expire (133)
\\
\hline
\end{tabulary}
\par
\sphinxattableend\end{savenotes}
\begin{quote}\begin{description}
\item[{Parameters}] \leavevmode
\sphinxstyleliteralstrong{\sphinxupquote{x\_df}} (\sphinxstyleliteralemphasis{\sphinxupquote{pd.DataFrame}}) \textendash{} Dataframe to be used.

\item[{Returns}] \leavevmode
\sphinxstylestrong{x\_df} \textendash{} Dataframe with days to expire calculated.

\item[{Return type}] \leavevmode
pd.DataFrame

\end{description}\end{quote}

\begin{sphinxadmonition}{note}{Note:}
Days to expire should be in ascending order.
\end{sphinxadmonition}

\end{fulllineitems}

\index{join() (in module optimization.datatools.dataprep)@\spxentry{join()}\spxextra{in module optimization.datatools.dataprep}}

\begin{fulllineitems}
\phantomsection\label{\detokenize{source/optimization.datatools:optimization.datatools.dataprep.join}}\pysiglinewithargsret{\sphinxcode{\sphinxupquote{optimization.datatools.dataprep.}}\sphinxbfcode{\sphinxupquote{join}}}{\emph{\DUrole{n}{x\_df}\DUrole{p}{:} \DUrole{n}{pandas.core.frame.DataFrame}}, \emph{\DUrole{n}{y\_df}\DUrole{p}{:} \DUrole{n}{pandas.core.frame.DataFrame}}, \emph{\DUrole{n}{on}\DUrole{p}{:} \DUrole{n}{Union\DUrole{p}{{[}}str\DUrole{p}{, }list\DUrole{p}{{]}}}}, \emph{\DUrole{n}{how}\DUrole{p}{:} \DUrole{n}{str} \DUrole{o}{=} \DUrole{default_value}{\textquotesingle{}left\textquotesingle{}}}}{}
Join dataframe.

Used to combine two DataFrames based on common set of keys and a specified type of join.
\begin{quote}\begin{description}
\item[{Parameters}] \leavevmode\begin{itemize}
\item {} 
\sphinxstyleliteralstrong{\sphinxupquote{x\_df}} (\sphinxstyleliteralemphasis{\sphinxupquote{pd.DataFrame}}) \textendash{} Left side Dataframe

\item {} 
\sphinxstyleliteralstrong{\sphinxupquote{y\_df}} (\sphinxstyleliteralemphasis{\sphinxupquote{pd.DataFrame}}) \textendash{} Right side Dataframe

\item {} 
\sphinxstyleliteralstrong{\sphinxupquote{on}} (\sphinxstyleliteralemphasis{\sphinxupquote{Union}}\sphinxstyleliteralemphasis{\sphinxupquote{{[}}}\sphinxstyleliteralemphasis{\sphinxupquote{str}}\sphinxstyleliteralemphasis{\sphinxupquote{, }}\sphinxstyleliteralemphasis{\sphinxupquote{list}}\sphinxstyleliteralemphasis{\sphinxupquote{{]}}}) \textendash{} Column or list of columns to be used as keys.

\item {} 
\sphinxstyleliteralstrong{\sphinxupquote{how}} (\sphinxstyleliteralemphasis{\sphinxupquote{str}}) \textendash{} How the join should be performed, by default ‘left’.

\end{itemize}

\item[{Returns}] \leavevmode
\sphinxstylestrong{Merged Dataframe}

\item[{Return type}] \leavevmode
pd.DataFrame

\end{description}\end{quote}

\end{fulllineitems}

\index{last\_transformations() (in module optimization.datatools.dataprep)@\spxentry{last\_transformations()}\spxextra{in module optimization.datatools.dataprep}}

\begin{fulllineitems}
\phantomsection\label{\detokenize{source/optimization.datatools:optimization.datatools.dataprep.last_transformations}}\pysiglinewithargsret{\sphinxcode{\sphinxupquote{optimization.datatools.dataprep.}}\sphinxbfcode{\sphinxupquote{last\_transformations}}}{\emph{\DUrole{n}{x\_df}\DUrole{p}{:} \DUrole{n}{pandas.core.frame.DataFrame}}}{}
Last column transformations.

Makes sure that if business unit has average
item daily use equal to zero, and items on
hand that these number of items is used as
doi balance.

Applies last columns transformations.
\begin{quote}\begin{description}
\item[{Parameters}] \leavevmode
\sphinxstyleliteralstrong{\sphinxupquote{x\_df}} (\sphinxstyleliteralemphasis{\sphinxupquote{pd.DataFrame}}) \textendash{} Dataframe to be used.

\item[{Returns}] \leavevmode
\sphinxstylestrong{x\_df} \textendash{} Dataframe with last transformed columns.

\item[{Return type}] \leavevmode
pd.DataFrame

\end{description}\end{quote}

\end{fulllineitems}

\index{rank\_column() (in module optimization.datatools.dataprep)@\spxentry{rank\_column()}\spxextra{in module optimization.datatools.dataprep}}

\begin{fulllineitems}
\phantomsection\label{\detokenize{source/optimization.datatools:optimization.datatools.dataprep.rank_column}}\pysiglinewithargsret{\sphinxcode{\sphinxupquote{optimization.datatools.dataprep.}}\sphinxbfcode{\sphinxupquote{rank\_column}}}{\emph{\DUrole{n}{x\_df}\DUrole{p}{:} \DUrole{n}{pandas.core.frame.DataFrame}}, \emph{\DUrole{n}{new\_column\_name}\DUrole{p}{:} \DUrole{n}{str}}, \emph{\DUrole{n}{group\_column}\DUrole{p}{:} \DUrole{n}{Union\DUrole{p}{{[}}str\DUrole{p}{, }list\DUrole{p}{{]}}}}, \emph{\DUrole{n}{rank\_column}\DUrole{p}{:} \DUrole{n}{str}}, \emph{\DUrole{n}{rank\_type}\DUrole{p}{:} \DUrole{n}{str} \DUrole{o}{=} \DUrole{default_value}{\textquotesingle{}dense\textquotesingle{}}}, \emph{\DUrole{n}{ascending}\DUrole{p}{:} \DUrole{n}{bool} \DUrole{o}{=} \DUrole{default_value}{True}}}{}
Ranks column values by specified groups.

Function divides inventory report into different column based groups.
It then determines the specified column’s rank relative to its
group—the function returns the initial input table with an additional
column with each row’s ranking.

Numerical data ranks go from 1 through n.

\begin{sphinxadmonition}{attention}{Attention:}
This function is not used by the model anymore. It was used during development to simulate
impact of transfer recommendations through time.
\end{sphinxadmonition}
\begin{quote}\begin{description}
\item[{Parameters}] \leavevmode\begin{itemize}
\item {} 
\sphinxstyleliteralstrong{\sphinxupquote{x\_df}} (\sphinxstyleliteralemphasis{\sphinxupquote{pd.DataFrame}}) \textendash{} Table with columns to be ranked.

\item {} 
\sphinxstyleliteralstrong{\sphinxupquote{new\_column\_name}} (\sphinxstyleliteralemphasis{\sphinxupquote{str}}) \textendash{} Name of the column that stores rank values.

\item {} 
\sphinxstyleliteralstrong{\sphinxupquote{group\_column}} (\sphinxstyleliteralemphasis{\sphinxupquote{Union}}\sphinxstyleliteralemphasis{\sphinxupquote{{[}}}\sphinxstyleliteralemphasis{\sphinxupquote{str}}\sphinxstyleliteralemphasis{\sphinxupquote{, }}\sphinxstyleliteralemphasis{\sphinxupquote{list}}\sphinxstyleliteralemphasis{\sphinxupquote{{]}}}) \textendash{} Used to determine the groups for the groups in which to rank by.

\item {} 
\sphinxstyleliteralstrong{\sphinxupquote{rank\_column}} (\sphinxstyleliteralemphasis{\sphinxupquote{str}}) \textendash{} Column with values to be ranked.

\item {} 
\sphinxstyleliteralstrong{\sphinxupquote{rank\_type}} (\sphinxstyleliteralemphasis{\sphinxupquote{str}}) \textendash{} 
Defaults to \sphinxcode{\sphinxupquote{dense}}. How to rank the group of records that have the same value (i.e. tries):
\begin{itemize}
\item {} 
\sphinxstylestrong{average:} average rank of the group

\item {} 
\sphinxstylestrong{min:} lowest rank in the group

\item {} 
\sphinxstylestrong{max:} highest rank in the group

\item {} 
\sphinxstylestrong{first:} ranks assigned in order they appear in the array

\item {} 
\sphinxstylestrong{dense:} like \sphinxcode{\sphinxupquote{min}}, but rank always increases by 1 between groups.

\end{itemize}


\end{itemize}

\end{description}\end{quote}
\begin{description}
\item[{ascending}] \leavevmode{[}bool, optional{]}
Whether or not the elements should be ranked in ascending order, defaults to \sphinxcode{\sphinxupquote{True.}}

\end{description}
\begin{quote}\begin{description}
\item[{Returns}] \leavevmode
\sphinxstylestrong{pd.DataFrame}

\item[{Return type}] \leavevmode
Returns Pandas \sphinxcode{\sphinxupquote{Series}} or \sphinxcode{\sphinxupquote{DataFrame}} with data ranks in the specified column.

\end{description}\end{quote}


\sphinxstrong{See also:}


\sphinxhref{http://pandas.pydata.org/pandas-docs/stable/reference/api/pandas.DataFrame.rank.html}{pandas.pydata.org/pandas\sphinxhyphen{}docs/stable/reference/api/pandas.DataFrame.rank.html}



\end{fulllineitems}

\index{remove\_expired\_items() (in module optimization.datatools.dataprep)@\spxentry{remove\_expired\_items()}\spxextra{in module optimization.datatools.dataprep}}

\begin{fulllineitems}
\phantomsection\label{\detokenize{source/optimization.datatools:optimization.datatools.dataprep.remove_expired_items}}\pysiglinewithargsret{\sphinxcode{\sphinxupquote{optimization.datatools.dataprep.}}\sphinxbfcode{\sphinxupquote{remove\_expired\_items}}}{\emph{\DUrole{n}{x\_df}\DUrole{p}{:} \DUrole{n}{pandas.core.frame.DataFrame}}}{}
Removes items that have expired.

Function verifies if \sphinxcode{\sphinxupquote{expire\_date}} is older than \sphinxcode{\sphinxupquote{report\_date}}.
If so, consider that \sphinxstylestrong{Lot as having zero items}.

\begin{sphinxadmonition}{attention}{Attention:}
We don’t remove that record from our table, since the business unit
might have an \sphinxcode{\sphinxupquote{average\_daily\_consumption\_rate}} bigger than \sphinxstylestrong{zero}.
If so, we might still want to transfer her items.
\end{sphinxadmonition}
\begin{quote}\begin{description}
\item[{Parameters}] \leavevmode
\sphinxstyleliteralstrong{\sphinxupquote{x\_df}} (\sphinxstyleliteralemphasis{\sphinxupquote{pd.DataFrame}}) \textendash{} Input table to be filtered.

\item[{Returns}] \leavevmode
Table without expired items.

\item[{Return type}] \leavevmode
pd.DataFrame

\end{description}\end{quote}

\end{fulllineitems}

\index{rename\_columns() (in module optimization.datatools.dataprep)@\spxentry{rename\_columns()}\spxextra{in module optimization.datatools.dataprep}}

\begin{fulllineitems}
\phantomsection\label{\detokenize{source/optimization.datatools:optimization.datatools.dataprep.rename_columns}}\pysiglinewithargsret{\sphinxcode{\sphinxupquote{optimization.datatools.dataprep.}}\sphinxbfcode{\sphinxupquote{rename\_columns}}}{\emph{\DUrole{n}{x\_df}\DUrole{p}{:} \DUrole{n}{pandas.core.frame.DataFrame}}, \emph{\DUrole{n}{old\_column\_name}\DUrole{p}{:} \DUrole{n}{str}}, \emph{\DUrole{n}{new\_column\_name}\DUrole{p}{:} \DUrole{n}{str}}}{}
Rename one or more columns
\begin{quote}\begin{description}
\item[{Parameters}] \leavevmode\begin{itemize}
\item {} 
\sphinxstyleliteralstrong{\sphinxupquote{x\_df}} (\sphinxstyleliteralemphasis{\sphinxupquote{pd.DataFrame}}) \textendash{} 

\item {} 
\sphinxstyleliteralstrong{\sphinxupquote{old\_column\_name}} (\sphinxstyleliteralemphasis{\sphinxupquote{str}}) \textendash{} 

\item {} 
\sphinxstyleliteralstrong{\sphinxupquote{new\_column\_name}} (\sphinxstyleliteralemphasis{\sphinxupquote{str}}) \textendash{} 

\end{itemize}

\end{description}\end{quote}

\end{fulllineitems}

\index{replace\_values() (in module optimization.datatools.dataprep)@\spxentry{replace\_values()}\spxextra{in module optimization.datatools.dataprep}}

\begin{fulllineitems}
\phantomsection\label{\detokenize{source/optimization.datatools:optimization.datatools.dataprep.replace_values}}\pysiglinewithargsret{\sphinxcode{\sphinxupquote{optimization.datatools.dataprep.}}\sphinxbfcode{\sphinxupquote{replace\_values}}}{\emph{\DUrole{n}{x\_df}\DUrole{p}{:} \DUrole{n}{pandas.core.frame.DataFrame}}, \emph{\DUrole{n}{column\_name}\DUrole{p}{:} \DUrole{n}{str}}, \emph{\DUrole{n}{value}\DUrole{p}{:} \DUrole{n}{None}}, \emph{\DUrole{n}{column\_transformed}\DUrole{p}{:} \DUrole{n}{str}}, \emph{\DUrole{n}{new\_value}\DUrole{p}{:} \DUrole{n}{None}}}{}
Replace Values on column based on condition.
\begin{quote}\begin{description}
\item[{Parameters}] \leavevmode\begin{itemize}
\item {} 
\sphinxstyleliteralstrong{\sphinxupquote{x\_df}} (\sphinxstyleliteralemphasis{\sphinxupquote{pd.DataFrame}}) \textendash{} Dataframe to be used.

\item {} 
\sphinxstyleliteralstrong{\sphinxupquote{column\_name}} (\sphinxstyleliteralemphasis{\sphinxupquote{str}}) \textendash{} Name of the column where condition applies.

\item {} 
\sphinxstyleliteralstrong{\sphinxupquote{value}} (\sphinxstyleliteralemphasis{\sphinxupquote{float}}) \textendash{} Value to be used in condition.

\item {} 
\sphinxstyleliteralstrong{\sphinxupquote{column\_transformed}} (\sphinxstyleliteralemphasis{\sphinxupquote{str}}) \textendash{} Column with values to modify.

\item {} 
\sphinxstyleliteralstrong{\sphinxupquote{new\_value}} (\sphinxstyleliteralemphasis{\sphinxupquote{float}}) \textendash{} New value to be added based on condition.

\end{itemize}

\item[{Returns}] \leavevmode
\sphinxstylestrong{x\_df} \textendash{} Dataframe with new column values.

\item[{Return type}] \leavevmode
pd.DataFrame

\end{description}\end{quote}

\end{fulllineitems}

\index{reset\_index() (in module optimization.datatools.dataprep)@\spxentry{reset\_index()}\spxextra{in module optimization.datatools.dataprep}}

\begin{fulllineitems}
\phantomsection\label{\detokenize{source/optimization.datatools:optimization.datatools.dataprep.reset_index}}\pysiglinewithargsret{\sphinxcode{\sphinxupquote{optimization.datatools.dataprep.}}\sphinxbfcode{\sphinxupquote{reset\_index}}}{\emph{\DUrole{n}{x\_df}\DUrole{p}{:} \DUrole{n}{pandas.core.frame.DataFrame}}}{}
Reset the index to a new dataframe.
\begin{quote}\begin{description}
\item[{Parameters}] \leavevmode
\sphinxstyleliteralstrong{\sphinxupquote{x\_df}} (\sphinxstyleliteralemphasis{\sphinxupquote{pd.DataFrame}}) \textendash{} 

\end{description}\end{quote}

\end{fulllineitems}

\index{substitute\_flag() (in module optimization.datatools.dataprep)@\spxentry{substitute\_flag()}\spxextra{in module optimization.datatools.dataprep}}

\begin{fulllineitems}
\phantomsection\label{\detokenize{source/optimization.datatools:optimization.datatools.dataprep.substitute_flag}}\pysiglinewithargsret{\sphinxcode{\sphinxupquote{optimization.datatools.dataprep.}}\sphinxbfcode{\sphinxupquote{substitute\_flag}}}{\emph{\DUrole{n}{x\_df}\DUrole{p}{:} \DUrole{n}{pandas.core.frame.DataFrame}}, \emph{\DUrole{n}{column\_name}\DUrole{p}{:} \DUrole{n}{str}}}{}
Substitute a flag in a column.

Changes column flags from \sphinxcode{\sphinxupquote{Yes}} and \sphinxcode{\sphinxupquote{No}} to \sphinxcode{\sphinxupquote{1}} and \sphinxcode{\sphinxupquote{0}}, respectively.
\begin{quote}\begin{description}
\item[{Parameters}] \leavevmode\begin{itemize}
\item {} 
\sphinxstyleliteralstrong{\sphinxupquote{x\_df}} (\sphinxstyleliteralemphasis{\sphinxupquote{pd.DataFrame}}) \textendash{} Inventory report.

\item {} 
\sphinxstyleliteralstrong{\sphinxupquote{column\_name}} (\sphinxstyleliteralemphasis{\sphinxupquote{str}}) \textendash{} Column with flag to be substituted.

\end{itemize}

\item[{Returns}] \leavevmode
\sphinxstylestrong{Dataframe with new flag type}

\item[{Return type}] \leavevmode
pd.DataFrame

\end{description}\end{quote}

\end{fulllineitems}

\index{try\_possible\_names() (in module optimization.datatools.dataprep)@\spxentry{try\_possible\_names()}\spxextra{in module optimization.datatools.dataprep}}

\begin{fulllineitems}
\phantomsection\label{\detokenize{source/optimization.datatools:optimization.datatools.dataprep.try_possible_names}}\pysiglinewithargsret{\sphinxcode{\sphinxupquote{optimization.datatools.dataprep.}}\sphinxbfcode{\sphinxupquote{try\_possible\_names}}}{\emph{\DUrole{n}{x\_df}\DUrole{p}{:} \DUrole{n}{pandas.core.frame.DataFrame}}, \emph{\DUrole{n}{column\_name}\DUrole{p}{:} \DUrole{n}{str}}, \emph{\DUrole{n}{possible\_names}\DUrole{p}{:} \DUrole{n}{list}}}{}
Rename column if found in list of possible names.

The method tries to find the column name in list of
specified possible names for that given column. If matching
value is found it renames the column using the name
specifieed at the column\_name variable.

When this method is called by the columns\_error\_handling
procedure, we don’t specify that this method needs to find the
column. If no match is found, we simlpy pass as a result the x\_df
itself. I didn’t add any exception if match is not found because not
necessarily that imposes an error, but chances are, the model will
return with error.
\subsubsection*{Example}
\begin{itemize}
\item {} 
Column report\_date was not found at x\_df.columns.

\item {} 
So we loop throung all the columns and try find if any of x\_df columns has one of the following names:
\begin{quote}

possible\_names = {[}‘query\_date’, …{]}
\end{quote}

\item {} 
Found that name inside x\_df columns so we rename the column from  query\_date to column\_name (= report\_date)

\end{itemize}
\begin{quote}\begin{description}
\item[{Parameters}] \leavevmode\begin{itemize}
\item {} 
\sphinxstyleliteralstrong{\sphinxupquote{x\_df}} (\sphinxstyleliteralemphasis{\sphinxupquote{pd.DataFrame}}) \textendash{} 

\item {} 
\sphinxstyleliteralstrong{\sphinxupquote{column\_name}} (\sphinxstyleliteralemphasis{\sphinxupquote{str}}) \textendash{} New column name.

\item {} 
\sphinxstyleliteralstrong{\sphinxupquote{possible\_names}} (\sphinxstyleliteralemphasis{\sphinxupquote{list}}) \textendash{} List of possible column names.

\end{itemize}

\item[{Returns}] \leavevmode
\sphinxstylestrong{Inventory report after trying to find the column amongst possible column names list}

\item[{Return type}] \leavevmode
pd.DataFrame

\end{description}\end{quote}

\end{fulllineitems}



\subsubsection{Extra Output}
\label{\detokenize{source/optimization.datatools:module-optimization.datatools.extra_output}}\label{\detokenize{source/optimization.datatools:extra-output}}\index{module@\spxentry{module}!optimization.datatools.extra\_output@\spxentry{optimization.datatools.extra\_output}}\index{optimization.datatools.extra\_output@\spxentry{optimization.datatools.extra\_output}!module@\spxentry{module}}
Extra output scripts for creating output columns to optimization model.
\index{FindLot (class in optimization.datatools.extra\_output)@\spxentry{FindLot}\spxextra{class in optimization.datatools.extra\_output}}

\begin{fulllineitems}
\phantomsection\label{\detokenize{source/optimization.datatools:optimization.datatools.extra_output.FindLot}}\pysiglinewithargsret{\sphinxbfcode{\sphinxupquote{class }}\sphinxcode{\sphinxupquote{optimization.datatools.extra\_output.}}\sphinxbfcode{\sphinxupquote{FindLot}}}{\emph{\DUrole{n}{inventory\_report}\DUrole{p}{:} \DUrole{n}{pandas.core.frame.DataFrame}}, \emph{\DUrole{n}{inv\_bu}\DUrole{p}{:} \DUrole{n}{object}}, \emph{\DUrole{n}{item\_id}\DUrole{p}{:} \DUrole{n}{object}}, \emph{\DUrole{n}{lot\_id}\DUrole{p}{:} \DUrole{n}{object}}}{}
Bases: \sphinxcode{\sphinxupquote{object}}

Determine Lot behaviour for item at BU.

Class created to determine if item from a given Lot is found or not
on some business unit. If found we then determine if it is being used
or not. Possible scenarios can be:
\begin{itemize}
\item {} 
\sphinxstylestrong{found:} BU has Lot we’re analyzing.
\begin{itemize}
\item {} 
\sphinxstylestrong{in use:} BU is currently using that Lot.

\item {} 
\sphinxstylestrong{in inventory:} BU is not currently using that Lot.

\end{itemize}

\item {} 
\sphinxstylestrong{not found:} Lot is not found at the BU we’re analyzing. (Can only happen for receiving BU’s).

\end{itemize}
\index{inventory\_report (optimization.datatools.extra\_output.FindLot attribute)@\spxentry{inventory\_report}\spxextra{optimization.datatools.extra\_output.FindLot attribute}}

\begin{fulllineitems}
\phantomsection\label{\detokenize{source/optimization.datatools:optimization.datatools.extra_output.FindLot.inventory_report}}\pysigline{\sphinxbfcode{\sphinxupquote{inventory\_report}}}
Inventory report table with information about all items from all BU’s.
\begin{quote}\begin{description}
\item[{Type}] \leavevmode
pd.DataFrame

\end{description}\end{quote}

\end{fulllineitems}

\index{inv\_bu (optimization.datatools.extra\_output.FindLot attribute)@\spxentry{inv\_bu}\spxextra{optimization.datatools.extra\_output.FindLot attribute}}

\begin{fulllineitems}
\phantomsection\label{\detokenize{source/optimization.datatools:optimization.datatools.extra_output.FindLot.inv_bu}}\pysigline{\sphinxbfcode{\sphinxupquote{inv\_bu}}}
Business Unit ID we’re analyzing.
\begin{quote}\begin{description}
\item[{Type}] \leavevmode
object

\end{description}\end{quote}

\end{fulllineitems}

\index{item\_id (optimization.datatools.extra\_output.FindLot attribute)@\spxentry{item\_id}\spxextra{optimization.datatools.extra\_output.FindLot attribute}}

\begin{fulllineitems}
\phantomsection\label{\detokenize{source/optimization.datatools:optimization.datatools.extra_output.FindLot.item_id}}\pysigline{\sphinxbfcode{\sphinxupquote{item\_id}}}
Item ID we’re analyzing.
\begin{quote}\begin{description}
\item[{Type}] \leavevmode
object

\end{description}\end{quote}

\end{fulllineitems}

\index{lot\_id (optimization.datatools.extra\_output.FindLot attribute)@\spxentry{lot\_id}\spxextra{optimization.datatools.extra\_output.FindLot attribute}}

\begin{fulllineitems}
\phantomsection\label{\detokenize{source/optimization.datatools:optimization.datatools.extra_output.FindLot.lot_id}}\pysigline{\sphinxbfcode{\sphinxupquote{lot\_id}}}
Lot ID we’re analyzing.
\begin{quote}\begin{description}
\item[{Type}] \leavevmode
object

\end{description}\end{quote}

\end{fulllineitems}

\index{expire\_status() (optimization.datatools.extra\_output.FindLot method)@\spxentry{expire\_status()}\spxextra{optimization.datatools.extra\_output.FindLot method}}

\begin{fulllineitems}
\phantomsection\label{\detokenize{source/optimization.datatools:optimization.datatools.extra_output.FindLot.expire_status}}\pysiglinewithargsret{\sphinxbfcode{\sphinxupquote{expire\_status}}}{\emph{\DUrole{n}{receiver\_doi}}}{}
\end{fulllineitems}

\index{get\_status() (optimization.datatools.extra\_output.FindLot method)@\spxentry{get\_status()}\spxextra{optimization.datatools.extra\_output.FindLot method}}

\begin{fulllineitems}
\phantomsection\label{\detokenize{source/optimization.datatools:optimization.datatools.extra_output.FindLot.get_status}}\pysiglinewithargsret{\sphinxbfcode{\sphinxupquote{get\_status}}}{\emph{\DUrole{n}{level}\DUrole{p}{:} \DUrole{n}{str} \DUrole{o}{=} \DUrole{default_value}{\textquotesingle{}lot\textquotesingle{}}}}{}
Return table with matching Item or Lot IDs.

If level is set to lot, filter inventory report by Item ID, Lot ID and BU ID.
Else filter by Item and BU ID only.
\begin{quote}\begin{description}
\item[{Parameters}] \leavevmode
\sphinxstyleliteralstrong{\sphinxupquote{level}} (\sphinxstyleliteralemphasis{\sphinxupquote{str}}\sphinxstyleliteralemphasis{\sphinxupquote{, }}\sphinxstyleliteralemphasis{\sphinxupquote{optional}}) \textendash{} Level that we want to filter inventory report, by default “lot”.

\item[{Returns}] \leavevmode
Records with matching criterias.

\item[{Return type}] \leavevmode
pd.DataFrame

\item[{Raises}] \leavevmode
\sphinxstyleliteralstrong{\sphinxupquote{AssertionError}} \textendash{} Raises an error when value passed to level is not \sphinxtitleref{lot} or \sphinxtitleref{item}.

\end{description}\end{quote}

\end{fulllineitems}

\index{lot\_usage() (optimization.datatools.extra\_output.FindLot method)@\spxentry{lot\_usage()}\spxextra{optimization.datatools.extra\_output.FindLot method}}

\begin{fulllineitems}
\phantomsection\label{\detokenize{source/optimization.datatools:optimization.datatools.extra_output.FindLot.lot_usage}}\pysiglinewithargsret{\sphinxbfcode{\sphinxupquote{lot\_usage}}}{}{}
Determine if lot is in use or not.
\begin{quote}\begin{description}
\item[{Returns}] \leavevmode
Returns flag determining if lot was found, is in use or in inventory.

\item[{Return type}] \leavevmode
str

\end{description}\end{quote}

\end{fulllineitems}


\end{fulllineitems}

\index{classify\_row() (in module optimization.datatools.extra\_output)@\spxentry{classify\_row()}\spxextra{in module optimization.datatools.extra\_output}}

\begin{fulllineitems}
\phantomsection\label{\detokenize{source/optimization.datatools:optimization.datatools.extra_output.classify_row}}\pysiglinewithargsret{\sphinxcode{\sphinxupquote{optimization.datatools.extra\_output.}}\sphinxbfcode{\sphinxupquote{classify\_row}}}{\emph{\DUrole{n}{row}}, \emph{\DUrole{n}{inventory\_report}}}{}
\end{fulllineitems}

\index{expire\_rows() (in module optimization.datatools.extra\_output)@\spxentry{expire\_rows()}\spxextra{in module optimization.datatools.extra\_output}}

\begin{fulllineitems}
\phantomsection\label{\detokenize{source/optimization.datatools:optimization.datatools.extra_output.expire_rows}}\pysiglinewithargsret{\sphinxcode{\sphinxupquote{optimization.datatools.extra\_output.}}\sphinxbfcode{\sphinxupquote{expire\_rows}}}{\emph{\DUrole{n}{receiver\_status}}, \emph{\DUrole{n}{expire\_status}}}{}
Populates \sphinxcode{\sphinxupquote{Transfer Recommendation Reason}} column, after determining that transfers had a reduction o \sphinxstylestrong{items to expire}.
\begin{quote}\begin{description}
\item[{Parameters}] \leavevmode\begin{itemize}
\item {} 
\sphinxstyleliteralstrong{\sphinxupquote{receiver\_status}} (\sphinxstyleliteralemphasis{\sphinxupquote{str}}) \textendash{} 
Receiver status contains information about the situation of a given item at the receiver end. Can be:
\begin{itemize}
\item {} 
\sphinxcode{\sphinxupquote{in use}} \sphinxhyphen{} Receiver has that given item from that given Lot and is currently using it.

\item {} 
\sphinxcode{\sphinxupquote{in inventory}} \sphinxhyphen{} Receiver has that given item \sphinxstylestrong{from that given Lot} but is \sphinxstylestrong{NOT} currently using it. \_

\item {} 
\sphinxcode{\sphinxupquote{not found}} \sphinxhyphen{}  Receiver doesn’t have that given item \sphinxstylestrong{from that given Lot} (and is not using it since it doesn’t have that Lot). \_

\end{itemize}


\item {} 
\sphinxstyleliteralstrong{\sphinxupquote{expire\_status}} (\sphinxstyleliteralemphasis{\sphinxupquote{str}}) \textendash{} 
Expire status contains information about the \sphinxstylestrong{expire situation} at the sender end.         Can be:
\begin{itemize}
\item {} 
\sphinxcode{\sphinxupquote{less than min}} \sphinxhyphen{} Sender has items to expire and less than minimum days of transfers applies.

\item {} 
\sphinxcode{\sphinxupquote{more than min}} \sphinxhyphen{} Sender has items to expire and more than minimum days of transfers applies.

\item {} 
\sphinxcode{\sphinxupquote{surplus}} \sphinxhyphen{} Sender has items to expire and surplus.

\end{itemize}


\end{itemize}

\item[{Returns}] \leavevmode
\sphinxstylestrong{Transfer reason to be added to the column \textasciigrave{}\textasciigrave{}Transfer Recommendation Reason\textasciigrave{}\textasciigrave{}}

\item[{Return type}] \leavevmode
str

\end{description}\end{quote}

\begin{sphinxadmonition}{note}{Note:}
\sphinxcode{\sphinxupquote{minimum days of transfers applies}} is a metric based on a column from the original report. \sphinxcode{\sphinxupquote{minimum days of transfers applies}} means that the total transfered     items accounts for an increase at the receiver days of inventory that is bigger than this \sphinxcode{\sphinxupquote{minimum days of transfers}} threshold. For example, if the \sphinxcode{\sphinxupquote{minimum days of transfers}}     is 14 days and you send to a receiver business unit 25 items and when we divide those transfers by the receivers \sphinxcode{\sphinxupquote{Avg Daily Consumption Rate}} results on total increase on days of inventory     of 30 days (in this case this would happen if \sphinxcode{\sphinxupquote{Avg Daily Consumption Rate}} = 0.833 Items/day) then \sphinxcode{\sphinxupquote{minimum days of transfers applies}}.
\end{sphinxadmonition}

\end{fulllineitems}

\index{extra\_output() (in module optimization.datatools.extra\_output)@\spxentry{extra\_output()}\spxextra{in module optimization.datatools.extra\_output}}

\begin{fulllineitems}
\phantomsection\label{\detokenize{source/optimization.datatools:optimization.datatools.extra_output.extra_output}}\pysiglinewithargsret{\sphinxcode{\sphinxupquote{optimization.datatools.extra\_output.}}\sphinxbfcode{\sphinxupquote{extra\_output}}}{\emph{\DUrole{n}{inventory\_report}\DUrole{p}{:} \DUrole{n}{pandas.core.frame.DataFrame}}}{{ $\rightarrow$ pandas.core.frame.DataFrame}}
Create extra output table.
\begin{description}
\item[{\sphinxstylestrong{Steps:}}] \leavevmode\begin{enumerate}
\sphinxsetlistlabels{\arabic}{enumi}{enumii}{}{.}%
\item {} 
Select necessary columns.

\item {} 
Rename necessary columns according to their specified names.

\end{enumerate}

\end{description}
\begin{quote}\begin{description}
\item[{Parameters}] \leavevmode
\sphinxstyleliteralstrong{\sphinxupquote{inventory\_report}} (\sphinxstyleliteralemphasis{\sphinxupquote{pd.DataFrame}}) \textendash{} Original inventory report.

\item[{Returns}] \leavevmode
\sphinxstylestrong{pd.DataFrame}

\item[{Return type}] \leavevmode
Extra output table.

\end{description}\end{quote}

\end{fulllineitems}

\index{get\_status() (in module optimization.datatools.extra\_output)@\spxentry{get\_status()}\spxextra{in module optimization.datatools.extra\_output}}

\begin{fulllineitems}
\phantomsection\label{\detokenize{source/optimization.datatools:optimization.datatools.extra_output.get_status}}\pysiglinewithargsret{\sphinxcode{\sphinxupquote{optimization.datatools.extra\_output.}}\sphinxbfcode{\sphinxupquote{get\_status}}}{\emph{\DUrole{n}{inventory\_report}}, \emph{\DUrole{n}{row}}, \emph{\DUrole{n}{what}\DUrole{p}{:} \DUrole{n}{str} \DUrole{o}{=} \DUrole{default_value}{\textquotesingle{}both\textquotesingle{}}}}{}
Get status of Lot for receiver and sender BU.
\begin{quote}\begin{description}
\item[{Parameters}] \leavevmode\begin{itemize}
\item {} 
\sphinxstyleliteralstrong{\sphinxupquote{inventory\_report}} (\sphinxstyleliteralemphasis{\sphinxupquote{pd.DataFrame}}) \textendash{} Table with inventory report.

\item {} 
\sphinxstyleliteralstrong{\sphinxupquote{row}} (\sphinxstyleliteralemphasis{\sphinxupquote{pd.Series}}) \textendash{} Row from optimization report we’re analyzing.

\item {} 
\sphinxstyleliteralstrong{\sphinxupquote{what}} (\sphinxstyleliteralemphasis{\sphinxupquote{str}}\sphinxstyleliteralemphasis{\sphinxupquote{, }}\sphinxstyleliteralemphasis{\sphinxupquote{optional}}) \textendash{} Get status for both receiver and sender or only receiver or receiver DOI status, by default “both”.

\end{itemize}

\item[{Returns}] \leavevmode
Status of lot at receiver bu and sender bu

\item[{Return type}] \leavevmode
str

\end{description}\end{quote}

\end{fulllineitems}

\index{optimizable() (in module optimization.datatools.extra\_output)@\spxentry{optimizable()}\spxextra{in module optimization.datatools.extra\_output}}

\begin{fulllineitems}
\phantomsection\label{\detokenize{source/optimization.datatools:optimization.datatools.extra_output.optimizable}}\pysiglinewithargsret{\sphinxcode{\sphinxupquote{optimization.datatools.extra\_output.}}\sphinxbfcode{\sphinxupquote{optimizable}}}{\emph{\DUrole{n}{df\_inventory}\DUrole{p}{:} \DUrole{n}{pandas.core.frame.DataFrame}}, \emph{\DUrole{n}{item\_id}\DUrole{p}{:} \DUrole{n}{int}}}{{ $\rightarrow$ tuple}}
Calculates the maximum amount of optimizable inventory in terms of \sphinxcode{\sphinxupquote{Items to Expire}} and \sphinxcode{\sphinxupquote{Inventory Balance}}

This method is called by \sphinxcode{\sphinxupquote{optimization.model.main.ModelOptimization.optimize}}
in order to calculate the theoretical maximum optimizable quantity for a given \sphinxcode{\sphinxupquote{Item ID}} in terms of \sphinxcode{\sphinxupquote{Items to Expire}} and \sphinxcode{\sphinxupquote{Inventory Balance}}.

In order to obtain these values (\sphinxcode{\sphinxupquote{Items to Expire}} and \sphinxcode{\sphinxupquote{Inventory Balance}} theoretical maximum) we filter the inventory using the \sphinxcode{\sphinxupquote{Item ID}} column,
then we group all values by business unit, aggregating into sepparate fields the negative amount of inventory balance from the positive part. Then we aggregate
another time by \sphinxcode{\sphinxupquote{Item ID}}. For inventory balance, the \sphinxstylestrong{theoretical maximum optimizable inventory will be the smallest number} between \sphinxcode{\sphinxupquote{Surplus}} and \sphinxcode{\sphinxupquote{Shortage}}
(since we can only reduce shortage from one BU by eliminating surplus from another).
\begin{quote}\begin{description}
\item[{Parameters}] \leavevmode\begin{itemize}
\item {} 
\sphinxstyleliteralstrong{\sphinxupquote{df\_inventory}} (\sphinxstyleliteralemphasis{\sphinxupquote{pd.DataFrame}}) \textendash{} Inventory DataFrame.

\item {} 
\sphinxstyleliteralstrong{\sphinxupquote{item\_id}} (\sphinxstyleliteralemphasis{\sphinxupquote{int}}) \textendash{} \sphinxcode{\sphinxupquote{Item ID}} that we want to determine the maximum optimizable inventory.

\end{itemize}

\item[{Returns}] \leavevmode
Total \$ optimizable amount fo inventory balance and items to expire (in that specific order).

\item[{Return type}] \leavevmode
tuple

\end{description}\end{quote}

\end{fulllineitems}

\index{surplus\_rows() (in module optimization.datatools.extra\_output)@\spxentry{surplus\_rows()}\spxextra{in module optimization.datatools.extra\_output}}

\begin{fulllineitems}
\phantomsection\label{\detokenize{source/optimization.datatools:optimization.datatools.extra_output.surplus_rows}}\pysiglinewithargsret{\sphinxcode{\sphinxupquote{optimization.datatools.extra\_output.}}\sphinxbfcode{\sphinxupquote{surplus\_rows}}}{\emph{\DUrole{n}{receiver\_status}}, \emph{\DUrole{n}{sender\_status}}}{}
Classify rows with surplus at sender BU.
\begin{quote}\begin{description}
\item[{Parameters}] \leavevmode\begin{itemize}
\item {} 
\sphinxstyleliteralstrong{\sphinxupquote{receiver\_status}} (\sphinxstyleliteralemphasis{\sphinxupquote{str}}) \textendash{} Status of Lot for receiver BU.

\item {} 
\sphinxstyleliteralstrong{\sphinxupquote{sender\_status}} (\sphinxstyleliteralemphasis{\sphinxupquote{str}}) \textendash{} Status of Lot for sender BU.

\end{itemize}

\item[{Returns}] \leavevmode
Recommendation reason classification for suplus.

\item[{Return type}] \leavevmode
str

\end{description}\end{quote}

\end{fulllineitems}

\index{transfer\_reason() (in module optimization.datatools.extra\_output)@\spxentry{transfer\_reason()}\spxextra{in module optimization.datatools.extra\_output}}

\begin{fulllineitems}
\phantomsection\label{\detokenize{source/optimization.datatools:optimization.datatools.extra_output.transfer_reason}}\pysiglinewithargsret{\sphinxcode{\sphinxupquote{optimization.datatools.extra\_output.}}\sphinxbfcode{\sphinxupquote{transfer\_reason}}}{\emph{\DUrole{n}{inventory\_report}\DUrole{p}{:} \DUrole{n}{pandas.core.frame.DataFrame}}, \emph{\DUrole{n}{optimization\_results}\DUrole{p}{:} \DUrole{n}{pandas.core.frame.DataFrame}}}{{ $\rightarrow$ pandas.core.frame.DataFrame}}
Populate \sphinxcode{\sphinxupquote{Transfer Recommendation Reason}} column.
\begin{quote}\begin{description}
\item[{Parameters}] \leavevmode\begin{itemize}
\item {} 
\sphinxstyleliteralstrong{\sphinxupquote{inventory\_report}} (\sphinxstyleliteralemphasis{\sphinxupquote{pd.DataFrame}}) \textendash{} Original inventory report \sphinxstylestrong{after data ingestion and validation processess}

\item {} 
\sphinxstyleliteralstrong{\sphinxupquote{optimization\_results}} (\sphinxstyleliteralemphasis{\sphinxupquote{pd.DataFrame}}) \textendash{} {[}description{]}

\end{itemize}

\item[{Returns}] \leavevmode
{[}description{]}

\item[{Return type}] \leavevmode
pd.DataFrame

\end{description}\end{quote}

\begin{sphinxadmonition}{tip}{Tip:}
The cleaned version of inventory reportm is obtained inside {\hyperref[\detokenize{source/optimization.model:optimization.model.main.main}]{\sphinxcrossref{\sphinxcode{\sphinxupquote{optimization.model.main.main()}}}}} method.
\end{sphinxadmonition}

\end{fulllineitems}



\subsubsection{Pipelines}
\label{\detokenize{source/optimization.datatools:module-optimization.datatools.pipelines}}\label{\detokenize{source/optimization.datatools:pipelines}}\index{module@\spxentry{module}!optimization.datatools.pipelines@\spxentry{optimization.datatools.pipelines}}\index{optimization.datatools.pipelines@\spxentry{optimization.datatools.pipelines}!module@\spxentry{module}}
Pipelines used for adjusting input inventory reports to be passed to optimization model.
\index{active\_status() (in module optimization.datatools.pipelines)@\spxentry{active\_status()}\spxextra{in module optimization.datatools.pipelines}}

\begin{fulllineitems}
\phantomsection\label{\detokenize{source/optimization.datatools:optimization.datatools.pipelines.active_status}}\pysiglinewithargsret{\sphinxcode{\sphinxupquote{optimization.datatools.pipelines.}}\sphinxbfcode{\sphinxupquote{active\_status}}}{\emph{\DUrole{n}{row1}}, \emph{\DUrole{n}{before}}}{}
\end{fulllineitems}

\index{calculations\_pipeline() (in module optimization.datatools.pipelines)@\spxentry{calculations\_pipeline()}\spxextra{in module optimization.datatools.pipelines}}

\begin{fulllineitems}
\phantomsection\label{\detokenize{source/optimization.datatools:optimization.datatools.pipelines.calculations_pipeline}}\pysiglinewithargsret{\sphinxcode{\sphinxupquote{optimization.datatools.pipelines.}}\sphinxbfcode{\sphinxupquote{calculations\_pipeline}}}{\emph{\DUrole{n}{x\_df}\DUrole{p}{:} \DUrole{n}{pandas.core.frame.DataFrame}}}{{ $\rightarrow$ pandas.core.frame.DataFrame}}
Based on our input reports, calculate necessary columns that will we used by the optimization model.

\sphinxstylestrong{Added Calculated Columns:}
\begin{itemize}
\item {} 
\sphinxcode{\sphinxupquote{Inventory Balance}}: How far from the target number of days of inventory each item at each business unit are.

\item {} 
\sphinxcode{\sphinxupquote{Days to Expire}}: Total number of days that each Lot item has before expiring.

\item {} \begin{description}
\item[{\sphinxcode{\sphinxupquote{Items to Expire}}: Based on the business unit consumption rate for that item, and the quantity it has on hand, how}] \leavevmode
how many of those items will expire before the business unit has a chance to consume them.

\end{description}

\end{itemize}
\begin{quote}\begin{description}
\item[{Parameters}] \leavevmode
\sphinxstyleliteralstrong{\sphinxupquote{x\_df}} (\sphinxstyleliteralemphasis{\sphinxupquote{pd.DataFrame}}) \textendash{} Combined Inventory report for lot and non\sphinxhyphen{}lot items.

\item[{Returns}] \leavevmode
\sphinxstylestrong{pd.DataFrame} \textendash{} Table with additional columns for \sphinxcode{\sphinxupquote{inventory balance}} and \sphinxcode{\sphinxupquote{items to expire}}

\item[{Return type}] \leavevmode
pd.DataFrame

\end{description}\end{quote}

\begin{sphinxadmonition}{note}{Note:}
If you want to add new calculations to the report, we recommend adding the calculations to this pipeline.
Thats because there are other steps that are performed before that clean our input data and performing the calculations
before might give you bad results.
\end{sphinxadmonition}

\end{fulllineitems}

\index{clean\_data() (in module optimization.datatools.pipelines)@\spxentry{clean\_data()}\spxextra{in module optimization.datatools.pipelines}}

\begin{fulllineitems}
\phantomsection\label{\detokenize{source/optimization.datatools:optimization.datatools.pipelines.clean_data}}\pysiglinewithargsret{\sphinxcode{\sphinxupquote{optimization.datatools.pipelines.}}\sphinxbfcode{\sphinxupquote{clean\_data}}}{\emph{\DUrole{n}{x\_df}\DUrole{p}{:} \DUrole{n}{pandas.core.frame.DataFrame}}}{{ $\rightarrow$ pandas.core.frame.DataFrame}}
Filter input data so we only consider datapoints that can be optimized and that we’re removing possible data with errors.

The filters that are applied in this function take into consideration
not only the data types that the columns have (or that they should have).
It also considers aspects related to what that column really means.
\subsubsection*{Example}

The column \sphinxcode{\sphinxupquote{bu\_qty\_on\_hand}} represents the quantity of items a given BU has.
It can be expected that this column should have only numerical values that are integers
(we can’t have half an item) but also that they should be, at least theoretically bigger
or equal to zero.

\sphinxstylestrong{Steps}
\begin{itemize}
\item {} 
\sphinxstylestrong{Step 1:} Replaces columns can\_transfer\_inventory and can\_receive\_inventory values from {[}Y, N{]} to {[}1, 0{]}

\item {} 
\sphinxstylestrong{Step 2:} Changes columns with rows that have dates from str to real dates dtype

\item {} 
\sphinxstylestrong{Step 3:} Fill blank values tha can impact in the model

\item {} 
\sphinxstylestrong{Step 4:} Filter out rows with price value equal to zero.

\item {} 
\sphinxstylestrong{Step 5:} Make sure BU can at least receive or transfer inventory

\item {} 
\sphinxstylestrong{Step 6:} Filter out SKU’s that have only one BU that uses them

\item {} 
\sphinxstylestrong{Step 7:} Removes expired items. This is not the same as deleting the entire row. We just consider that given Lot to have 0 items on hand.

\end{itemize}
\begin{quote}\begin{description}
\item[{Parameters}] \leavevmode
\sphinxstyleliteralstrong{\sphinxupquote{(}}\sphinxstyleliteralstrong{\sphinxupquote{pd.DataFrame}}\sphinxstyleliteralstrong{\sphinxupquote{)}} (\sphinxstyleliteralemphasis{\sphinxupquote{x\_df}}) \textendash{} 

\item[{Returns}] \leavevmode
\sphinxstylestrong{pd.DataFrame}

\item[{Return type}] \leavevmode
Cleaned dataframe

\end{description}\end{quote}

\end{fulllineitems}

\index{data\_pipeline() (in module optimization.datatools.pipelines)@\spxentry{data\_pipeline()}\spxextra{in module optimization.datatools.pipelines}}

\begin{fulllineitems}
\phantomsection\label{\detokenize{source/optimization.datatools:optimization.datatools.pipelines.data_pipeline}}\pysiglinewithargsret{\sphinxcode{\sphinxupquote{optimization.datatools.pipelines.}}\sphinxbfcode{\sphinxupquote{data\_pipeline}}}{\emph{\DUrole{n}{lot\_df}\DUrole{p}{:} \DUrole{n}{pandas.core.frame.DataFrame}}, \emph{\DUrole{n}{nonlot\_df}\DUrole{p}{:} \DUrole{n}{pandas.core.frame.DataFrame}}, \emph{\DUrole{n}{period}\DUrole{p}{:} \DUrole{n}{int} \DUrole{o}{=} \DUrole{default_value}{0}}, \emph{\DUrole{n}{last\_month\_results}\DUrole{p}{:} \DUrole{n}{None} \DUrole{o}{=} \DUrole{default_value}{None}}}{{ $\rightarrow$ pandas.core.frame.DataFrame}}
Pipeline with data transformations.
\begin{quote}\begin{description}
\item[{Parameters}] \leavevmode\begin{itemize}
\item {} 
\sphinxstyleliteralstrong{\sphinxupquote{period}} (\sphinxstyleliteralemphasis{\sphinxupquote{(}}\sphinxstyleliteralemphasis{\sphinxupquote{int}}\sphinxstyleliteralemphasis{\sphinxupquote{, }}\sphinxstyleliteralemphasis{\sphinxupquote{optional}}\sphinxstyleliteralemphasis{\sphinxupquote{)}}) \textendash{} Period of simulation. Defaults to 0.

\item {} 
\sphinxstyleliteralstrong{\sphinxupquote{last\_month\_results}} (\sphinxstyleliteralemphasis{\sphinxupquote{(}}\sphinxstyleliteralemphasis{\sphinxupquote{None}}\sphinxstyleliteralemphasis{\sphinxupquote{, }}\sphinxstyleliteralemphasis{\sphinxupquote{optional}}\sphinxstyleliteralemphasis{\sphinxupquote{)}}) \textendash{} Results from last month simulation. Defaults to None.

\end{itemize}

\item[{Returns}] \leavevmode
\sphinxstylestrong{pd.DataFrame}

\item[{Return type}] \leavevmode
Current month inventory.

\end{description}\end{quote}

\end{fulllineitems}

\index{doi\_balance\_pipeline() (in module optimization.datatools.pipelines)@\spxentry{doi\_balance\_pipeline()}\spxextra{in module optimization.datatools.pipelines}}

\begin{fulllineitems}
\phantomsection\label{\detokenize{source/optimization.datatools:optimization.datatools.pipelines.doi_balance_pipeline}}\pysiglinewithargsret{\sphinxcode{\sphinxupquote{optimization.datatools.pipelines.}}\sphinxbfcode{\sphinxupquote{doi\_balance\_pipeline}}}{\emph{\DUrole{n}{x\_df}\DUrole{p}{:} \DUrole{n}{pandas.core.frame.DataFrame}}}{{ $\rightarrow$ pandas.core.frame.DataFrame}}
Calculate DOI balance pipeline.
\begin{quote}\begin{description}
\item[{Parameters}] \leavevmode
\sphinxstyleliteralstrong{\sphinxupquote{x\_df}} (\sphinxstyleliteralemphasis{\sphinxupquote{pd.DataFrame}}) \textendash{} 

\end{description}\end{quote}

\end{fulllineitems}

\index{load\_data\_pipeline() (in module optimization.datatools.pipelines)@\spxentry{load\_data\_pipeline()}\spxextra{in module optimization.datatools.pipelines}}

\begin{fulllineitems}
\phantomsection\label{\detokenize{source/optimization.datatools:optimization.datatools.pipelines.load_data_pipeline}}\pysiglinewithargsret{\sphinxcode{\sphinxupquote{optimization.datatools.pipelines.}}\sphinxbfcode{\sphinxupquote{load\_data\_pipeline}}}{\emph{\DUrole{n}{x\_df}\DUrole{p}{:} \DUrole{n}{pandas.core.frame.DataFrame}}, \emph{\DUrole{n}{columns\_list}\DUrole{p}{:} \DUrole{n}{list}}, \emph{\DUrole{n}{non\_lot}\DUrole{p}{:} \DUrole{n}{bool} \DUrole{o}{=} \DUrole{default_value}{True}}}{{ $\rightarrow$ pandas.core.frame.DataFrame}}
Pipeline used for loading input files and selectiong necessary columns.

This function is called by \sphinxcode{\sphinxupquote{data\_pipeline}} and the following steps are performed:
\begin{itemize}
\item {} 
\sphinxstylestrong{Step 1:} \sphinxcode{\sphinxupquote{clean\_names}} Clean column names. This step is used to make sure possible inconsistencies on column names won’t affect the optimization model.

\item {} 
\sphinxstylestrong{Step 2 (OPTIONAL):} \sphinxcode{\sphinxupquote{change\_dtype}} Change some columns data types from \sphinxcode{\sphinxupquote{int64}} to \sphinxcode{\sphinxupquote{int32}} or \sphinxcode{\sphinxupquote{float64}} to \sphinxcode{\sphinxupquote{float32}} to improve performance.

\item {} 
\sphinxstylestrong{Step 3:} \sphinxcode{\sphinxupquote{columns\_error\_handling}} If one of the necessary columns needed by the model is not found, this step tries to find the column based on other possible names for that given column. These lists of possible alternative column names are stored inside the method and were populated with names seen on different inventory reports throughout development phase.

\item {} 
\sphinxstylestrong{Step 4:} \sphinxcode{\sphinxupquote{columns\_needed}} After making sure all columns were found, select all necessary columns. These columns are required either on the \sphinxstylestrong{final output report or by the model}.

\item {} 
\sphinxstylestrong{Step 5 (For Non Lot Only):} Populates column \sphinxcode{\sphinxupquote{lot\_id}} with \sphinxstylestrong{NONLOT}. Column is passed with null values and this can impact some processes, since later on both lot and non lot repots are joined togheter.

\item {} 
\sphinxstylestrong{Step 6 (For Non Lot Only):} Populate column \sphinxcode{\sphinxupquote{lot\_qty\_on\_hand}} with same value as \sphinxcode{\sphinxupquote{bu\_qty\_on\_hand}} for the same reasons as \sphinxstylestrong{Step 5}.

\item {} 
\sphinxstylestrong{Step 7 (For Non Lot Only):} Populate column \sphinxcode{\sphinxupquote{expire\_date}} with value that in general terms can be considered infinite for the same reasons as \sphinxstylestrong{Step 5} and \sphinxstylestrong{Step 6}.

\end{itemize}
\begin{quote}\begin{description}
\item[{Parameters}] \leavevmode\begin{itemize}
\item {} 
\sphinxstyleliteralstrong{\sphinxupquote{x\_df}} (\sphinxstyleliteralemphasis{\sphinxupquote{pd.DataFrame}}) \textendash{} Dataframe with lot or non\sphinxhyphen{}lot input input data.

\item {} 
\sphinxstyleliteralstrong{\sphinxupquote{columns\_list}} (\sphinxstyleliteralemphasis{\sphinxupquote{list}}) \textendash{} List of columns used throughout the model.

\item {} 
\sphinxstyleliteralstrong{\sphinxupquote{non\_lot}} (\sphinxstyleliteralemphasis{\sphinxupquote{bool}}\sphinxstyleliteralemphasis{\sphinxupquote{, }}\sphinxstyleliteralemphasis{\sphinxupquote{optional}}) \textendash{} Flag if input data being processed is for lot or non lot inventory. If \sphinxstylestrong{True} then model assumes \sphinxstylestrong{non lot} inventory, else \sphinxstylestrong{lot}, by default True.

\end{itemize}

\item[{Returns}] \leavevmode
Dataframe with lot or non\sphinxhyphen{}lot normalized data.

\item[{Return type}] \leavevmode
pd.DataFrame

\end{description}\end{quote}

\begin{sphinxadmonition}{note}{Note:}
During all the described processes, all changes to input data are either informed for user by raising and error or by adding it to the \sphinxcode{\sphinxupquote{log file}}.
\end{sphinxadmonition}

\end{fulllineitems}

\index{test() (in module optimization.datatools.pipelines)@\spxentry{test()}\spxextra{in module optimization.datatools.pipelines}}

\begin{fulllineitems}
\phantomsection\label{\detokenize{source/optimization.datatools:optimization.datatools.pipelines.test}}\pysiglinewithargsret{\sphinxcode{\sphinxupquote{optimization.datatools.pipelines.}}\sphinxbfcode{\sphinxupquote{test}}}{}{}
\end{fulllineitems}



\subsubsection{Module contents}
\label{\detokenize{source/optimization.datatools:module-optimization.datatools}}\label{\detokenize{source/optimization.datatools:module-contents}}\index{module@\spxentry{module}!optimization.datatools@\spxentry{optimization.datatools}}\index{optimization.datatools@\spxentry{optimization.datatools}!module@\spxentry{module}}
Scripts used to generate necessary columns and perform data cleanning steps.


\paragraph{Submodules}
\label{\detokenize{source/optimization.datatools:submodules}}

\begin{savenotes}\sphinxatlongtablestart\begin{longtable}[c]{\X{1}{2}\X{1}{2}}
\hline

\endfirsthead

\multicolumn{2}{c}%
{\makebox[0pt]{\sphinxtablecontinued{\tablename\ \thetable{} \textendash{} continued from previous page}}}\\
\hline

\endhead

\hline
\multicolumn{2}{r}{\makebox[0pt][r]{\sphinxtablecontinued{continues on next page}}}\\
\endfoot

\endlastfoot

{\hyperref[\detokenize{source/optimization.datatools:module-optimization.datatools.dataprep}]{\sphinxcrossref{\sphinxcode{\sphinxupquote{dataprep}}}}}
&
Functions created for data transformation or validation processes
\\
\hline
{\hyperref[\detokenize{source/optimization.datatools:module-optimization.datatools.pipelines}]{\sphinxcrossref{\sphinxcode{\sphinxupquote{pipelines}}}}}
&
Pipelines used for adjusting input inventory reports to be passed to optimization model.
\\
\hline
{\hyperref[\detokenize{source/optimization.datatools:module-optimization.datatools.extra_output}]{\sphinxcrossref{\sphinxcode{\sphinxupquote{extra\_output}}}}}
&
Extra output scripts for creating output columns to optimization model.
\\
\hline
\end{longtable}\sphinxatlongtableend\end{savenotes}


\subsection{Optimization Tools}
\label{\detokenize{source/optimization.opt_tools:optimization-tools}}\label{\detokenize{source/optimization.opt_tools::doc}}

\subsubsection{Auxiliary Functions}
\label{\detokenize{source/optimization.opt_tools:module-optimization.opt_tools.aux_funcs}}\label{\detokenize{source/optimization.opt_tools:auxiliary-functions}}\index{module@\spxentry{module}!optimization.opt\_tools.aux\_funcs@\spxentry{optimization.opt\_tools.aux\_funcs}}\index{optimization.opt\_tools.aux\_funcs@\spxentry{optimization.opt\_tools.aux\_funcs}!module@\spxentry{module}}
Python module used to store auxiliary functions that are used in the rest of the code.
\index{create\_folder() (in module optimization.opt\_tools.aux\_funcs)@\spxentry{create\_folder()}\spxextra{in module optimization.opt\_tools.aux\_funcs}}

\begin{fulllineitems}
\phantomsection\label{\detokenize{source/optimization.opt_tools:optimization.opt_tools.aux_funcs.create_folder}}\pysiglinewithargsret{\sphinxcode{\sphinxupquote{optimization.opt\_tools.aux\_funcs.}}\sphinxbfcode{\sphinxupquote{create\_folder}}}{\emph{\DUrole{n}{fdir}\DUrole{p}{:} \DUrole{n}{str}}}{{ $\rightarrow$ None}}
Tries to create the directory specified.

If new folder can’t be created, function raises error.
\begin{quote}\begin{description}
\item[{Parameters}] \leavevmode
\sphinxstyleliteralstrong{\sphinxupquote{fdir}} (\sphinxstyleliteralemphasis{\sphinxupquote{str}}) \textendash{} Directory of new folder to be created.

\item[{Raises}] \leavevmode
\sphinxstyleliteralstrong{\sphinxupquote{ValueError}} \textendash{} For some reason folder could not be created.

\end{description}\end{quote}

\end{fulllineitems}

\index{folder\_exists() (in module optimization.opt\_tools.aux\_funcs)@\spxentry{folder\_exists()}\spxextra{in module optimization.opt\_tools.aux\_funcs}}

\begin{fulllineitems}
\phantomsection\label{\detokenize{source/optimization.opt_tools:optimization.opt_tools.aux_funcs.folder_exists}}\pysiglinewithargsret{\sphinxcode{\sphinxupquote{optimization.opt\_tools.aux\_funcs.}}\sphinxbfcode{\sphinxupquote{folder\_exists}}}{\emph{\DUrole{n}{fdir}\DUrole{p}{:} \DUrole{n}{str}}}{{ $\rightarrow$ None}}
If folder does not exist, create one.
\begin{quote}\begin{description}
\item[{Parameters}] \leavevmode
\sphinxstyleliteralstrong{\sphinxupquote{fdir}} (\sphinxstyleliteralemphasis{\sphinxupquote{str}}) \textendash{} Directory to check for existence.

\end{description}\end{quote}

\end{fulllineitems}

\index{get\_num() (in module optimization.opt\_tools.aux\_funcs)@\spxentry{get\_num()}\spxextra{in module optimization.opt\_tools.aux\_funcs}}

\begin{fulllineitems}
\phantomsection\label{\detokenize{source/optimization.opt_tools:optimization.opt_tools.aux_funcs.get_num}}\pysiglinewithargsret{\sphinxcode{\sphinxupquote{optimization.opt\_tools.aux\_funcs.}}\sphinxbfcode{\sphinxupquote{get\_num}}}{\emph{\DUrole{n}{string}\DUrole{p}{:} \DUrole{n}{str}}}{}
Return number from last results .csv version.
\begin{quote}\begin{description}
\item[{Parameters}] \leavevmode
\sphinxstyleliteralstrong{\sphinxupquote{string}} (\sphinxstyleliteralemphasis{\sphinxupquote{str}}) \textendash{} String with the name of the results.csv file.

\item[{Returns}] \leavevmode
Number of last version of results.

\item[{Return type}] \leavevmode
int

\end{description}\end{quote}
\subsubsection*{Examples}
\begin{itemize}
\item {} 
\sphinxcode{\sphinxupquote{Results\_V7.csv}}: 7 (Function captures the number)

\item {} 
\sphinxcode{\sphinxupquote{Results\_V12.csv}}: 12 (Function captures the whole number!)

\end{itemize}

\end{fulllineitems}

\index{last\_result() (in module optimization.opt\_tools.aux\_funcs)@\spxentry{last\_result()}\spxextra{in module optimization.opt\_tools.aux\_funcs}}

\begin{fulllineitems}
\phantomsection\label{\detokenize{source/optimization.opt_tools:optimization.opt_tools.aux_funcs.last_result}}\pysiglinewithargsret{\sphinxcode{\sphinxupquote{optimization.opt\_tools.aux\_funcs.}}\sphinxbfcode{\sphinxupquote{last\_result}}}{\emph{\DUrole{n}{fdir}\DUrole{p}{:} \DUrole{n}{str}}, \emph{\DUrole{n}{fname}\DUrole{p}{:} \DUrole{n}{str}}}{{ $\rightarrow$ int}}
Get last version number from all results file on directory.
\begin{quote}\begin{description}
\item[{Parameters}] \leavevmode
\sphinxstyleliteralstrong{\sphinxupquote{fdir}} (\sphinxstyleliteralemphasis{\sphinxupquote{str}}) \textendash{} Directory where results are being stored.

\item[{Returns}] \leavevmode
Last version number from all results at the base directory.

\item[{Return type}] \leavevmode
int

\end{description}\end{quote}

\end{fulllineitems}

\index{path\_exists() (in module optimization.opt\_tools.aux\_funcs)@\spxentry{path\_exists()}\spxextra{in module optimization.opt\_tools.aux\_funcs}}

\begin{fulllineitems}
\phantomsection\label{\detokenize{source/optimization.opt_tools:optimization.opt_tools.aux_funcs.path_exists}}\pysiglinewithargsret{\sphinxcode{\sphinxupquote{optimization.opt\_tools.aux\_funcs.}}\sphinxbfcode{\sphinxupquote{path\_exists}}}{\emph{\DUrole{n}{fdir}\DUrole{p}{:} \DUrole{n}{str}}, \emph{\DUrole{n}{raise\_error}\DUrole{p}{:} \DUrole{n}{bool} \DUrole{o}{=} \DUrole{default_value}{False}}}{{ $\rightarrow$ bool}}
Check if path exists.
\begin{quote}\begin{description}
\item[{Parameters}] \leavevmode
\sphinxstyleliteralstrong{\sphinxupquote{fdir}} (\sphinxstyleliteralemphasis{\sphinxupquote{str}}) \textendash{} Directory to check for existence.

\item[{Returns}] \leavevmode
Returns True if path exists and nothing if it doesn’t.

\item[{Return type}] \leavevmode
True, None

\end{description}\end{quote}

\end{fulllineitems}

\index{round\_values() (in module optimization.opt\_tools.aux\_funcs)@\spxentry{round\_values()}\spxextra{in module optimization.opt\_tools.aux\_funcs}}

\begin{fulllineitems}
\phantomsection\label{\detokenize{source/optimization.opt_tools:optimization.opt_tools.aux_funcs.round_values}}\pysiglinewithargsret{\sphinxcode{\sphinxupquote{optimization.opt\_tools.aux\_funcs.}}\sphinxbfcode{\sphinxupquote{round\_values}}}{\emph{\DUrole{n}{df}\DUrole{p}{:} \DUrole{n}{pandas.core.frame.DataFrame}}, \emph{\DUrole{n}{column\_name}\DUrole{p}{:} \DUrole{n}{str}}, \emph{\DUrole{n}{decimal\_points}\DUrole{p}{:} \DUrole{n}{int}}}{}
Round a DataFrame to a variable number of decimal places.

Method uses for…loop in order to be able to round values even when
column has non\sphinxhyphen{}numeric values.
\begin{quote}\begin{description}
\item[{Parameters}] \leavevmode\begin{itemize}
\item {} 
\sphinxstyleliteralstrong{\sphinxupquote{df}} (\sphinxstyleliteralemphasis{\sphinxupquote{pd.DataFrame}}) \textendash{} Dataframe object with column to be rounded.

\item {} 
\sphinxstyleliteralstrong{\sphinxupquote{column\_name}} (\sphinxstyleliteralemphasis{\sphinxupquote{str}}) \textendash{} Name of the column with values to be rounded.

\item {} 
\sphinxstyleliteralstrong{\sphinxupquote{decimal\_points}} (\sphinxstyleliteralemphasis{\sphinxupquote{int}}\sphinxstyleliteralemphasis{\sphinxupquote{, }}\sphinxstyleliteralemphasis{\sphinxupquote{dict}}\sphinxstyleliteralemphasis{\sphinxupquote{, }}\sphinxstyleliteralemphasis{\sphinxupquote{Series}}) \textendash{} Number of decimal places to round each column to.
If an int is given, round each column to the same number of places.
Otherwise dict and Series round to variable numbers of places.
Column names should be in the keys if decimals is a dict\sphinxhyphen{}like, or
in the index if decimals is a Series. Any columns not included in decimals
will be left as is. Elements of decimals which are
not columns of the input will be ignored.

\end{itemize}

\item[{Returns}] \leavevmode
\sphinxstylestrong{df} \textendash{} DataFrame object.

\item[{Return type}] \leavevmode
pd.DataFrame

\end{description}\end{quote}

\end{fulllineitems}

\index{save\_results() (in module optimization.opt\_tools.aux\_funcs)@\spxentry{save\_results()}\spxextra{in module optimization.opt\_tools.aux\_funcs}}

\begin{fulllineitems}
\phantomsection\label{\detokenize{source/optimization.opt_tools:optimization.opt_tools.aux_funcs.save_results}}\pysiglinewithargsret{\sphinxcode{\sphinxupquote{optimization.opt\_tools.aux\_funcs.}}\sphinxbfcode{\sphinxupquote{save\_results}}}{\emph{\DUrole{n}{result}}, \emph{\DUrole{n}{fdir}\DUrole{p}{:} \DUrole{n}{str}}, \emph{\DUrole{n}{fname}\DUrole{p}{:} \DUrole{n}{str} \DUrole{o}{=} \DUrole{default_value}{\textquotesingle{}results\textquotesingle{}}}}{{ $\rightarrow$ None}}
Save model results.

Saves the results obtained from the model at the specified directory using the
version control convention estabilshed.
\begin{quote}\begin{description}
\item[{Parameters}] \leavevmode\begin{itemize}
\item {} 
\sphinxstyleliteralstrong{\sphinxupquote{fdir}} (\sphinxstyleliteralemphasis{\sphinxupquote{str}}) \textendash{} Directory that the results will be saved.

\item {} 
\sphinxstyleliteralstrong{\sphinxupquote{result}} (\sphinxstyleliteralemphasis{\sphinxupquote{pandas.core.frame.DataFrame}}) \textendash{} Dataframe with the optimization model’s results.

\item {} 
\sphinxstyleliteralstrong{\sphinxupquote{fname}} (\sphinxstyleliteralemphasis{\sphinxupquote{str Defaults=\textquotesingle{}results\textquotesingle{}}}) \textendash{} Filename in which results should be saved.

\end{itemize}

\item[{Returns}] \leavevmode


\item[{Return type}] \leavevmode
None.

\end{description}\end{quote}

\end{fulllineitems}



\subsubsection{Load Data}
\label{\detokenize{source/optimization.opt_tools:module-optimization.opt_tools.load_data}}\label{\detokenize{source/optimization.opt_tools:load-data}}\index{module@\spxentry{module}!optimization.opt\_tools.load\_data@\spxentry{optimization.opt\_tools.load\_data}}\index{optimization.opt\_tools.load\_data@\spxentry{optimization.opt\_tools.load\_data}!module@\spxentry{module}}
Module with methods for loading inventory reports.

\begin{sphinxadmonition}{note}{Note:}
At this moment, this model supports .csv .xlsx file formats.
\end{sphinxadmonition}

\begin{sphinxadmonition}{warning}{Warning:}
\sphinxtitleref{.xlsx} files should have inventory report as first sheet (otherwise it won’t work).
\end{sphinxadmonition}
\index{check\_extension() (in module optimization.opt\_tools.load\_data)@\spxentry{check\_extension()}\spxextra{in module optimization.opt\_tools.load\_data}}

\begin{fulllineitems}
\phantomsection\label{\detokenize{source/optimization.opt_tools:optimization.opt_tools.load_data.check_extension}}\pysiglinewithargsret{\sphinxcode{\sphinxupquote{optimization.opt\_tools.load\_data.}}\sphinxbfcode{\sphinxupquote{check\_extension}}}{\emph{\DUrole{n}{fext}\DUrole{p}{:} \DUrole{n}{str}}}{{ $\rightarrow$ None}}
Check if file extension is suported.
\begin{quote}\begin{description}
\item[{Parameters}] \leavevmode
\sphinxstyleliteralstrong{\sphinxupquote{(}}\sphinxstyleliteralstrong{\sphinxupquote{str}}\sphinxstyleliteralstrong{\sphinxupquote{)}} (\sphinxstyleliteralemphasis{\sphinxupquote{fext}}) \textendash{} 

\item[{Raises}] \leavevmode
\sphinxstyleliteralstrong{\sphinxupquote{ValueError}} \textendash{} Raises error if extension is not supported.:

\end{description}\end{quote}

\end{fulllineitems}

\index{get\_file\_extension() (in module optimization.opt\_tools.load\_data)@\spxentry{get\_file\_extension()}\spxextra{in module optimization.opt\_tools.load\_data}}

\begin{fulllineitems}
\phantomsection\label{\detokenize{source/optimization.opt_tools:optimization.opt_tools.load_data.get_file_extension}}\pysiglinewithargsret{\sphinxcode{\sphinxupquote{optimization.opt\_tools.load\_data.}}\sphinxbfcode{\sphinxupquote{get\_file\_extension}}}{\emph{\DUrole{n}{fname}\DUrole{p}{:} \DUrole{n}{str}}}{{ $\rightarrow$ str}}
Return file extension.
\begin{quote}\begin{description}
\item[{Parameters}] \leavevmode
\sphinxstyleliteralstrong{\sphinxupquote{(}}\sphinxstyleliteralstrong{\sphinxupquote{str}}\sphinxstyleliteralstrong{\sphinxupquote{)}} (\sphinxstyleliteralemphasis{\sphinxupquote{fname}}) \textendash{} 

\item[{Returns}] \leavevmode


\item[{Return type}] \leavevmode
File extension.

\end{description}\end{quote}

\end{fulllineitems}

\index{load\_csv() (in module optimization.opt\_tools.load\_data)@\spxentry{load\_csv()}\spxextra{in module optimization.opt\_tools.load\_data}}

\begin{fulllineitems}
\phantomsection\label{\detokenize{source/optimization.opt_tools:optimization.opt_tools.load_data.load_csv}}\pysiglinewithargsret{\sphinxcode{\sphinxupquote{optimization.opt\_tools.load\_data.}}\sphinxbfcode{\sphinxupquote{load\_csv}}}{\emph{\DUrole{n}{fdir}\DUrole{p}{:} \DUrole{n}{str}}}{{ $\rightarrow$ pandas.core.frame.DataFrame}}
Loads csv file.
\begin{quote}\begin{description}
\item[{Parameters}] \leavevmode
\sphinxstyleliteralstrong{\sphinxupquote{(}}\sphinxstyleliteralstrong{\sphinxupquote{str}}\sphinxstyleliteralstrong{\sphinxupquote{)}} (\sphinxstyleliteralemphasis{\sphinxupquote{fdir}}) \textendash{} 

\item[{Returns}] \leavevmode
\sphinxstylestrong{pd.DataFrame}

\item[{Return type}] \leavevmode
Pandas dataframe of the loaded csv.

\end{description}\end{quote}

\end{fulllineitems}

\index{load\_data() (in module optimization.opt\_tools.load\_data)@\spxentry{load\_data()}\spxextra{in module optimization.opt\_tools.load\_data}}

\begin{fulllineitems}
\phantomsection\label{\detokenize{source/optimization.opt_tools:optimization.opt_tools.load_data.load_data}}\pysiglinewithargsret{\sphinxcode{\sphinxupquote{optimization.opt\_tools.load\_data.}}\sphinxbfcode{\sphinxupquote{load\_data}}}{\emph{\DUrole{n}{fdir}\DUrole{p}{:} \DUrole{n}{str}}}{{ $\rightarrow$ pandas.core.frame.DataFrame}}
Load dataframe.

Method calls function that determines the file extension,
then it calls the appropriate method to load the file.

\begin{sphinxadmonition}{note}{Note:}
Supported file formats should be added at module constants list named
\sphinxtitleref{supported\_file\_extensions}

\sphinxstylestrong{Methods}
\begin{itemize}
\item {} 
get\_file\_extension(fdir)

\item {} 
load\_csv(fdir)

\item {} 
load\_excel(fdir)

\end{itemize}
\end{sphinxadmonition}
\begin{quote}\begin{description}
\item[{Parameters}] \leavevmode
\sphinxstyleliteralstrong{\sphinxupquote{fdir}} (\sphinxstyleliteralemphasis{\sphinxupquote{str}}) \textendash{} Filepath of the dataframe.

\item[{Returns}] \leavevmode
\sphinxstylestrong{x\_df} \textendash{} Dataframe loaded.

\item[{Return type}] \leavevmode
pd.DataFrame

\end{description}\end{quote}

\end{fulllineitems}

\index{load\_excel() (in module optimization.opt\_tools.load\_data)@\spxentry{load\_excel()}\spxextra{in module optimization.opt\_tools.load\_data}}

\begin{fulllineitems}
\phantomsection\label{\detokenize{source/optimization.opt_tools:optimization.opt_tools.load_data.load_excel}}\pysiglinewithargsret{\sphinxcode{\sphinxupquote{optimization.opt\_tools.load\_data.}}\sphinxbfcode{\sphinxupquote{load\_excel}}}{\emph{\DUrole{n}{fdir}\DUrole{p}{:} \DUrole{n}{str}}}{{ $\rightarrow$ pandas.core.frame.DataFrame}}
Loads excel file into pandas DataFrame.

This method is an extension of load\_data.
After load\_data determines the file extension,
and if the extension is of type .xlsx, load\_data
calls this method to load the excel file.
\begin{quote}\begin{description}
\item[{Parameters}] \leavevmode
\sphinxstyleliteralstrong{\sphinxupquote{(}}\sphinxstyleliteralstrong{\sphinxupquote{str}}\sphinxstyleliteralstrong{\sphinxupquote{)}} (\sphinxstyleliteralemphasis{\sphinxupquote{fdir}}) \textendash{} 

\item[{Returns}] \leavevmode
\sphinxstylestrong{pd.DataFrame}

\item[{Return type}] \leavevmode
Dataframe loaded.

\end{description}\end{quote}

\end{fulllineitems}



\subsubsection{Module Contents}
\label{\detokenize{source/optimization.opt_tools:module-optimization.opt_tools}}\label{\detokenize{source/optimization.opt_tools:module-contents}}\index{module@\spxentry{module}!optimization.opt\_tools@\spxentry{optimization.opt\_tools}}\index{optimization.opt\_tools@\spxentry{optimization.opt\_tools}!module@\spxentry{module}}
Auxiliary functions created to assist on generating the optimization results.


\paragraph{Submodules}
\label{\detokenize{source/optimization.opt_tools:submodules}}

\begin{savenotes}\sphinxatlongtablestart\begin{longtable}[c]{\X{1}{2}\X{1}{2}}
\hline

\endfirsthead

\multicolumn{2}{c}%
{\makebox[0pt]{\sphinxtablecontinued{\tablename\ \thetable{} \textendash{} continued from previous page}}}\\
\hline

\endhead

\hline
\multicolumn{2}{r}{\makebox[0pt][r]{\sphinxtablecontinued{continues on next page}}}\\
\endfoot

\endlastfoot

{\hyperref[\detokenize{source/optimization.opt_tools:module-optimization.opt_tools.aux_funcs}]{\sphinxcrossref{\sphinxcode{\sphinxupquote{aux\_funcs}}}}}
&
Python module used to store auxiliary functions that are used in the rest of the code.
\\
\hline
{\hyperref[\detokenize{source/optimization.opt_tools:module-optimization.opt_tools.load_data}]{\sphinxcrossref{\sphinxcode{\sphinxupquote{load\_data}}}}}
&
Module with methods for loading inventory reports.
\\
\hline
\end{longtable}\sphinxatlongtableend\end{savenotes}


\subsection{Model Configurations}
\label{\detokenize{source/optimization.configuration:model-configurations}}\label{\detokenize{source/optimization.configuration::doc}}

\subsubsection{Basic Configurations}
\label{\detokenize{source/optimization.configuration:basic-configurations}}
The optimization model configurations are stored at {\hyperref[\detokenize{source/optimization:module-optimization.constants}]{\sphinxcrossref{\sphinxcode{\sphinxupquote{optimization.constants}}}}}.
Bellow, you can find all available configuration options and their respective description.
\begin{description}
\item[{LOG\_MODE\index{LOG\_MODE@\spxentry{LOG\_MODE}|spxpagem}\phantomsection\label{\detokenize{source/optimization.configuration:term-LOG_MODE}}}] \leavevmode
Can be set to \sphinxcode{\sphinxupquote{True}} or \sphinxcode{\sphinxupquote{False}}. When enabled the model will store all \sphinxstylestrong{logs}  into \sphinxcode{\sphinxupquote{./optimization/logs/optimizer.log}}.
This configuration is set to \sphinxcode{\sphinxupquote{True}} by default.

\item[{SAVE\_MODEL\index{SAVE\_MODEL@\spxentry{SAVE\_MODEL}|spxpagem}\phantomsection\label{\detokenize{source/optimization.configuration:term-SAVE_MODEL}}}] \leavevmode
Can be set to \sphinxcode{\sphinxupquote{True}} or \sphinxcode{\sphinxupquote{False}}. If enabled will save the model results into the folder \sphinxcode{\sphinxupquote{./optimization/Results/}} as a \sphinxcode{\sphinxupquote{.csv}} file. Since the model is set to run on \sphinxstylestrong{Alteryx}, this configuration is set to \sphinxcode{\sphinxupquote{False}} by default.

\item[{USE\_DYNAMIC\_TIME\index{USE\_DYNAMIC\_TIME@\spxentry{USE\_DYNAMIC\_TIME}|spxpagem}\phantomsection\label{\detokenize{source/optimization.configuration:term-USE_DYNAMIC_TIME}}}] \leavevmode
It can be set to \sphinxcode{\sphinxupquote{True}} or \sphinxcode{\sphinxupquote{False}}. If enabled will determine the maximum amount of time the model can spend trying
to optimize a single \sphinxcode{\sphinxupquote{Item ID}} dynamically, depending on the \sphinxstylestrong{total dollar value corresponding to that item}.
This configuration is set to \sphinxcode{\sphinxupquote{False}} by default.

\item[{MAX\_TIME\index{MAX\_TIME@\spxentry{MAX\_TIME}|spxpagem}\phantomsection\label{\detokenize{source/optimization.configuration:term-MAX_TIME}}}] \leavevmode
It can be set to any \sphinxstylestrong{positive} \sphinxcode{\sphinxupquote{integer}}. When \sphinxcode{\sphinxupquote{USE\_DYNAMIC\_TIME}} is set to \sphinxcode{\sphinxupquote{True}}, this parameter is then used as upper boundary
to determine the maximum amount of time the model will spend trying to optimize a single item. \sphinxstylestrong{By default is set to 500 and the value represents time in seconds.}

\item[{DEFAULT\_TIME\index{DEFAULT\_TIME@\spxentry{DEFAULT\_TIME}|spxpagem}\phantomsection\label{\detokenize{source/optimization.configuration:term-DEFAULT_TIME}}}] \leavevmode
It can be set to any \sphinxstylestrong{positive} \sphinxcode{\sphinxupquote{integer}}. Default maximum amount of seconda the model can spend trying to find the optimal solution for a single \sphinxcode{\sphinxupquote{Item ID}}.

\item[{LOT\_DF\index{LOT\_DF@\spxentry{LOT\_DF}|spxpagem}\phantomsection\label{\detokenize{source/optimization.configuration:term-LOT_DF}}}] \leavevmode
Variable used to store inventory reports comming from \sphinxstylestrong{Alteryx}.
This variable is then used at {\hyperref[\detokenize{source/optimization.datatools:optimization.datatools.pipelines.data_pipeline}]{\sphinxcrossref{\sphinxcode{\sphinxupquote{optimization.datatools.pipelines.data\_pipeline}}}}}
in conjunction with \sphinxcode{\sphinxupquote{NONLOT\_DF}} to get the final input that will be used by the model.

\item[{NONLOT\_DF\index{NONLOT\_DF@\spxentry{NONLOT\_DF}|spxpagem}\phantomsection\label{\detokenize{source/optimization.configuration:term-NONLOT_DF}}}] \leavevmode
Same as \sphinxcode{\sphinxupquote{LOT\_DF}}, but for \sphinxstylestrong{Non Lot} items.

\end{description}


\subsubsection{Columns Used by the Model}
\label{\detokenize{source/optimization.configuration:columns-used-by-the-model}}
The columns below are the ones being used as input by the model. Some of these columns are used on the final output of the model,
and others are necessary to calculate attributes used by the model. These columns are referenced by a \sphinxcode{\sphinxupquote{list}} that is stored inside the method {\hyperref[\detokenize{source/optimization.datatools:optimization.datatools.pipelines.data_pipeline}]{\sphinxcrossref{\sphinxcode{\sphinxupquote{optimization.datatools.pipelines.data\_pipeline}}}}}.
If the user wants to pass new columns to the model, add their respective names at the list inside {\hyperref[\detokenize{source/optimization.datatools:optimization.datatools.pipelines.data_pipeline}]{\sphinxcrossref{\sphinxcode{\sphinxupquote{optimization.datatools.pipelines.data\_pipeline}}}}}.


\begin{savenotes}\sphinxattablestart
\centering
\begin{tabular}[t]{|\X{50}{160}|\X{30}{160}|\X{80}{160}|}
\hline
\sphinxstyletheadfamily 
Column Name
&\sphinxstyletheadfamily 
Used By
&\sphinxstyletheadfamily 
Description
\\
\hline
Inv BU
&
\sphinxcode{\sphinxupquote{Model}}
&
Unique identification number assigned to every business unit from Quest Diagnostics.
\\
\hline
Item ID
&
\sphinxcode{\sphinxupquote{Model}}
&
Unique identification number assigned to every SKU from Quest Diagnostics.
\\
\hline
Lot ID
&
\sphinxcode{\sphinxupquote{Model}}
&
Identification number assigned to every Lot SKU from Quest Diagnostics. This column is only used for Lot items. \sphinxstylestrong{Lot items are items that have expiration date}.
\\
\hline
Report Date (Query Date)
&
\sphinxcode{\sphinxupquote{Model}}
&
Date that the inventory report was generated. This column is used togheter with the column \sphinxcode{\sphinxupquote{Expire Date}} to calculate the \sphinxcode{\sphinxupquote{Days to Expire}} column. This last column is then used to determine items that might get expired before their consumption.
\\
\hline
Expire Date
&
\sphinxcode{\sphinxupquote{Model}}
&
Expiration date of Lot Items.
\\
\hline
Average Item Daily Use
&
\sphinxcode{\sphinxupquote{Model}}
&
Average daily consumption rate of a given item at a given business unit. This column is used in order to calculate \sphinxcode{\sphinxupquote{Inventory Balance}} and \sphinxcode{\sphinxupquote{Items to Expire}} columns used that are the main columns used to define the optimization problem \sphinxstylestrong{objective function}.
\\
\hline
BU Qty on Hand
&
\sphinxcode{\sphinxupquote{Model}}
&
This column shows the total number of items from a particular \sphinxcode{\sphinxupquote{Item ID}} all a single BU have. When combined with the column \sphinxcode{\sphinxupquote{DOI Target}} and \sphinxcode{\sphinxupquote{Average Item Daily Use}} we obtain the \sphinxstylestrong{inventory balance} that the given BU has for a given item.
\\
\hline
Lot Qty on Hand
&
\sphinxcode{\sphinxupquote{Model}}
&\\
\hline
DOI Target
&
\sphinxcode{\sphinxupquote{Model}}
&\\
\hline
Can Receive Inventory
&
\sphinxcode{\sphinxupquote{Model}}
&\\
\hline
Can Transfer Inventory
&
\sphinxcode{\sphinxupquote{Model}}
&\\
\hline
Min Shipment  Value
&
\sphinxcode{\sphinxupquote{Model}}
&\\
\hline
Price
&
\sphinxcode{\sphinxupquote{Model}}
&\\
\hline
Default Shipment Days
&
\sphinxcode{\sphinxupquote{Model}}
&\\
\hline
Item Stats
&
\sphinxcode{\sphinxupquote{Model}}
&\\
\hline
BUv Item Qty in Transf
&
\sphinxcode{\sphinxupquote{Model}}
&\\
\hline
Item Description
&
\sphinxcode{\sphinxupquote{Output}}
&\\
\hline
BU Region
&
\sphinxcode{\sphinxupquote{Output}}
&\\
\hline
Contact Email
&
\sphinxcode{\sphinxupquote{Output}}
&\\
\hline
Supplier Name
&
\sphinxcode{\sphinxupquote{Output}}
&\\
\hline
On Site Email
&
\sphinxcode{\sphinxupquote{Output}}
&\\
\hline
STD UOM
&
\sphinxcode{\sphinxupquote{Output}}
&\\
\hline
Chart of Accounts
&
\sphinxcode{\sphinxupquote{Output}}
&\\
\hline
BU Address
&
\sphinxcode{\sphinxupquote{Output}}
&\\
\hline
BU Descrip
&
\sphinxcode{\sphinxupquote{Output}}
&\\
\hline
\end{tabular}
\par
\sphinxattableend\end{savenotes}


\subsection{Solution Space}
\label{\detokenize{source/optimization:module-optimization.solspace}}\label{\detokenize{source/optimization:solution-space}}\index{module@\spxentry{module}!optimization.solspace@\spxentry{optimization.solspace}}\index{optimization.solspace@\spxentry{optimization.solspace}!module@\spxentry{module}}
Scripts used for generating the \sphinxstylestrong{solution space.}

Python module used to generate the solution space that is then used by \sphinxcode{\sphinxupquote{optimizer}} module to generate the
optimization transfer recommendations.

In simple terms this module contains one class named {\hyperref[\detokenize{source/optimization:optimization.solspace.SolutionSpace}]{\sphinxcrossref{\sphinxcode{\sphinxupquote{optimization.solspace.SolutionSpace}}}}} that is used to:
\begin{enumerate}
\sphinxsetlistlabels{\arabic}{enumi}{enumii}{(}{)}%
\item {} 
Filter inventory dataframe for the SKU that we’re analyzing;

\item {} 
Make one aggregation at a BU granularity for determining providing BU’s and another at Lot ID granularity for determining receiving BU’s/Lot ID’s

\item {} 
Query both grouping results, filtering for BU’s that don’t have items to expire (those will be the receiver BU’s) and for those BU’s that have either items to expire or surplus inventory consider them as providers.

\item {} 
Create the solution space matrix adding column\sphinxhyphen{}wise, receiving BU’s and row\sphinxhyphen{}wise providing  BU’s/Lot ID’s’.

\item {} 
Add extra columns to the solution matrix that will be used on the optimization model constraints or objective function definitions.

\end{enumerate}
\index{SolutionSpace (class in optimization.solspace)@\spxentry{SolutionSpace}\spxextra{class in optimization.solspace}}

\begin{fulllineitems}
\phantomsection\label{\detokenize{source/optimization:optimization.solspace.SolutionSpace}}\pysiglinewithargsret{\sphinxbfcode{\sphinxupquote{class }}\sphinxcode{\sphinxupquote{optimization.solspace.}}\sphinxbfcode{\sphinxupquote{SolutionSpace}}}{\emph{\DUrole{n}{df\_inventory}}, \emph{\DUrole{n}{item\_id}}}{}
Bases: \sphinxcode{\sphinxupquote{object}}

Creates a basic solution space for the given inventory.

This class uses the module constants \sphinxcode{\sphinxupquote{PROVIDING\_COLUMNS}} and \sphinxcode{\sphinxupquote{RECEIVING\_COLUMNS}}
for mapping additional values to the solution space matrix rows and columns
that might be used by the optimization model in the next step.

\sphinxstylestrong{Receivers}
\begin{itemize}
\item {} 
\sphinxstyleemphasis{Granularity:} \sphinxcode{\sphinxupquote{BU ID}}

\item {} 
\sphinxstyleemphasis{Conditions}
\begin{quote}

BU can’t have items to expire. It might have surplus though.
Adding BU’s with surplus at the receiving list should
not affect the model, because of its elastic constraints. Transfering
items to BU’s that already have surplus (essentially making that BU surplus even higher)
results in very high penalties to the objective function, therefore the model will only
do so in extreme situations.
\end{quote}

\end{itemize}

\sphinxstylestrong{Providers}
\begin{itemize}
\item {} 
\sphinxstyleemphasis{Granularity:} \sphinxcode{\sphinxupquote{Lot ID}}

\item {} 
\sphinxstyleemphasis{Conditions}
\begin{quote}

Needs to have items to expire (Lot\sphinxhyphen{}wise) or the BU has surplus.
The total monetary value of possible items that can be transfered
needs to be greater than minimum shipment value.
\end{quote}

\end{itemize}
\begin{quote}\begin{description}
\item[{Parameters}] \leavevmode\begin{itemize}
\item {} 
\sphinxstyleliteralstrong{\sphinxupquote{EXTRA\_COLUMNS}} (\sphinxstyleliteralemphasis{\sphinxupquote{int}}) \textendash{} Quantity of extra columns that store information about \sphinxcode{\sphinxupquote{Providing BU\textquotesingle{}s}} that are either used to define the \sphinxstylestrong{optimization problem}
or are used in the \sphinxstylestrong{output file.}

\item {} 
\sphinxstyleliteralstrong{\sphinxupquote{EXTRA\_ROWS}} (\sphinxstyleliteralemphasis{\sphinxupquote{int}}) \textendash{} Quantity of extra columns that store information about \sphinxcode{\sphinxupquote{Receiving BU\textquotesingle{}s}} that are either used to define the \sphinxstylestrong{optimization problem}
or are used in the \sphinxstylestrong{output file.}

\end{itemize}

\item[{Returns}] \leavevmode
Solution space matrix in the form of a dataframe.

\item[{Return type}] \leavevmode
pd.DataFrame

\end{description}\end{quote}

\begin{sphinxadmonition}{note}{Note:}
For minimum shipment value we’re considering total monetary value of the Lot
and not the BU. Since items from different Lots but from from the same BU can
be transfered, we theoretically should perform this filter using BU total values.
We’re not doing so at the moment for 2 reasons: first there aren’t so many items
with price bellow \$50.00 and even if the model filters out those values the total
amount that we’re not considering in the optimization is very small in comparisson
to the rest of the SKUs.
\end{sphinxadmonition}


\sphinxstrong{See also:}


\sphinxhref{http://coin-or.github.io/pulp/guides/how\_to\_elastic\_constraints.html}{coin\sphinxhyphen{}or.github.io/pulp/guides/how\_to\_elastic\_constraints.html}

\sphinxhref{http://royalsocietypublishing.org/doi/10.1098/rsta.2007.2122}{royalsocietypublishing.org/doi/10.1098/rsta.2007.2122}

\sphinxhref{http://en.wikipedia.org/wiki/Constrained\_optimization}{en.wikipedia.org/wiki/Constrained\_optimization}


\index{EXTRA\_COLUMNS (optimization.solspace.SolutionSpace attribute)@\spxentry{EXTRA\_COLUMNS}\spxextra{optimization.solspace.SolutionSpace attribute}}

\begin{fulllineitems}
\phantomsection\label{\detokenize{source/optimization:optimization.solspace.SolutionSpace.EXTRA_COLUMNS}}\pysigline{\sphinxbfcode{\sphinxupquote{EXTRA\_COLUMNS}}\sphinxbfcode{\sphinxupquote{ = 30}}}
\end{fulllineitems}

\index{EXTRA\_ROWS (optimization.solspace.SolutionSpace attribute)@\spxentry{EXTRA\_ROWS}\spxextra{optimization.solspace.SolutionSpace attribute}}

\begin{fulllineitems}
\phantomsection\label{\detokenize{source/optimization:optimization.solspace.SolutionSpace.EXTRA_ROWS}}\pysigline{\sphinxbfcode{\sphinxupquote{EXTRA\_ROWS}}\sphinxbfcode{\sphinxupquote{ = 25}}}
\end{fulllineitems}

\index{PROVIDING\_AGG (optimization.solspace.SolutionSpace attribute)@\spxentry{PROVIDING\_AGG}\spxextra{optimization.solspace.SolutionSpace attribute}}

\begin{fulllineitems}
\phantomsection\label{\detokenize{source/optimization:optimization.solspace.SolutionSpace.PROVIDING_AGG}}\pysigline{\sphinxbfcode{\sphinxupquote{PROVIDING\_AGG}}\sphinxbfcode{\sphinxupquote{ = \{\textquotesingle{}approver\_contact\_email\textquotesingle{}: \textquotesingle{}first\textquotesingle{}, \textquotesingle{}average\_item\_daily\_use\textquotesingle{}: \textquotesingle{}mean\textquotesingle{}, \textquotesingle{}bu\_address\textquotesingle{}: \textquotesingle{}first\textquotesingle{}, \textquotesingle{}bu\_descrip\textquotesingle{}: \textquotesingle{}first\textquotesingle{}, \textquotesingle{}bu\_doi\_balance\textquotesingle{}: \textquotesingle{}mean\textquotesingle{}, \textquotesingle{}bu\_item\_qty\_in\_transfer\textquotesingle{}: \textquotesingle{}mean\textquotesingle{}, \textquotesingle{}bu\_item\_status\textquotesingle{}: \textquotesingle{}first\textquotesingle{}, \textquotesingle{}bu\_oh\_days\_supply\textquotesingle{}: \textquotesingle{}mean\textquotesingle{}, \textquotesingle{}bu\_oh\_transit\_days\_supply\textquotesingle{}: \textquotesingle{}mean\textquotesingle{}, \textquotesingle{}bu\_qty\_on\_hand\textquotesingle{}: \textquotesingle{}mean\textquotesingle{}, \textquotesingle{}bu\_region\textquotesingle{}: \textquotesingle{}first\textquotesingle{}, \textquotesingle{}can\_receive\_inventory\textquotesingle{}: \textquotesingle{}min\textquotesingle{}, \textquotesingle{}can\_transfer\_inventory\textquotesingle{}: \textquotesingle{}min\textquotesingle{}, \textquotesingle{}chart\_of\_accounts\textquotesingle{}: \textquotesingle{}first\textquotesingle{}, \textquotesingle{}days\_to\_expire\textquotesingle{}: \textquotesingle{}mean\textquotesingle{}, \textquotesingle{}default\_shipment\_days\textquotesingle{}: \textquotesingle{}mean\textquotesingle{}, \textquotesingle{}delta\_doi\textquotesingle{}: \textquotesingle{}mean\textquotesingle{}, \textquotesingle{}doi\_balance\textquotesingle{}: \textquotesingle{}sum\textquotesingle{}, \textquotesingle{}doi\_target\textquotesingle{}: \textquotesingle{}mean\textquotesingle{}, \textquotesingle{}expire\_date\textquotesingle{}: \textquotesingle{}first\textquotesingle{}, \textquotesingle{}item\_descrip\textquotesingle{}: \textquotesingle{}first\textquotesingle{}, \textquotesingle{}items\_to\_expire\textquotesingle{}: \textquotesingle{}sum\textquotesingle{}, \textquotesingle{}lot\_qty\_on\_hand\textquotesingle{}: \textquotesingle{}sum\textquotesingle{}, \textquotesingle{}min\_shipment\_value\textquotesingle{}: \textquotesingle{}mean\textquotesingle{}, \textquotesingle{}onsite\_contact\_email\textquotesingle{}: \textquotesingle{}first\textquotesingle{}, \textquotesingle{}price\textquotesingle{}: \textquotesingle{}mean\textquotesingle{}, \textquotesingle{}std\_uom\textquotesingle{}: \textquotesingle{}first\textquotesingle{}, \textquotesingle{}supplier\_name\textquotesingle{}: \textquotesingle{}first\textquotesingle{}\}}}}
\end{fulllineitems}

\index{PROVIDING\_GROUPBY (optimization.solspace.SolutionSpace attribute)@\spxentry{PROVIDING\_GROUPBY}\spxextra{optimization.solspace.SolutionSpace attribute}}

\begin{fulllineitems}
\phantomsection\label{\detokenize{source/optimization:optimization.solspace.SolutionSpace.PROVIDING_GROUPBY}}\pysigline{\sphinxbfcode{\sphinxupquote{PROVIDING\_GROUPBY}}\sphinxbfcode{\sphinxupquote{ = {[}\textquotesingle{}inv\_bu\textquotesingle{}, \textquotesingle{}lot\_id\textquotesingle{}{]}}}}
\end{fulllineitems}

\index{RECEIVING\_AGG (optimization.solspace.SolutionSpace attribute)@\spxentry{RECEIVING\_AGG}\spxextra{optimization.solspace.SolutionSpace attribute}}

\begin{fulllineitems}
\phantomsection\label{\detokenize{source/optimization:optimization.solspace.SolutionSpace.RECEIVING_AGG}}\pysigline{\sphinxbfcode{\sphinxupquote{RECEIVING\_AGG}}\sphinxbfcode{\sphinxupquote{ = \{\textquotesingle{}approver\_contact\_email\textquotesingle{}: \textquotesingle{}first\textquotesingle{}, \textquotesingle{}average\_item\_daily\_use\textquotesingle{}: \textquotesingle{}mean\textquotesingle{}, \textquotesingle{}bu\_address\textquotesingle{}: \textquotesingle{}first\textquotesingle{}, \textquotesingle{}bu\_descrip\textquotesingle{}: \textquotesingle{}first\textquotesingle{}, \textquotesingle{}bu\_doi\_balance\textquotesingle{}: \textquotesingle{}mean\textquotesingle{}, \textquotesingle{}bu\_item\_qty\_in\_transfer\textquotesingle{}: \textquotesingle{}mean\textquotesingle{}, \textquotesingle{}bu\_item\_status\textquotesingle{}: \textquotesingle{}first\textquotesingle{}, \textquotesingle{}bu\_oh\_days\_supply\textquotesingle{}: \textquotesingle{}mean\textquotesingle{}, \textquotesingle{}bu\_oh\_transit\_days\_supply\textquotesingle{}: \textquotesingle{}mean\textquotesingle{}, \textquotesingle{}bu\_qty\_on\_hand\textquotesingle{}: \textquotesingle{}mean\textquotesingle{}, \textquotesingle{}bu\_region\textquotesingle{}: \textquotesingle{}first\textquotesingle{}, \textquotesingle{}can\_receive\_inventory\textquotesingle{}: \textquotesingle{}min\textquotesingle{}, \textquotesingle{}can\_transfer\_inventory\textquotesingle{}: \textquotesingle{}min\textquotesingle{}, \textquotesingle{}chart\_of\_accounts\textquotesingle{}: \textquotesingle{}first\textquotesingle{}, \textquotesingle{}default\_shipment\_days\textquotesingle{}: \textquotesingle{}mean\textquotesingle{}, \textquotesingle{}delta\_doi\textquotesingle{}: \textquotesingle{}mean\textquotesingle{}, \textquotesingle{}doi\_target\textquotesingle{}: \textquotesingle{}mean\textquotesingle{}, \textquotesingle{}item\_descrip\textquotesingle{}: \textquotesingle{}first\textquotesingle{}, \textquotesingle{}items\_to\_expire\textquotesingle{}: \textquotesingle{}sum\textquotesingle{}, \textquotesingle{}min\_shipment\_value\textquotesingle{}: \textquotesingle{}mean\textquotesingle{}, \textquotesingle{}onsite\_contact\_email\textquotesingle{}: \textquotesingle{}first\textquotesingle{}, \textquotesingle{}price\textquotesingle{}: \textquotesingle{}mean\textquotesingle{}, \textquotesingle{}std\_uom\textquotesingle{}: \textquotesingle{}first\textquotesingle{}, \textquotesingle{}supplier\_name\textquotesingle{}: \textquotesingle{}first\textquotesingle{}\}}}}
\end{fulllineitems}

\index{RECEIVING\_GROUPBY (optimization.solspace.SolutionSpace attribute)@\spxentry{RECEIVING\_GROUPBY}\spxextra{optimization.solspace.SolutionSpace attribute}}

\begin{fulllineitems}
\phantomsection\label{\detokenize{source/optimization:optimization.solspace.SolutionSpace.RECEIVING_GROUPBY}}\pysigline{\sphinxbfcode{\sphinxupquote{RECEIVING\_GROUPBY}}\sphinxbfcode{\sphinxupquote{ = {[}\textquotesingle{}inv\_bu\textquotesingle{}{]}}}}
\end{fulllineitems}

\index{\_create\_matrix() (optimization.solspace.SolutionSpace method)@\spxentry{\_create\_matrix()}\spxextra{optimization.solspace.SolutionSpace method}}

\begin{fulllineitems}
\phantomsection\label{\detokenize{source/optimization:optimization.solspace.SolutionSpace._create_matrix}}\pysiglinewithargsret{\sphinxbfcode{\sphinxupquote{\_create\_matrix}}}{\emph{\DUrole{n}{providing\_list}}, \emph{\DUrole{n}{receiving\_list}}}{}
This method is used to create the solution space matrix.
\begin{quote}\begin{description}
\item[{Parameters}] \leavevmode\begin{itemize}
\item {} 
\sphinxstyleliteralstrong{\sphinxupquote{providing\_list}} (\sphinxstyleliteralemphasis{\sphinxupquote{pd.DataFrame}}) \textendash{} List with providing BU ID’s.

\item {} 
\sphinxstyleliteralstrong{\sphinxupquote{receiving\_list}} (\sphinxstyleliteralemphasis{\sphinxupquote{pd.DataFrame}}) \textendash{} List with receiving BU and Lot ID’s.

\end{itemize}

\item[{Returns}] \leavevmode
\sphinxstylestrong{sol\_space} \textendash{} Numpy matrix to be used in the optimization model.

\item[{Return type}] \leavevmode
numpy.matrix

\end{description}\end{quote}

\end{fulllineitems}

\index{\_get\_bu\_list() (optimization.solspace.SolutionSpace method)@\spxentry{\_get\_bu\_list()}\spxextra{optimization.solspace.SolutionSpace method}}

\begin{fulllineitems}
\phantomsection\label{\detokenize{source/optimization:optimization.solspace.SolutionSpace._get_bu_list}}\pysiglinewithargsret{\sphinxbfcode{\sphinxupquote{\_get\_bu\_list}}}{\emph{\DUrole{n}{group\_by\_list}}, \emph{\DUrole{n}{agg\_dict}}}{}
Filter for BU’s that have the item that we’re trying to optimize and
groups values according to the specified \sphinxcode{\sphinxupquote{group\_by\_list}} and \sphinxcode{\sphinxupquote{agg\_dict}}.
\begin{quote}\begin{description}
\item[{Parameters}] \leavevmode\begin{itemize}
\item {} 
\sphinxstyleliteralstrong{\sphinxupquote{group\_by\_list}} (\sphinxstyleliteralemphasis{\sphinxupquote{list}}) \textendash{} List of columns to be used at the groupby.

\item {} 
\sphinxstyleliteralstrong{\sphinxupquote{agg\_dict}} (\sphinxstyleliteralemphasis{\sphinxupquote{dict}}) \textendash{} Dictionary with columns and aggregation type to be used at the groupby.

\end{itemize}

\item[{Returns}] \leavevmode
\sphinxstylestrong{pd.DataFrame} \textendash{} DataFrame with filtered and grouped values.

\item[{Return type}] \leavevmode
pd.DataFrame

\end{description}\end{quote}

\end{fulllineitems}

\index{\_get\_providing() (optimization.solspace.SolutionSpace method)@\spxentry{\_get\_providing()}\spxextra{optimization.solspace.SolutionSpace method}}

\begin{fulllineitems}
\phantomsection\label{\detokenize{source/optimization:optimization.solspace.SolutionSpace._get_providing}}\pysiglinewithargsret{\sphinxbfcode{\sphinxupquote{\_get\_providing}}}{\emph{\DUrole{n}{sku\_prov}\DUrole{p}{:} \DUrole{n}{pandas.core.frame.DataFrame}}}{}
Get list of providing BU’s by Lot.
\begin{quote}\begin{description}
\item[{Parameters}] \leavevmode
\sphinxstyleliteralstrong{\sphinxupquote{sku\_prov}} (\sphinxstyleliteralemphasis{\sphinxupquote{pd.DataFrame}}) \textendash{} Pandas dataframe with possible providing BU’s.

\item[{Returns}] \leavevmode
\sphinxstylestrong{pd.DataFrame} \textendash{} BU’s to be considered as providers.

\item[{Return type}] \leavevmode
pd.DataFrame

\end{description}\end{quote}

\end{fulllineitems}

\index{\_get\_receiving() (optimization.solspace.SolutionSpace method)@\spxentry{\_get\_receiving()}\spxextra{optimization.solspace.SolutionSpace method}}

\begin{fulllineitems}
\phantomsection\label{\detokenize{source/optimization:optimization.solspace.SolutionSpace._get_receiving}}\pysiglinewithargsret{\sphinxbfcode{\sphinxupquote{\_get\_receiving}}}{\emph{\DUrole{n}{sku\_rec}\DUrole{p}{:} \DUrole{n}{pandas.core.frame.DataFrame}}}{}
Get list of receiving BU’s.
\begin{quote}\begin{description}
\item[{Parameters}] \leavevmode
\sphinxstyleliteralstrong{\sphinxupquote{sku\_rec}} (\sphinxstyleliteralemphasis{\sphinxupquote{pd.DataFrame}}) \textendash{} Pandas dataframe with possible receiving BU’s.

\item[{Returns}] \leavevmode
\sphinxstylestrong{pd.DataFrame} \textendash{} BU’s to be considered as receivers.

\item[{Return type}] \leavevmode
pd.DataFrame

\end{description}\end{quote}

\end{fulllineitems}

\index{\_populate\_matrix() (optimization.solspace.SolutionSpace method)@\spxentry{\_populate\_matrix()}\spxextra{optimization.solspace.SolutionSpace method}}

\begin{fulllineitems}
\phantomsection\label{\detokenize{source/optimization:optimization.solspace.SolutionSpace._populate_matrix}}\pysiglinewithargsret{\sphinxbfcode{\sphinxupquote{\_populate\_matrix}}}{\emph{\DUrole{n}{sol\_space}\DUrole{p}{:} \DUrole{n}{numpy.matrix}}, \emph{\DUrole{n}{bu\_prov}\DUrole{p}{:} \DUrole{n}{pandas.core.frame.DataFrame}}, \emph{\DUrole{n}{bu\_rec}\DUrole{p}{:} \DUrole{n}{pandas.core.frame.DataFrame}}}{}
Populate the sol\_space matrix with the given data for provider and receiver BU’s.
\begin{quote}\begin{description}
\item[{Parameters}] \leavevmode\begin{itemize}
\item {} 
\sphinxstyleliteralstrong{\sphinxupquote{sol\_space}} (\sphinxstyleliteralemphasis{\sphinxupquote{np.matrix}}) \textendash{} Final solution space to be used in the next step
by the optimization model

\item {} 
\sphinxstyleliteralstrong{\sphinxupquote{bu\_prov}} (\sphinxstyleliteralemphasis{\sphinxupquote{pd.DataFrame}}) \textendash{} BU’s to be considered as providers.

\item {} 
\sphinxstyleliteralstrong{\sphinxupquote{bu\_rec}} (\sphinxstyleliteralemphasis{\sphinxupquote{pd.DataFrame}}) \textendash{} BU’s to be considered as receivers.

\end{itemize}

\item[{Returns}] \leavevmode
\sphinxstylestrong{pd.DataFrame} \textendash{} Solution space with added extra columns.

\item[{Return type}] \leavevmode
pd.DataFrame

\end{description}\end{quote}


\sphinxstrong{See also:}


\sphinxcode{\sphinxupquote{PROVIDING\_COLUMNS}} for information of what extra columns for providing BU’s
are being added to the solution matrix.

\sphinxcode{\sphinxupquote{RECEIVING\_COLUMNS}} for information of what extra columns for receiving BU’s
are being added to the solution matrix.

If you want to add new columns to the solution matrix, change both dictionaries accordingly
and the rest of the code will adapt automatically.



\end{fulllineitems}

\index{\_rounding\_columns() (optimization.solspace.SolutionSpace static method)@\spxentry{\_rounding\_columns()}\spxextra{optimization.solspace.SolutionSpace static method}}

\begin{fulllineitems}
\phantomsection\label{\detokenize{source/optimization:optimization.solspace.SolutionSpace._rounding_columns}}\pysiglinewithargsret{\sphinxbfcode{\sphinxupquote{static }}\sphinxbfcode{\sphinxupquote{\_rounding\_columns}}}{\emph{\DUrole{n}{x\_df}}}{}
Round the values of the columns to the nearest 2 \sphinxhyphen{} byte order.
\begin{quote}\begin{description}
\item[{Parameters}] \leavevmode
\sphinxstyleliteralstrong{\sphinxupquote{x\_df}} (\sphinxstyleliteralemphasis{\sphinxupquote{pd.DataFrame}}) \textendash{} Dataframe to be used for performing the pipe transformations.

\item[{Returns}] \leavevmode
\sphinxstylestrong{pd.DataFrame} \textendash{} Transformed dataframe.

\item[{Return type}] \leavevmode
pd.DataFrame

\end{description}\end{quote}

\end{fulllineitems}

\index{filter\_matrix() (optimization.solspace.SolutionSpace method)@\spxentry{filter\_matrix()}\spxextra{optimization.solspace.SolutionSpace method}}

\begin{fulllineitems}
\phantomsection\label{\detokenize{source/optimization:optimization.solspace.SolutionSpace.filter_matrix}}\pysiglinewithargsret{\sphinxbfcode{\sphinxupquote{filter\_matrix}}}{\emph{\DUrole{n}{provider}}, \emph{\DUrole{n}{receiver}}}{}
Apply check to solution matrix to check if problem is optimizable.

This filter runs after we obtain \sphinxcode{\sphinxupquote{bu\_prov}} and \sphinxcode{\sphinxupquote{bu\_rec}} to check if there
are any optimizable BU’s. This check is done by verifying the following conditions:
\begin{itemize}
\item {} 
\sphinxcode{\sphinxupquote{receiver{[}Columns.average\_item\_daily\_use{]}.sum() != 0}} \sphinxhyphen{} If all receiver BU’s have an average consumption rate of zero, the are no possible BU’s that will reduce inventory balance or items to expire

\end{itemize}
\begin{quote}\begin{description}
\item[{Parameters}] \leavevmode\begin{itemize}
\item {} 
\sphinxstyleliteralstrong{\sphinxupquote{provider}} (\sphinxstyleliteralemphasis{\sphinxupquote{{[}}}\sphinxstyleliteralemphasis{\sphinxupquote{type}}\sphinxstyleliteralemphasis{\sphinxupquote{{]}}}) \textendash{} {[}description{]}

\item {} 
\sphinxstyleliteralstrong{\sphinxupquote{receiver}} (\sphinxstyleliteralemphasis{\sphinxupquote{{[}}}\sphinxstyleliteralemphasis{\sphinxupquote{type}}\sphinxstyleliteralemphasis{\sphinxupquote{{]}}}) \textendash{} {[}description{]}

\end{itemize}

\item[{Returns}] \leavevmode
{[}description{]}

\item[{Return type}] \leavevmode
{[}type{]}

\end{description}\end{quote}

\end{fulllineitems}

\index{sol\_matrix() (optimization.solspace.SolutionSpace method)@\spxentry{sol\_matrix()}\spxextra{optimization.solspace.SolutionSpace method}}

\begin{fulllineitems}
\phantomsection\label{\detokenize{source/optimization:optimization.solspace.SolutionSpace.sol_matrix}}\pysiglinewithargsret{\sphinxbfcode{\sphinxupquote{sol\_matrix}}}{}{}
Create the solution matrix.

Based on the item ID of our inventory database, it
creates the matrix that contains in the first column
all the BU’s that can provide items and at the first
row all BU’s that can receive items.
\begin{quote}\begin{description}
\item[{Returns}] \leavevmode
\sphinxstylestrong{sol\_space} \textendash{} Numpy matrix to be used in the optimization model.

\item[{Return type}] \leavevmode
numpy.matrix

\end{description}\end{quote}

\end{fulllineitems}

\index{testing() (optimization.solspace.SolutionSpace method)@\spxentry{testing()}\spxextra{optimization.solspace.SolutionSpace method}}

\begin{fulllineitems}
\phantomsection\label{\detokenize{source/optimization:optimization.solspace.SolutionSpace.testing}}\pysiglinewithargsret{\sphinxbfcode{\sphinxupquote{testing}}}{}{}
\end{fulllineitems}


\end{fulllineitems}



\subsection{Constants}
\label{\detokenize{source/optimization:module-optimization.constants}}\label{\detokenize{source/optimization:constants}}\index{module@\spxentry{module}!optimization.constants@\spxentry{optimization.constants}}\index{optimization.constants@\spxentry{optimization.constants}!module@\spxentry{module}}
Constants used throughout optimization model scripts.
\index{Columns (class in optimization.constants)@\spxentry{Columns}\spxextra{class in optimization.constants}}

\begin{fulllineitems}
\phantomsection\label{\detokenize{source/optimization:optimization.constants.Columns}}\pysigline{\sphinxbfcode{\sphinxupquote{class }}\sphinxcode{\sphinxupquote{optimization.constants.}}\sphinxbfcode{\sphinxupquote{Columns}}}
Bases: \sphinxcode{\sphinxupquote{object}}

Specify inventory input report column names.

Below you can find \sphinxstylestrong{all names of columns} that are either \sphinxstylestrong{created or used} by the model.

\begin{sphinxadmonition}{hint}{Hint:}
If the \sphinxstylestrong{names of the columns used by the model change, there is no need to go through all modules and change manually}.
Just \sphinxstylestrong{add their new name to their respective field.}
\end{sphinxadmonition}

\begin{sphinxadmonition}{tip}{Tip:}
If new columns need to be specified to the model, place them in this script as the other columns.

Place for specifying column names used on Inventory Dataframe.
\end{sphinxadmonition}
\index{average\_item\_daily\_use (optimization.constants.Columns attribute)@\spxentry{average\_item\_daily\_use}\spxextra{optimization.constants.Columns attribute}}

\begin{fulllineitems}
\phantomsection\label{\detokenize{source/optimization:optimization.constants.Columns.average_item_daily_use}}\pysigline{\sphinxbfcode{\sphinxupquote{average\_item\_daily\_use}}\sphinxbfcode{\sphinxupquote{ = \textquotesingle{}average\_item\_daily\_use\textquotesingle{}}}}
\end{fulllineitems}

\index{bu\_address (optimization.constants.Columns attribute)@\spxentry{bu\_address}\spxextra{optimization.constants.Columns attribute}}

\begin{fulllineitems}
\phantomsection\label{\detokenize{source/optimization:optimization.constants.Columns.bu_address}}\pysigline{\sphinxbfcode{\sphinxupquote{bu\_address}}\sphinxbfcode{\sphinxupquote{ = \textquotesingle{}bu\_address\textquotesingle{}}}}
\end{fulllineitems}

\index{bu\_descrip (optimization.constants.Columns attribute)@\spxentry{bu\_descrip}\spxextra{optimization.constants.Columns attribute}}

\begin{fulllineitems}
\phantomsection\label{\detokenize{source/optimization:optimization.constants.Columns.bu_descrip}}\pysigline{\sphinxbfcode{\sphinxupquote{bu\_descrip}}\sphinxbfcode{\sphinxupquote{ = \textquotesingle{}bu\_descrip\textquotesingle{}}}}
\end{fulllineitems}

\index{bu\_doi\_balance (optimization.constants.Columns attribute)@\spxentry{bu\_doi\_balance}\spxextra{optimization.constants.Columns attribute}}

\begin{fulllineitems}
\phantomsection\label{\detokenize{source/optimization:optimization.constants.Columns.bu_doi_balance}}\pysigline{\sphinxbfcode{\sphinxupquote{bu\_doi\_balance}}\sphinxbfcode{\sphinxupquote{ = \textquotesingle{}bu\_doi\_balance\textquotesingle{}}}}
\end{fulllineitems}

\index{bu\_item\_qty\_in\_transf (optimization.constants.Columns attribute)@\spxentry{bu\_item\_qty\_in\_transf}\spxextra{optimization.constants.Columns attribute}}

\begin{fulllineitems}
\phantomsection\label{\detokenize{source/optimization:optimization.constants.Columns.bu_item_qty_in_transf}}\pysigline{\sphinxbfcode{\sphinxupquote{bu\_item\_qty\_in\_transf}}\sphinxbfcode{\sphinxupquote{ = \textquotesingle{}bu\_item\_qty\_in\_transfer\textquotesingle{}}}}
\end{fulllineitems}

\index{bu\_oh (optimization.constants.Columns attribute)@\spxentry{bu\_oh}\spxextra{optimization.constants.Columns attribute}}

\begin{fulllineitems}
\phantomsection\label{\detokenize{source/optimization:optimization.constants.Columns.bu_oh}}\pysigline{\sphinxbfcode{\sphinxupquote{bu\_oh}}\sphinxbfcode{\sphinxupquote{ = \textquotesingle{}bu\_oh\_days\_supply\textquotesingle{}}}}
\end{fulllineitems}

\index{bu\_oh\_plus\_transit (optimization.constants.Columns attribute)@\spxentry{bu\_oh\_plus\_transit}\spxextra{optimization.constants.Columns attribute}}

\begin{fulllineitems}
\phantomsection\label{\detokenize{source/optimization:optimization.constants.Columns.bu_oh_plus_transit}}\pysigline{\sphinxbfcode{\sphinxupquote{bu\_oh\_plus\_transit}}\sphinxbfcode{\sphinxupquote{ = \textquotesingle{}bu\_oh\_transit\_days\_supply\textquotesingle{}}}}
\end{fulllineitems}

\index{bu\_qty\_end\_month (optimization.constants.Columns attribute)@\spxentry{bu\_qty\_end\_month}\spxextra{optimization.constants.Columns attribute}}

\begin{fulllineitems}
\phantomsection\label{\detokenize{source/optimization:optimization.constants.Columns.bu_qty_end_month}}\pysigline{\sphinxbfcode{\sphinxupquote{bu\_qty\_end\_month}}\sphinxbfcode{\sphinxupquote{ = \textquotesingle{}bu\_qty\_end\_month\textquotesingle{}}}}
\end{fulllineitems}

\index{bu\_qty\_on\_hand (optimization.constants.Columns attribute)@\spxentry{bu\_qty\_on\_hand}\spxextra{optimization.constants.Columns attribute}}

\begin{fulllineitems}
\phantomsection\label{\detokenize{source/optimization:optimization.constants.Columns.bu_qty_on_hand}}\pysigline{\sphinxbfcode{\sphinxupquote{bu\_qty\_on\_hand}}\sphinxbfcode{\sphinxupquote{ = \textquotesingle{}bu\_qty\_on\_hand\textquotesingle{}}}}
\end{fulllineitems}

\index{bu\_region (optimization.constants.Columns attribute)@\spxentry{bu\_region}\spxextra{optimization.constants.Columns attribute}}

\begin{fulllineitems}
\phantomsection\label{\detokenize{source/optimization:optimization.constants.Columns.bu_region}}\pysigline{\sphinxbfcode{\sphinxupquote{bu\_region}}\sphinxbfcode{\sphinxupquote{ = \textquotesingle{}bu\_region\textquotesingle{}}}}
\end{fulllineitems}

\index{can\_receive\_inventory (optimization.constants.Columns attribute)@\spxentry{can\_receive\_inventory}\spxextra{optimization.constants.Columns attribute}}

\begin{fulllineitems}
\phantomsection\label{\detokenize{source/optimization:optimization.constants.Columns.can_receive_inventory}}\pysigline{\sphinxbfcode{\sphinxupquote{can\_receive\_inventory}}\sphinxbfcode{\sphinxupquote{ = \textquotesingle{}can\_receive\_inventory\textquotesingle{}}}}
\end{fulllineitems}

\index{can\_transfer\_inventory (optimization.constants.Columns attribute)@\spxentry{can\_transfer\_inventory}\spxextra{optimization.constants.Columns attribute}}

\begin{fulllineitems}
\phantomsection\label{\detokenize{source/optimization:optimization.constants.Columns.can_transfer_inventory}}\pysigline{\sphinxbfcode{\sphinxupquote{can\_transfer\_inventory}}\sphinxbfcode{\sphinxupquote{ = \textquotesingle{}can\_transfer\_inventory\textquotesingle{}}}}
\end{fulllineitems}

\index{chart\_of\_accounts (optimization.constants.Columns attribute)@\spxentry{chart\_of\_accounts}\spxextra{optimization.constants.Columns attribute}}

\begin{fulllineitems}
\phantomsection\label{\detokenize{source/optimization:optimization.constants.Columns.chart_of_accounts}}\pysigline{\sphinxbfcode{\sphinxupquote{chart\_of\_accounts}}\sphinxbfcode{\sphinxupquote{ = \textquotesingle{}chart\_of\_accounts\textquotesingle{}}}}
\end{fulllineitems}

\index{consume (optimization.constants.Columns attribute)@\spxentry{consume}\spxextra{optimization.constants.Columns attribute}}

\begin{fulllineitems}
\phantomsection\label{\detokenize{source/optimization:optimization.constants.Columns.consume}}\pysigline{\sphinxbfcode{\sphinxupquote{consume}}\sphinxbfcode{\sphinxupquote{ = \textquotesingle{}consume\textquotesingle{}}}}
\end{fulllineitems}

\index{contact\_email (optimization.constants.Columns attribute)@\spxentry{contact\_email}\spxextra{optimization.constants.Columns attribute}}

\begin{fulllineitems}
\phantomsection\label{\detokenize{source/optimization:optimization.constants.Columns.contact_email}}\pysigline{\sphinxbfcode{\sphinxupquote{contact\_email}}\sphinxbfcode{\sphinxupquote{ = \textquotesingle{}approver\_contact\_email\textquotesingle{}}}}
\end{fulllineitems}

\index{cum\_doi (optimization.constants.Columns attribute)@\spxentry{cum\_doi}\spxextra{optimization.constants.Columns attribute}}

\begin{fulllineitems}
\phantomsection\label{\detokenize{source/optimization:optimization.constants.Columns.cum_doi}}\pysigline{\sphinxbfcode{\sphinxupquote{cum\_doi}}\sphinxbfcode{\sphinxupquote{ = \textquotesingle{}cumulative\_days\_of\_inventory\textquotesingle{}}}}
\end{fulllineitems}

\index{cum\_doi\_balance (optimization.constants.Columns attribute)@\spxentry{cum\_doi\_balance}\spxextra{optimization.constants.Columns attribute}}

\begin{fulllineitems}
\phantomsection\label{\detokenize{source/optimization:optimization.constants.Columns.cum_doi_balance}}\pysigline{\sphinxbfcode{\sphinxupquote{cum\_doi\_balance}}\sphinxbfcode{\sphinxupquote{ = \textquotesingle{}cumulative\_doi\_balance\textquotesingle{}}}}
\end{fulllineitems}

\index{cum\_items\_to\_expire (optimization.constants.Columns attribute)@\spxentry{cum\_items\_to\_expire}\spxextra{optimization.constants.Columns attribute}}

\begin{fulllineitems}
\phantomsection\label{\detokenize{source/optimization:optimization.constants.Columns.cum_items_to_expire}}\pysigline{\sphinxbfcode{\sphinxupquote{cum\_items\_to\_expire}}\sphinxbfcode{\sphinxupquote{ = \textquotesingle{}cumulative\_items\_to\_expire\textquotesingle{}}}}
\end{fulllineitems}

\index{cum\_stock (optimization.constants.Columns attribute)@\spxentry{cum\_stock}\spxextra{optimization.constants.Columns attribute}}

\begin{fulllineitems}
\phantomsection\label{\detokenize{source/optimization:optimization.constants.Columns.cum_stock}}\pysigline{\sphinxbfcode{\sphinxupquote{cum\_stock}}\sphinxbfcode{\sphinxupquote{ = \textquotesingle{}cumulative\_item\_qty\textquotesingle{}}}}
\end{fulllineitems}

\index{date\_lot\_added\_to\_bu\_inv (optimization.constants.Columns attribute)@\spxentry{date\_lot\_added\_to\_bu\_inv}\spxextra{optimization.constants.Columns attribute}}

\begin{fulllineitems}
\phantomsection\label{\detokenize{source/optimization:optimization.constants.Columns.date_lot_added_to_bu_inv}}\pysigline{\sphinxbfcode{\sphinxupquote{date\_lot\_added\_to\_bu\_inv}}\sphinxbfcode{\sphinxupquote{ = \textquotesingle{}date\_lot\_added\_to\_bu\_inv\textquotesingle{}}}}
\end{fulllineitems}

\index{days\_of\_inventory (optimization.constants.Columns attribute)@\spxentry{days\_of\_inventory}\spxextra{optimization.constants.Columns attribute}}

\begin{fulllineitems}
\phantomsection\label{\detokenize{source/optimization:optimization.constants.Columns.days_of_inventory}}\pysigline{\sphinxbfcode{\sphinxupquote{days\_of\_inventory}}\sphinxbfcode{\sphinxupquote{ = \textquotesingle{}days\_of\_inventory\textquotesingle{}}}}
\end{fulllineitems}

\index{days\_to\_expire (optimization.constants.Columns attribute)@\spxentry{days\_to\_expire}\spxextra{optimization.constants.Columns attribute}}

\begin{fulllineitems}
\phantomsection\label{\detokenize{source/optimization:optimization.constants.Columns.days_to_expire}}\pysigline{\sphinxbfcode{\sphinxupquote{days\_to\_expire}}\sphinxbfcode{\sphinxupquote{ = \textquotesingle{}days\_to\_expire\textquotesingle{}}}}
\end{fulllineitems}

\index{default\_shipment\_days (optimization.constants.Columns attribute)@\spxentry{default\_shipment\_days}\spxextra{optimization.constants.Columns attribute}}

\begin{fulllineitems}
\phantomsection\label{\detokenize{source/optimization:optimization.constants.Columns.default_shipment_days}}\pysigline{\sphinxbfcode{\sphinxupquote{default\_shipment\_days}}\sphinxbfcode{\sphinxupquote{ = \textquotesingle{}default\_shipment\_days\textquotesingle{}}}}
\end{fulllineitems}

\index{delta\_days\_of\_inventory (optimization.constants.Columns attribute)@\spxentry{delta\_days\_of\_inventory}\spxextra{optimization.constants.Columns attribute}}

\begin{fulllineitems}
\phantomsection\label{\detokenize{source/optimization:optimization.constants.Columns.delta_days_of_inventory}}\pysigline{\sphinxbfcode{\sphinxupquote{delta\_days\_of\_inventory}}\sphinxbfcode{\sphinxupquote{ = \textquotesingle{}delta\_days\_of\_inventory\textquotesingle{}}}}
\end{fulllineitems}

\index{delta\_doi (optimization.constants.Columns attribute)@\spxentry{delta\_doi}\spxextra{optimization.constants.Columns attribute}}

\begin{fulllineitems}
\phantomsection\label{\detokenize{source/optimization:optimization.constants.Columns.delta_doi}}\pysigline{\sphinxbfcode{\sphinxupquote{delta\_doi}}\sphinxbfcode{\sphinxupquote{ = \textquotesingle{}delta\_doi\textquotesingle{}}}}
\end{fulllineitems}

\index{doi\_balance (optimization.constants.Columns attribute)@\spxentry{doi\_balance}\spxextra{optimization.constants.Columns attribute}}

\begin{fulllineitems}
\phantomsection\label{\detokenize{source/optimization:optimization.constants.Columns.doi_balance}}\pysigline{\sphinxbfcode{\sphinxupquote{doi\_balance}}\sphinxbfcode{\sphinxupquote{ = \textquotesingle{}doi\_balance\textquotesingle{}}}}
\end{fulllineitems}

\index{doi\_target (optimization.constants.Columns attribute)@\spxentry{doi\_target}\spxextra{optimization.constants.Columns attribute}}

\begin{fulllineitems}
\phantomsection\label{\detokenize{source/optimization:optimization.constants.Columns.doi_target}}\pysigline{\sphinxbfcode{\sphinxupquote{doi\_target}}\sphinxbfcode{\sphinxupquote{ = \textquotesingle{}doi\_target\textquotesingle{}}}}
\end{fulllineitems}

\index{expire\_date (optimization.constants.Columns attribute)@\spxentry{expire\_date}\spxextra{optimization.constants.Columns attribute}}

\begin{fulllineitems}
\phantomsection\label{\detokenize{source/optimization:optimization.constants.Columns.expire_date}}\pysigline{\sphinxbfcode{\sphinxupquote{expire\_date}}\sphinxbfcode{\sphinxupquote{ = \textquotesingle{}expire\_date\textquotesingle{}}}}
\end{fulllineitems}

\index{flag\_avg\_daily\_use (optimization.constants.Columns attribute)@\spxentry{flag\_avg\_daily\_use}\spxextra{optimization.constants.Columns attribute}}

\begin{fulllineitems}
\phantomsection\label{\detokenize{source/optimization:optimization.constants.Columns.flag_avg_daily_use}}\pysigline{\sphinxbfcode{\sphinxupquote{flag\_avg\_daily\_use}}\sphinxbfcode{\sphinxupquote{ = \textquotesingle{}flag\_avg\_daily\_use\textquotesingle{}}}}
\end{fulllineitems}

\index{flag\_lot (optimization.constants.Columns attribute)@\spxentry{flag\_lot}\spxextra{optimization.constants.Columns attribute}}

\begin{fulllineitems}
\phantomsection\label{\detokenize{source/optimization:optimization.constants.Columns.flag_lot}}\pysigline{\sphinxbfcode{\sphinxupquote{flag\_lot}}\sphinxbfcode{\sphinxupquote{ = \textquotesingle{}flag\_non\_lot\textquotesingle{}}}}
\end{fulllineitems}

\index{how\_many\_bu (optimization.constants.Columns attribute)@\spxentry{how\_many\_bu}\spxextra{optimization.constants.Columns attribute}}

\begin{fulllineitems}
\phantomsection\label{\detokenize{source/optimization:optimization.constants.Columns.how_many_bu}}\pysigline{\sphinxbfcode{\sphinxupquote{how\_many\_bu}}\sphinxbfcode{\sphinxupquote{ = \textquotesingle{}how\_many\_bu\textquotesingle{}}}}
\end{fulllineitems}

\index{inv\_bu (optimization.constants.Columns attribute)@\spxentry{inv\_bu}\spxextra{optimization.constants.Columns attribute}}

\begin{fulllineitems}
\phantomsection\label{\detokenize{source/optimization:optimization.constants.Columns.inv_bu}}\pysigline{\sphinxbfcode{\sphinxupquote{inv\_bu}}\sphinxbfcode{\sphinxupquote{ = \textquotesingle{}inv\_bu\textquotesingle{}}}}
\end{fulllineitems}

\index{item\_descrip (optimization.constants.Columns attribute)@\spxentry{item\_descrip}\spxextra{optimization.constants.Columns attribute}}

\begin{fulllineitems}
\phantomsection\label{\detokenize{source/optimization:optimization.constants.Columns.item_descrip}}\pysigline{\sphinxbfcode{\sphinxupquote{item\_descrip}}\sphinxbfcode{\sphinxupquote{ = \textquotesingle{}item\_descrip\textquotesingle{}}}}
\end{fulllineitems}

\index{item\_id (optimization.constants.Columns attribute)@\spxentry{item\_id}\spxextra{optimization.constants.Columns attribute}}

\begin{fulllineitems}
\phantomsection\label{\detokenize{source/optimization:optimization.constants.Columns.item_id}}\pysigline{\sphinxbfcode{\sphinxupquote{item\_id}}\sphinxbfcode{\sphinxupquote{ = \textquotesingle{}item\_id\textquotesingle{}}}}
\end{fulllineitems}

\index{item\_stats (optimization.constants.Columns attribute)@\spxentry{item\_stats}\spxextra{optimization.constants.Columns attribute}}

\begin{fulllineitems}
\phantomsection\label{\detokenize{source/optimization:optimization.constants.Columns.item_stats}}\pysigline{\sphinxbfcode{\sphinxupquote{item\_stats}}\sphinxbfcode{\sphinxupquote{ = \textquotesingle{}bu\_item\_status\textquotesingle{}}}}
\end{fulllineitems}

\index{items\_to\_expire (optimization.constants.Columns attribute)@\spxentry{items\_to\_expire}\spxextra{optimization.constants.Columns attribute}}

\begin{fulllineitems}
\phantomsection\label{\detokenize{source/optimization:optimization.constants.Columns.items_to_expire}}\pysigline{\sphinxbfcode{\sphinxupquote{items\_to\_expire}}\sphinxbfcode{\sphinxupquote{ = \textquotesingle{}items\_to\_expire\textquotesingle{}}}}
\end{fulllineitems}

\index{lot\_id (optimization.constants.Columns attribute)@\spxentry{lot\_id}\spxextra{optimization.constants.Columns attribute}}

\begin{fulllineitems}
\phantomsection\label{\detokenize{source/optimization:optimization.constants.Columns.lot_id}}\pysigline{\sphinxbfcode{\sphinxupquote{lot\_id}}\sphinxbfcode{\sphinxupquote{ = \textquotesingle{}lot\_id\textquotesingle{}}}}
\end{fulllineitems}

\index{lot\_qty\_on\_hand (optimization.constants.Columns attribute)@\spxentry{lot\_qty\_on\_hand}\spxextra{optimization.constants.Columns attribute}}

\begin{fulllineitems}
\phantomsection\label{\detokenize{source/optimization:optimization.constants.Columns.lot_qty_on_hand}}\pysigline{\sphinxbfcode{\sphinxupquote{lot\_qty\_on\_hand}}\sphinxbfcode{\sphinxupquote{ = \textquotesingle{}lot\_qty\_on\_hand\textquotesingle{}}}}
\end{fulllineitems}

\index{min\_shipment\_value (optimization.constants.Columns attribute)@\spxentry{min\_shipment\_value}\spxextra{optimization.constants.Columns attribute}}

\begin{fulllineitems}
\phantomsection\label{\detokenize{source/optimization:optimization.constants.Columns.min_shipment_value}}\pysigline{\sphinxbfcode{\sphinxupquote{min\_shipment\_value}}\sphinxbfcode{\sphinxupquote{ = \textquotesingle{}min\_shipment\_value\textquotesingle{}}}}
\end{fulllineitems}

\index{minimum\_days\_of\_inventory\_for\_lot (optimization.constants.Columns attribute)@\spxentry{minimum\_days\_of\_inventory\_for\_lot}\spxextra{optimization.constants.Columns attribute}}

\begin{fulllineitems}
\phantomsection\label{\detokenize{source/optimization:optimization.constants.Columns.minimum_days_of_inventory_for_lot}}\pysigline{\sphinxbfcode{\sphinxupquote{minimum\_days\_of\_inventory\_for\_lot}}\sphinxbfcode{\sphinxupquote{ = \textquotesingle{}minimum\_days\_of\_inventory\_for\_lot\textquotesingle{}}}}
\end{fulllineitems}

\index{on\_site\_email (optimization.constants.Columns attribute)@\spxentry{on\_site\_email}\spxextra{optimization.constants.Columns attribute}}

\begin{fulllineitems}
\phantomsection\label{\detokenize{source/optimization:optimization.constants.Columns.on_site_email}}\pysigline{\sphinxbfcode{\sphinxupquote{on\_site\_email}}\sphinxbfcode{\sphinxupquote{ = \textquotesingle{}onsite\_contact\_email\textquotesingle{}}}}
\end{fulllineitems}

\index{optimization\_transfer (optimization.constants.Columns attribute)@\spxentry{optimization\_transfer}\spxextra{optimization.constants.Columns attribute}}

\begin{fulllineitems}
\phantomsection\label{\detokenize{source/optimization:optimization.constants.Columns.optimization_transfer}}\pysigline{\sphinxbfcode{\sphinxupquote{optimization\_transfer}}\sphinxbfcode{\sphinxupquote{ = \textquotesingle{}optimization\_transfer\textquotesingle{}}}}
\end{fulllineitems}

\index{period (optimization.constants.Columns attribute)@\spxentry{period}\spxextra{optimization.constants.Columns attribute}}

\begin{fulllineitems}
\phantomsection\label{\detokenize{source/optimization:optimization.constants.Columns.period}}\pysigline{\sphinxbfcode{\sphinxupquote{period}}\sphinxbfcode{\sphinxupquote{ = \textquotesingle{}period\textquotesingle{}}}}
\end{fulllineitems}

\index{price (optimization.constants.Columns attribute)@\spxentry{price}\spxextra{optimization.constants.Columns attribute}}

\begin{fulllineitems}
\phantomsection\label{\detokenize{source/optimization:optimization.constants.Columns.price}}\pysigline{\sphinxbfcode{\sphinxupquote{price}}\sphinxbfcode{\sphinxupquote{ = \textquotesingle{}price\textquotesingle{}}}}
\end{fulllineitems}

\index{provider\_bu (optimization.constants.Columns attribute)@\spxentry{provider\_bu}\spxextra{optimization.constants.Columns attribute}}

\begin{fulllineitems}
\phantomsection\label{\detokenize{source/optimization:optimization.constants.Columns.provider_bu}}\pysigline{\sphinxbfcode{\sphinxupquote{provider\_bu}}\sphinxbfcode{\sphinxupquote{ = \textquotesingle{}provider\_bu\textquotesingle{}}}}
\end{fulllineitems}

\index{real\_avg\_consumption (optimization.constants.Columns attribute)@\spxentry{real\_avg\_consumption}\spxextra{optimization.constants.Columns attribute}}

\begin{fulllineitems}
\phantomsection\label{\detokenize{source/optimization:optimization.constants.Columns.real_avg_consumption}}\pysigline{\sphinxbfcode{\sphinxupquote{real\_avg\_consumption}}\sphinxbfcode{\sphinxupquote{ = \textquotesingle{}real\_avg\_consumption\textquotesingle{}}}}
\end{fulllineitems}

\index{receiver\_bu (optimization.constants.Columns attribute)@\spxentry{receiver\_bu}\spxextra{optimization.constants.Columns attribute}}

\begin{fulllineitems}
\phantomsection\label{\detokenize{source/optimization:optimization.constants.Columns.receiver_bu}}\pysigline{\sphinxbfcode{\sphinxupquote{receiver\_bu}}\sphinxbfcode{\sphinxupquote{ = \textquotesingle{}receiver\_bu\textquotesingle{}}}}
\end{fulllineitems}

\index{report\_date (optimization.constants.Columns attribute)@\spxentry{report\_date}\spxextra{optimization.constants.Columns attribute}}

\begin{fulllineitems}
\phantomsection\label{\detokenize{source/optimization:optimization.constants.Columns.report_date}}\pysigline{\sphinxbfcode{\sphinxupquote{report\_date}}\sphinxbfcode{\sphinxupquote{ = \textquotesingle{}report\_date\textquotesingle{}}}}
\end{fulllineitems}

\index{std\_uom (optimization.constants.Columns attribute)@\spxentry{std\_uom}\spxextra{optimization.constants.Columns attribute}}

\begin{fulllineitems}
\phantomsection\label{\detokenize{source/optimization:optimization.constants.Columns.std_uom}}\pysigline{\sphinxbfcode{\sphinxupquote{std\_uom}}\sphinxbfcode{\sphinxupquote{ = \textquotesingle{}std\_uom\textquotesingle{}}}}
\end{fulllineitems}

\index{supplier\_name (optimization.constants.Columns attribute)@\spxentry{supplier\_name}\spxextra{optimization.constants.Columns attribute}}

\begin{fulllineitems}
\phantomsection\label{\detokenize{source/optimization:optimization.constants.Columns.supplier_name}}\pysigline{\sphinxbfcode{\sphinxupquote{supplier\_name}}\sphinxbfcode{\sphinxupquote{ = \textquotesingle{}supplier\_name\textquotesingle{}}}}
\end{fulllineitems}

\index{temp\_avg\_daily\_use (optimization.constants.Columns attribute)@\spxentry{temp\_avg\_daily\_use}\spxextra{optimization.constants.Columns attribute}}

\begin{fulllineitems}
\phantomsection\label{\detokenize{source/optimization:optimization.constants.Columns.temp_avg_daily_use}}\pysigline{\sphinxbfcode{\sphinxupquote{temp\_avg\_daily\_use}}\sphinxbfcode{\sphinxupquote{ = \textquotesingle{}temp\_avg\_daily\_use\textquotesingle{}}}}
\end{fulllineitems}


\end{fulllineitems}



\subsection{Debug}
\label{\detokenize{source/optimization:module-optimization.debug}}\label{\detokenize{source/optimization:debug}}\index{module@\spxentry{module}!optimization.debug@\spxentry{optimization.debug}}\index{optimization.debug@\spxentry{optimization.debug}!module@\spxentry{module}}
Debug scripts.
\index{Debugger (class in optimization.debug)@\spxentry{Debugger}\spxextra{class in optimization.debug}}

\begin{fulllineitems}
\phantomsection\label{\detokenize{source/optimization:optimization.debug.Debugger}}\pysiglinewithargsret{\sphinxbfcode{\sphinxupquote{class }}\sphinxcode{\sphinxupquote{optimization.debug.}}\sphinxbfcode{\sphinxupquote{Debugger}}}{\emph{\DUrole{n}{func}}}{}
Bases: \sphinxcode{\sphinxupquote{object}}
\index{enabled (optimization.debug.Debugger attribute)@\spxentry{enabled}\spxextra{optimization.debug.Debugger attribute}}

\begin{fulllineitems}
\phantomsection\label{\detokenize{source/optimization:optimization.debug.Debugger.enabled}}\pysigline{\sphinxbfcode{\sphinxupquote{enabled}}\sphinxbfcode{\sphinxupquote{ = False}}}
\end{fulllineitems}


\end{fulllineitems}



\subsection{Module contents}
\label{\detokenize{source/optimization:module-optimization}}\label{\detokenize{source/optimization:module-contents}}\index{module@\spxentry{module}!optimization@\spxentry{optimization}}\index{optimization@\spxentry{optimization}!module@\spxentry{module}}
Optimization model scripts for generating transfer recommendations of SKU’s between Quest Diagnostics business units.


\subsubsection{Submodules}
\label{\detokenize{source/optimization:submodules}}

\begin{savenotes}\sphinxatlongtablestart\begin{longtable}[c]{\X{1}{2}\X{1}{2}}
\hline

\endfirsthead

\multicolumn{2}{c}%
{\makebox[0pt]{\sphinxtablecontinued{\tablename\ \thetable{} \textendash{} continued from previous page}}}\\
\hline

\endhead

\hline
\multicolumn{2}{r}{\makebox[0pt][r]{\sphinxtablecontinued{continues on next page}}}\\
\endfoot

\endlastfoot

{\hyperref[\detokenize{source/optimization:module-optimization.solspace}]{\sphinxcrossref{\sphinxcode{\sphinxupquote{solspace}}}}}
&
Scripts used for generating the \sphinxstylestrong{solution space.}
\\
\hline
{\hyperref[\detokenize{source/optimization:module-optimization.constants}]{\sphinxcrossref{\sphinxcode{\sphinxupquote{constants}}}}}
&
Constants used throughout optimization model scripts.
\\
\hline
{\hyperref[\detokenize{source/optimization:module-optimization.debug}]{\sphinxcrossref{\sphinxcode{\sphinxupquote{debug}}}}}
&
Debug scripts.
\\
\hline
\end{longtable}\sphinxatlongtableend\end{savenotes}


\section{User Guides}
\label{\detokenize{source/guides:user-guides}}\label{\detokenize{source/guides::doc}}

\subsection{Model Installation}
\label{\detokenize{source/guides:model-installation}}
To install the {\hyperref[\detokenize{source/optimization:module-optimization}]{\sphinxcrossref{\sphinxcode{\sphinxupquote{optimization}}}}} package locally, first clone the source code available at the project \sphinxstylestrong{gitlab} page.
After clonning the repository, you can choose either to run it \sphinxstylestrong{with or without} adding the package to your python environment \sphinxcode{\sphinxupquote{site\sphinxhyphen{}packages}} folder.

\begin{sphinxadmonition}{attention}{Attention:}
If you choose \sphinxstylestrong{not} to install the package at your python environment, you will have to reference the folder where you stored the source
code every time you try importing the {\hyperref[\detokenize{source/optimization:module-optimization}]{\sphinxcrossref{\sphinxcode{\sphinxupquote{optimization}}}}} modules and submodules.
\end{sphinxadmonition}


\subsubsection{Installing Using Conda Environment}
\label{\detokenize{source/guides:installing-using-conda-environment}}

\paragraph{Activate Conda Environment}
\label{\detokenize{source/guides:activate-conda-environment}}
To install install the optimization package to your \sphinxcode{\sphinxupquote{conda}} environment, first activate the conda environment by running the Anaconda app or by adding
the following command at your \sphinxcode{\sphinxupquote{command prompt}}:
\sphinxSetupCaptionForVerbatim{cmd.exe}
\def\sphinxLiteralBlockLabel{\label{\detokenize{source/guides:id1}}}
\begin{sphinxVerbatim}[commandchars=\\\{\},numbers=left,firstnumber=1,stepnumber=1]
\PYG{g+go}{conda activate base}
\end{sphinxVerbatim}

\begin{sphinxadmonition}{note}{Note:}
Change the name \sphinxcode{\sphinxupquote{base}} to the respective name of your conda environment. If you don’t know the names of your conda environments you can find
them by running the following command:
\begin{quote}
\sphinxSetupCaptionForVerbatim{cmd.exe}
\def\sphinxLiteralBlockLabel{\label{\detokenize{source/guides:id2}}}
\begin{sphinxVerbatim}[commandchars=\\\{\},numbers=left,firstnumber=1,stepnumber=1]
\PYG{g+go}{\PYGZti{}  conda info \PYGZhy{}\PYGZhy{}envs                               ok | base py | at 00:09:21 }
\PYG{g+gp}{\PYGZsh{}} conda environments:
\PYG{g+gp}{\PYGZsh{}}
\PYG{g+go}{EY\PYGZhy{}Quest\PYGZhy{}Diagnostics        /Users/EY/.conda/envs/EY\PYGZhy{}Quest\PYGZhy{}Diagnostics}
\PYG{g+go}{                            /Users/EY/opt/anaconda3}
\PYG{g+go}{base                  *     /opt/anaconda3}
\end{sphinxVerbatim}
\end{quote}
\end{sphinxadmonition}


\paragraph{Go to Optimization Source Code Folder}
\label{\detokenize{source/guides:go-to-optimization-source-code-folder}}
After activating your conda environment, go to the folder where you placed the source code:
\sphinxSetupCaptionForVerbatim{cmd.exe}
\def\sphinxLiteralBlockLabel{\label{\detokenize{source/guides:id3}}}
\begin{sphinxVerbatim}[commandchars=\\\{\},numbers=left,firstnumber=1,stepnumber=1]
\PYG{g+go}{\PYGZti{}  cd /Users/EY/Desktop/EY\PYGZhy{}Quest\PYGZhy{}Diagnostics}
\end{sphinxVerbatim}

There you should find a file named \sphinxcode{\sphinxupquote{setup.py}}.
\sphinxSetupCaptionForVerbatim{cmd.exe}
\def\sphinxLiteralBlockLabel{\label{\detokenize{source/guides:id4}}}
\fvset{hllines={, 3,}}%
\begin{sphinxVerbatim}[commandchars=\\\{\},numbers=left,firstnumber=1,stepnumber=1]
\PYG{g+go}{\PYGZti{}/De/EY\PYGZhy{}Quest\PYGZhy{}Diagnostics ls}
\PYG{g+go}{build                 docs                  optimization.egg\PYGZhy{}info}
\PYG{g+go}{dist                  optimization          setup.py}
\end{sphinxVerbatim}
\sphinxresetverbatimhllines

The file \sphinxcode{\sphinxupquote{setup.py}} is used to store all necessary package configurations that are needed to run the optimization model.


\paragraph{Pip Install Package}
\label{\detokenize{source/guides:pip-install-package}}
After completing the previous steps \sphinxcode{\sphinxupquote{pip install}} the package by running the following command:
\sphinxSetupCaptionForVerbatim{cmd.exe}
\def\sphinxLiteralBlockLabel{\label{\detokenize{source/guides:id5}}}
\begin{sphinxVerbatim}[commandchars=\\\{\},numbers=left,firstnumber=1,stepnumber=1]
\PYGZti{}/De/EY\PYGZhy{}Quest\PYGZhy{}Diagnostics pip install \PYGZhy{}e .

Obtaining file:///Users/EY/Desktop/EY\PYGZhy{}Quest\PYGZhy{}Diagnostics
Installing collected packages: optimization
    Attempting uninstall: optimization
        Found existing installation: optimization \PYG{l+m}{0}.1.3
        Uninstalling optimization\PYGZhy{}0.1.3:
        Successfully uninstalled optimization\PYGZhy{}0.1.3
    Running setup.py develop \PYG{k}{for} optimization
Successfully installed optimization
\end{sphinxVerbatim}

After installation is complete you can test the model by running the \sphinxcode{\sphinxupquote{optimization/test\_py}} script:

\begin{sphinxVerbatim}[commandchars=\\\{\},numbers=left,firstnumber=1,stepnumber=1]
\PYGZti{}/De/EY\PYGZhy{}Quest\PYGZhy{}Diagnostics/optimization  python test\PYGZus{}model.py
\PYGZgt{}\PYGZgt{}\PYGZgt{} 100\PYGZpc{}|███████████████████████████████████████| 7297/7297 [10:59\PYGZlt{}00:00, 21.90it/s]
\end{sphinxVerbatim}


\subsubsection{Running Locally Without Installing the Model}
\label{\detokenize{source/guides:running-locally-without-installing-the-model}}
If you don’t want to add the package to your Python environment \sphinxcode{\sphinxupquote{site\sphinxhyphen{}packages}} folder, just place any testing scripts you created inside the \sphinxcode{\sphinxupquote{/optimization}} folder.
This process is necessary since when importing modules, Python first searches the local folder where your scripts are stored before trying to find them in the \sphinxcode{\sphinxupquote{site\sphinxhyphen{}packages}} folder.

\begin{sphinxadmonition}{important}{Important:}
The {\hyperref[\detokenize{source/optimization.model:module-optimization.model}]{\sphinxcrossref{\sphinxcode{\sphinxupquote{optimization.model}}}}} package requires other modules in order to run. If these packages are not found at your Python environment,
the first time you import any of the optimization model modules, Python will inform that they are missing and ask if you want to download and install them.
\begin{quote}

\begin{sphinxVerbatim}[commandchars=\\\{\}]
\PYG{k+kn}{import} \PYG{n+nn}{optimization}
\end{sphinxVerbatim}

\begin{sphinxVerbatim}[commandchars=\\\{\}]
\PYG{g+gp}{\PYGZgt{}\PYGZgt{}\PYGZgt{} }\PYG{l+s+s2}{\PYGZdq{}}\PYG{l+s+s2}{Pulp package not found. To install it, type [}\PYG{l+s+s2}{\PYGZsq{}}\PYG{l+s+s2}{Y}\PYG{l+s+s2}{\PYGZsq{}}\PYG{l+s+s2}{]/N:  Y}\PYG{l+s+s2}{\PYGZdq{}}
\end{sphinxVerbatim}
\end{quote}
\end{sphinxadmonition}


\chapter{Indices and tables}
\label{\detokenize{index:indices-and-tables}}\begin{itemize}
\item {} 
\DUrole{xref,std,std-ref}{genindex}

\item {} 
\DUrole{xref,std,std-ref}{modindex}

\item {} 
\DUrole{xref,std,std-ref}{search}

\end{itemize}


\renewcommand{\indexname}{Python Module Index}
\begin{sphinxtheindex}
\let\bigletter\sphinxstyleindexlettergroup
\bigletter{o}
\item\relax\sphinxstyleindexentry{optimization}\sphinxstyleindexpageref{source/optimization:\detokenize{module-optimization}}
\item\relax\sphinxstyleindexentry{optimization.constants}\sphinxstyleindexpageref{source/optimization:\detokenize{module-optimization.constants}}
\item\relax\sphinxstyleindexentry{optimization.datatools}\sphinxstyleindexpageref{source/optimization.datatools:\detokenize{module-optimization.datatools}}
\item\relax\sphinxstyleindexentry{optimization.datatools.dataprep}\sphinxstyleindexpageref{source/optimization.datatools:\detokenize{module-optimization.datatools.dataprep}}
\item\relax\sphinxstyleindexentry{optimization.datatools.extra\_output}\sphinxstyleindexpageref{source/optimization.datatools:\detokenize{module-optimization.datatools.extra_output}}
\item\relax\sphinxstyleindexentry{optimization.datatools.pipelines}\sphinxstyleindexpageref{source/optimization.datatools:\detokenize{module-optimization.datatools.pipelines}}
\item\relax\sphinxstyleindexentry{optimization.debug}\sphinxstyleindexpageref{source/optimization:\detokenize{module-optimization.debug}}
\item\relax\sphinxstyleindexentry{optimization.model}\sphinxstyleindexpageref{source/optimization.model:\detokenize{module-optimization.model}}
\item\relax\sphinxstyleindexentry{optimization.model.main}\sphinxstyleindexpageref{source/optimization.model:\detokenize{module-optimization.model.main}}
\item\relax\sphinxstyleindexentry{optimization.model.optimizer}\sphinxstyleindexpageref{source/optimization.model:\detokenize{module-optimization.model.optimizer}}
\item\relax\sphinxstyleindexentry{optimization.opt\_tools}\sphinxstyleindexpageref{source/optimization.opt_tools:\detokenize{module-optimization.opt_tools}}
\item\relax\sphinxstyleindexentry{optimization.opt\_tools.aux\_funcs}\sphinxstyleindexpageref{source/optimization.opt_tools:\detokenize{module-optimization.opt_tools.aux_funcs}}
\item\relax\sphinxstyleindexentry{optimization.opt\_tools.load\_data}\sphinxstyleindexpageref{source/optimization.opt_tools:\detokenize{module-optimization.opt_tools.load_data}}
\item\relax\sphinxstyleindexentry{optimization.solspace}\sphinxstyleindexpageref{source/optimization:\detokenize{module-optimization.solspace}}
\end{sphinxtheindex}

\renewcommand{\indexname}{Index}
\printindex
\end{document}